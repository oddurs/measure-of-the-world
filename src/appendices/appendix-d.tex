\chapter{Visiting Greenwich: A Practical Guide}
\label{app:visiting-greenwich}

This appendix provides practical information for astronomers, students, and visitors interested in Greenwich Observatory and related sites. The National Maritime Museum at Greenwich maintains both historical instruments and contemporary exhibits on the history of timekeeping and navigation.

\section{D.1 Getting There}

Greenwich Observatory, now part of the National Maritime Museum, is located in southeast London at $51°28'40''$ N, $0°00'05''$ W (near the Prime Meridian). The site is accessible via multiple transport routes:

\textbf{Public Transport}:
\begin{enumerate*}[label=(\roman*)]
\item Tube: Cutty Sark for Maritime Greenwich station (Jubilee Line) or Bank/Monument (Circle, District, Northern lines, then walk via Thames path)
\item Train: Mainline services to London Bridge or Cannon Street stations; connect to Jubilee Line
\item Riverboat: Thames Clipper service from central London piers (summer months)
\end{enumerate*}

\noindent\textbf{Driving}: National Car Park at Greenwich with public parking available. Walking distance ($\sim$10 minutes) from Greenwich town center along the Thames.

\noindent\textbf{From Major Airports}: Gatwick (45 min via rail + tube), Stansted (60 min), Luton (70 min).

\section{D.2 Royal Observatory Site}

The hill at Greenwich park offers panoramic views of London and the Thames. The Observatory itself occupies two main structures:

\textbf{Flamsteed House} (1675, Sir Christopher Wren, architect):
\begin{itemize}
\item Original observing room with historic 17th-century installations
\item Now a museum displaying period instruments and documents from Flamsteed era
\item Limited access (reserved tours); contact museum for scheduling
\end{itemize}

\textbf{Meridian Building} (1884, reconstructed 1950s):
\begin{itemize}
\item Houses the Prime Meridian (0° longitude reference)
\item Contains replica of Airy transit circle (mounted in working position)
\item Open to public; visitors can photograph at the meridian line
\item Gift shop and exhibition space
\end{itemize}

\textbf{The Prime Meridian Line}:
The famous green laser line projected at night indicates the 0° longitude reference, established by the International Meridian Conference (1884). The physical meridian marker is engraved on the ground, running north-south through Meridian Building courtyard. Visitors can straddle the hemispheres (Eastern hemisphere on left, Western on right, by convention).

\section{D.3 National Maritime Museum}

Immediately adjacent to the Observatory, the museum maintains extensive collections related to navigation, timekeeping, and astronomy:

\textbf{Maritime Galleries}:
\begin{itemize}
\item Navigation and timekeeping instruments (sextants, chronometers, tide predictors)
\item Ship models from 16th--20th centuries
\item Charts and navigation manuals
\item Personal effects of famous navigators (HMS Endeavour, Franklin Expedition, etc.)
\end{itemize}

\textbf{Astronomy and Time Collection}:
\begin{itemize}
\item Harrison's marine chronometers (H1--H5, originals in specially climate-controlled display)
\item Historical telescopes and transit instruments (Grubb Great Equatorial replica components)
\item Photographs and archival documents on Greenwich Observatory history
\item Detailed exhibits on the development of celestial mechanics and astronomical theory
\end{itemize}

\textbf{Interactive Galleries}:
\begin{itemize}
\item Working orrery (mechanical model of solar system)
\item Hands-on chronometer and latitude-determination demonstrations
\item Planetarium theater (seasonal shows on celestial navigation and astronomical history)
\end{itemize}

\noindent\textbf{Admission}: Generally free to museum galleries; small charge for special exhibitions and planetarium shows. Combined pass available for Observatory + Museum.

\section{D.4 Astronomy Centre and Educational Programs}

The museum hosts ongoing educational initiatives for students and researchers:

\textbf{Courses and Workshops}:
\begin{itemize}
\item Celestial navigation certificate program (multi-week course on practical astronomy)
\item History of astronomy seminars (led by museum curators and visiting scholars)
\item Hands-on chronometer and timekeeping workshops
\end{itemize}

\textbf{Research Access}:
\begin{itemize}
\item Archive of Greenwich Observatory records (1675--1960s) available by appointment
\item Photographic plate collection (60,000+ historical star positions) digitized and searchable
\item Astronomical observation records accessible to scholars
\item Contact: National Maritime Museum Archive Department
\end{itemize}

\textbf{Public Observing Events}:
\begin{itemize}
\item Seasonal evening viewing programs (weather permitting)
\item Solar observation events (with proper filters)
\item Lunar observation nights (especially near full moon)
\item Check museum website for current schedule
\end{itemize}

\section{D.5 For the Mathematically Inclined Visitor}

Those interested in the technical history of timekeeping and positional astronomy may wish to focus on specific exhibits:

\textbf{Recommended Exhibits}:
\begin{enumerate}
\item \textbf{Harrison Chronometers} (Maritime Museum): Study the mechanical solutions to longitude determination. The display includes working models demonstrating the bimetallic strip temperature compensation and grasshopper escapement.
\item \textbf{Airy Transit Circle Replica} (Observatory): The working replica demonstrates the optical principles of meridian instruments. The micrometer screw (linear motion translated to angular rotation) shows precision engineering of the 19th century.
\item \textbf{Photographic Zenith Tube Display} (Museum archive section): Learn how 20th-century observatories automated position measurement via photographic plate analysis.
\item \textbf{Earth Orientation Exhibits} (Observatory): Displays on precession, nutation, and Chandler wobble with interactive models showing 18.6-year nutation cycle and 41,000-year precession cycle.
\item \textbf{Time Standards Timeline} (Museum): Visual representation of transition from solar time → mean solar time → sidereal time → atomic time, with equations and historical context.
\end{enumerate}

\textbf{Suggested Mathematical Deep-Dives}:
\begin{itemize}
\item Request access to exhibits on spherical trigonometry applications (Bradley's aberration measurement, Flamsteed's star catalog reduction)
\item Ask curators about photographic plate measurement techniques (least-squares fitting of stellar positions)
\item Review technical documentation on Airy's error analysis methods (standard deviation calculations, systematic vs. random error separation)
\item Study the mathematics of chronometer testing (statistical analysis of rate stability over weeks-long trials)
\end{itemize}

\section{D.6 Beyond Greenwich: Related Sites and Resources}

\textbf{Nearby Institutions}:
\begin{itemize}
\item \textbf{Queen Mary, University of London} (5 miles): Department of Astronomy; seminars and colloquia open to public
\item \textbf{University of Cambridge, Institute of Astronomy} (50 miles): Summer school programs in astronomical history; archive containing copies of Bradley-era observation records
\item \textbf{Oxford University, History of Science Museum} (60 miles): Instruments from Oxford's medieval astronomy programs; exhibits on Halley and Bradley's Oxford connections
\end{itemize}

\textbf{Digital Resources}:
\begin{itemize}
\item \textbf{International Astronomical Union (IAU)}: Technical documentation on celestial coordinate systems, precession/nutation models, Earth orientation parameters
\item \textbf{SOFA (Standards of Fundamental Astronomy) Library}: Free software implementing SOFA C/FORTRAN algorithms for astronomical calculations (transformations, time scales, etc.); includes historical models
\item \textbf{NASA JPL Horizons System}: Ephemerides and historical position data; allows verification of Greenwich Observatory measurements against modern ephemerides
\item \textbf{Greenwich Observatory Archives (online)}: Digitized records, photographic plates, and astronomical observations (1675--1970s) available for research
\item \textbf{Bibliography of Greenwich Observatory}: Curated collection of publications, with links to full text where available
\end{itemize}

\textbf{Recommended Reading}:
\begin{itemize}
\item \emph{The King's Astronomer} (Willmoth, 1993): Biography of John Flamsteed; discusses observational methods and institutional development
\item \emph{Measuring the Universe} (Maury, 2010): Comprehensive history of astrometry; chapters on Airy and Greenwich instrumentation
\item \emph{Chasing Venus} (Sobel, 2012): Historical narrative on Venus transits and international scientific cooperation; discusses Greenwich's role
\item \emph{Empire of Time} (Sobel, 2011): Detailed account of chronometer development and testing at Greenwich Observatory
\end{itemize}

The orbital speed of Earth at any point in its elliptical orbit is determined by conservation of energy and angular momentum. The vis-viva equation gives the speed $v$ at distance $r$ from the Sun:

\[
  v^2 = GM \left( \frac{2}{r} - \frac{1}{a} \right),
\]

where $G$ is the gravitational constant, $M$ is the Sun's mass, and $a$ is the semi-major axis of Earth's orbit. For Earth, $a = 1$ AU (by definition). At perihelion, $r = a(1 - e) = 1 - 0.0167 \approx 0.9833$ AU. At aphelion, $r = a(1 + e) = 1 + 0.0167 \approx 1.0167$ AU.

The ratio of orbital speeds is:

\[
  \frac{v_{\text{perihelion}}}{v_{\text{aphelion}}} = \sqrt{\frac{2 - (1 + e)}{2 - (1 - e)}} = \sqrt{\frac{1 - e}{1 + e}} \approx \sqrt{\frac{0.9833}{1.0167}} \approx 0.983.
\]

Wait, this seems backward. Let me recalculate. At perihelion, the term $(2/r - 1/a)$ is larger, so $v_{\text{perihelion}} > v_{\text{aphelion}}$. Specifically,

\[
  v_{\text{perihelion}}^2 = GM \left( \frac{2}{a(1-e)} - \frac{1}{a} \right) = \frac{GM}{a} \left( \frac{2}{1-e} - 1 \right) = \frac{GM}{a} \left( \frac{2 - (1-e)}{1-e} \right) = \frac{GM}{a} \left( \frac{1 + e}{1-e} \right).
\]

And

\[
  v_{\text{aphelion}}^2 = \frac{GM}{a} \left( \frac{1 - e}{1+e} \right).
\]

Thus

\[
  \frac{v_{\text{perihelion}}}{v_{\text{aphelion}}} = \sqrt{\frac{(1+e)^2}{(1-e)^2}} = \frac{1+e}{1-e} \approx \frac{1.0167}{0.9833} \approx 1.034.
\]

Earth moves 3.4\% faster at perihelion than at aphelion. This variation in speed, sustained over weeks, causes the Sun to advance faster along the ecliptic at certain times of year.

\section*{Mean Anomaly and Kepler's Equation}

The mean anomaly $M$ is defined as the angle from perihelion, measured uniformly in time. If $t$ is the time since perihelion and $T$ is the orbital period, then

\[
  M = 2\pi \frac{t}{T}.
\]

The true anomaly $\nu$ (the actual angle from perihelion to the planet, as seen from the Sun) is related to mean anomaly by Kepler's equation:

\[
  M = \nu - e \sin \nu.
\]

This is a transcendental equation; there is no closed-form solution for $\nu$ in terms of $M$ for nonzero $e$. However, for small eccentricity, we can expand $\nu$ in powers of $e$:

\[
  \nu = M + e \sin M + \frac{e^2}{2} \sin 2M + \cdots
\]

For Earth's orbit with $e = 0.0167$, the second-order term contributes about $\sin 2M \times (0.0167)^2 / 2 \approx 0.00014 \sin 2M$, which is negligible compared to the first-order term $e \sin M \approx 0.0167 \sin M$.

\section*{Ecliptic Longitude from True Anomaly}

The true anomaly $\nu$ is measured from perihelion. Since perihelion occurs in early January, approximately 282.9° from the vernal equinox, the ecliptic longitude $\lambda$ is related to true anomaly by

\[
  \lambda = \nu + 282.9° = M + e \sin M + 282.9°.
\]

More precisely, if we measure ecliptic longitude from the vernal equinox (as is standard), then the mean ecliptic longitude $L$ (the ecliptic longitude of the mean Sun) is:

\[
  L = M + 282.9° = L_0 + M,
\]

where $L_0$ is the ecliptic longitude at epoch (January 1, 2000 noon, approximately 280.46°). On a given day of the year, measured from January 1, the mean anomaly is

\[
  M = 360° \frac{d - 1}{365.25},
\]

where $d$ is the day of the year (day 1 = January 1, day 32 = February 1, etc.).

\section*{Spherical Trigonometry: The Obliquity Effect}

The Sun moves along the ecliptic, which is inclined at angle $\epsilon \approx 23.44°$ to the celestial equator. As the Sun moves from ecliptic longitude $\lambda$ to $\lambda + d\lambda$, its equatorial (right ascension) longitude $\alpha$ changes by an amount that depends on $\lambda$.

At the vernal equinox ($\lambda = 0°$), the Sun is on the equator with declination $\delta = 0°$. At this point, $d\alpha / d\lambda$ is maximum (approximately 1). At the summer solstice ($\lambda = 90°$), the Sun is farthest north with declination $\delta = +\epsilon$. At this point, $d\alpha / d\lambda < 1$ (the Sun moves slower in right ascension).

The relationship, from spherical trigonometry, is:

\[
  \cos \delta = \cos \epsilon \sin \lambda \quad \text{(Sun on ecliptic)}.
\]

Wait, let me reconsider. If the ecliptic is tilted $\epsilon$ from the equator, and the Sun is at ecliptic longitude $\lambda$, then the declination is:

\[
  \sin \delta = \sin \epsilon \sin \lambda.
\]

And the right ascension is found from:

\[
  \sin \alpha = \frac{\sin \lambda \sin \epsilon}{\cos \delta}.
\]

No, that's not quite right either. Let me use the standard transformation. If the Sun's ecliptic coordinates are $(\lambda, \beta)$ with $\beta = 0$ (Sun on the ecliptic), its equatorial coordinates $(\alpha, \delta)$ are related by:

\[
  \sin \delta = \sin \epsilon \sin \lambda.
\]

And

\[
  \tan \alpha = \frac{\sin \lambda}{\cos \epsilon \cos \lambda}.
\]

The rate of change of right ascension with respect to ecliptic longitude is:

\[
  \frac{d\alpha}{d\lambda} = \frac{\cos \epsilon}{1 - \sin^2 \epsilon \sin^2 \lambda}.
\]

Wait, let me derive this more carefully. From $\tan \alpha = \frac{\sin \lambda}{\cos \epsilon \cos \lambda}$, we have

\[
  \sec^2 \alpha \, d\alpha = \frac{\cos \lambda \cos \epsilon \cos \lambda - \sin \lambda \cos \epsilon (-\sin \lambda)}{\cos^2 \epsilon \cos^2 \lambda} d\lambda = \frac{\cos \epsilon}{\cos^2 \epsilon \cos^2 \lambda} d\lambda = \frac{1}{\cos \epsilon \cos^2 \lambda} d\lambda.
\]

Since $\sec^2 \alpha = 1 + \tan^2 \alpha = 1 + \frac{\sin^2 \lambda}{\cos^2 \epsilon \cos^2 \lambda}$, we get

\[
  d\alpha = \frac{\cos \epsilon \cos^2 \lambda}{\cos^2 \epsilon \cos^2 \lambda + \sin^2 \lambda} d\lambda = \frac{\cos \epsilon}{1 + \tan^2 \epsilon \sin^2 \lambda} d\lambda.
\]

For small $\lambda$ (near the vernal equinox), $\sin \lambda \approx \lambda$, so

\[
  \frac{d\alpha}{d\lambda} \approx \cos \epsilon \approx 0.9175.
\]

At $\lambda = 90°$ (summer solstice), $\sin \lambda = 1$, so

\[
  \frac{d\alpha}{d\lambda} = \frac{\cos \epsilon}{1 + \tan^2 \epsilon} = \cos^3 \epsilon \approx 0.774.
\]

The rate is about 15\% slower. Integrating this over a quarter of the year shows that the Sun lags the mean Sun by about 9.9 minutes by the time of the summer solstice.

\section*{The Equation of Time: Corrected Formula}

For precise calculations, the equation of time is typically given as:

\[
  E = -7.66 \sin(B) + 9.87 \sin(2B),
\]

in minutes, where $B = (d - 1) \times 360° / 365.25$ is the day angle in degrees, with $d$ being the day of the year. This empirical formula captures both the eccentricity and obliquity effects to good precision across the year.

Alternatively, using the components we derived:

\[
  E = E_{\text{ecc}} + E_{\text{obliq}} = -2e \sin M (360°/(2\pi)) \times 1440 \text{ min} + E_{\text{obliq}}.
\]

More precisely:

\[
  E_{\text{ecc}} = 7.66 \sin(B + 3.06°) \text{ minutes},
\]

and

\[
  E_{\text{obliq}} = 9.87 \sin(2B + 1.95°) \text{ minutes}.
\]

Their sum gives the full equation of time.

\section*{Extreme Values Throughout the Year}

The equation of time reaches four extrema in the course of a year:

\begin{center}
\begin{tabular}{llc}
\hline
\textbf{Date} & \textbf{Event} & \textbf{Equation of Time} \\
\hline
November 3 & Maximum positive & $+16.3$ min \\
February 12 & Minimum negative & $-14.3$ min \\
May 15 & Local maximum & $+3.8$ min \\
July 27 & Local minimum & $-5.8$ min \\
\hline
\end{tabular}
\end{center}

The large positive value in November occurs when aphelion (July) combines with the obliquity effect to produce a maximum. The large negative value in February occurs when perihelion (January) combines with the obliquity effect. The smaller extrema in May and July are local maxima and minima of less significance.

\section*{Analemma Coordinates}

If we plot the Sun's position in equatorial coordinates at the same clock time each day for a year, we obtain the analemma. In terms of declination $\delta$ (vertical coordinate) and equation of time $E$ (horizontal coordinate), the analemma is parameterized by:

\begin{align*}
  \delta &= \arcsin(\sin \epsilon \sin \lambda), \\
  \text{hour angle} &= -E/15° \text{ (in hours)}.
\end{align*}

The vertical extent is $2 \epsilon \approx 46.9°$, from $-23.44°$ to $+23.44°$. The horizontal extent is approximately 30 minutes (half an hour), from $-14.3$ to $+16.3$ minutes of time.

The figure-eight shape arises because the sign of the equation of time changes over the year, while the declination traces a smooth north-south path. The crossover points (where $E = 0$) occur near the equinoxes and solstices, with four dates per year when the Sun's clock time matches the mean time exactly: roughly April 15, June 15, September 1, and December 24.


