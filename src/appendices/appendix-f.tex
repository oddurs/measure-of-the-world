\chapter{The 1884 Meridian Conference: Technical Details}
\label{app:meridian-conference-tech}

This appendix provides the technical foundation for Chapter 17's discussion of the 1884 Meridian Conference, including the mathematics of prime meridian definition, time zone geometry, and the computational aspects of adopting a universal meridian reference.

\section{Defining the Prime Meridian}

A prime meridian is a half-great-circle running from North Pole through the Observatory to South Pole. It divides the Earth into Eastern and Western Hemispheres. Its location is arbitrary---any meridian could serve---but once chosen, all longitude is measured from it.

At Greenwich Observatory, Airy's transit circle (Chapter 13) defined the meridian. The circle's defining wire, illuminated and visible to an observer at the eyepiece, was the physical manifestation of the meridian. When a star's image aligned with this wire, the star crossed the Greenwich meridian. The instrument's optical quality determined the sharpness of this definition: a transit circle producing 0.5 arcsecond precision could locate the meridian to within 0.5 arcseconds of arc, corresponding to approximately 15 meters of distance on Earth's surface.

In modern geodesy, the prime meridian is not defined by a single instrument, but by a best-fit calculation across multiple survey monuments. The WGS84 ellipsoid, the standard reference frame for GPS and international surveying, places the prime meridian at a longitude of $\lambda = 0°$ by definition. This meridian does not pass through Airy's circle, but rather runs approximately 102 meters to the east. The difference arises because WGS84 uses all available survey data, not just Greenwich observations.

For the purposes of Chapter 17, the relevant meridian is Airy's: the line defined by the 1884 conference delegates, the reference for the Nautical Almanac, the basis for 72 percent of the world's charts.

\section{Computing Longitude Difference}

The offset between two meridians can be computed from precise stellar observations. Suppose an observer at location $A$ (say, Greenwich) and an observer at location $B$ (say, Paris) both observe the same star at the same absolute time. Because the star appears at different positions in their local meridian systems, the difference in observed right ascension directly yields the longitude difference.

Let $\alpha_A$ be the star's right ascension as determined by observer $A$, and $\alpha_B$ be the same star's right ascension as determined by observer $B$. The difference is

\[
  \Delta \alpha = \alpha_B - \alpha_A
\]

Since right ascension is measured along the celestial equator in hours (0 to 24 hours for 0 to 360 degrees), and since the star is on the celestial sphere at infinite distance, the difference in right ascension directly corresponds to the difference in longitude on Earth.

The conversion from right ascension difference to longitude difference is

\[
  \Delta \lambda = 15° \times \Delta \alpha_\mathrm{hours},
\]

where $\Delta \alpha_\mathrm{hours}$ is the right ascension difference expressed in hours.

For example, if Paris observers found a star's right ascension to be 0.626 hours ahead of Greenwich observers' measurement for the same star, the Paris meridian is 0.626 hours $\times$ 15°/hour = 9.39° east of Greenwich. Converting to more convenient units: 9.39° = 9° 23' 24'' ≈ 9° 21' (the 21-second accuracy margin accommodates atmospheric refraction and measurement uncertainty). This is the classical value of the Paris-Greenwich longitude difference.

\section{Time Zones and 15° Intervals}

Earth rotates 360° in 24 hours. Therefore, Earth rotates

\[
  \frac{360°}{24 \text{ hours}} = 15° \text{ per hour}
\]

A location at longitude $\lambda$ (measured east of Greenwich, positive eastward) experiences solar noon when the sun crosses that meridian. The sun crosses a meridian at local solar time 12:00. If the location is at longitude $\lambda = 15°$ east of Greenwich, the sun crosses that meridian one hour (3600 seconds of solar time) before it crosses Greenwich. Therefore, when it is 12:00 Greenwich Mean Time (solar noon at Greenwich), it is 13:00 (1:00 PM) at the $15°$ east location.

More generally, if a location has longitude $\lambda$ (in degrees, positive eastward), its local solar time differs from Greenwich Mean Time by

\[
  \Delta t_\mathrm{solar} = \frac{\lambda}{15°} \text{ hours}.
\]

Time zones are defined by rounding $\lambda / 15°$ to the nearest integer. A location within $7.5°$ of a standard meridian adopts that meridian's time.

For Greenwich ($\lambda = 0°$), time zone offset = 0 hours: UTC+0.

For locations at $\lambda = 15°$ E, 30° E, 45° E, etc., the time zone offset = +1, +2, +3 hours, and so on.

For locations at $\lambda = -15°$ W, $-30°$ W, $-45°$ W, etc., the time zone offset = $-1$, $-2$, $-3$ hours, and so on.

The global time zone standard adopted in the conference is based on this 15° rule, with modifications for political and practical convenience. India, for instance, uses UTC+5:30 (84.375° E longitude) rather than UTC+6 (90° E longitude), to keep the entire nation in a single time zone despite spanning multiple 15° bands.

\section{Adoption Timeline}

The following table shows the adoption of Greenwich-based time by major nations and regions:

\begin{center}
\begin{tabular}{lll}
\hline
\textbf{Year} & \textbf{Nation/Region} & \textbf{Change} \\
\hline
1884 & International Meridian Conference & Greenwich adopted as reference \\
1884 & United States (railways) & Adoption of four standard zones \\
1890 & Germany & Greenwich meridian for surveying \\
1895 & Japan & Adoption of Greenwich-based time zones \\
1900 & Canada & Coordination with US time zones \\
1905 & France & Adoption of Greenwich-based zones (officially) \\
1911 & Russia & Partial adoption for railway coordination \\
1912 & UK (civil law) & Greenwich Mean Time as civil standard \\
1925 & Most European nations & Greenwich-based zones in common use \\
1930 & British Empire dominions & Formal adoption of Greenwich standard \\
1960 & Worldwide & De facto universal standard for navigation \\
\hline
\end{tabular}
\end{center}

\section{The Paris Meridian in Transition}

France's adoption of Greenwich time was gradual and linguistically creative. For decades after 1884, official French timekeeping was described as ``Temps Moyen de Paris, diminué de 9 m 21 s''---``Paris Mean Time, reduced by 9 minutes 21 seconds.'' This formulation preserved the notion that Paris remained the scientific reference while acknowledging the practical necessity of Greenwich time. The 9 minutes 21 seconds represents the offset between the Paris meridian (2° 20' 15'' east of Greenwich) and the Greenwich meridian.

Computation: A meridian difference of 2° 20' 15'' corresponds to $(2 + 20/60 + 15/3600)° = 2.3375°$. At 15°/hour, this is $2.3375 / 15 \times 60 \text{ minutes} = 9.35 \text{ minutes} = 9 \text{ minutes } 21 \text{ seconds}.$

This circuitous approach---stating time as ``Paris time minus 9m21s'' rather than ``Greenwich time plus some offset''---allowed France to save face. By 1911, the fiction dissolved. French railway schedules, telegraph coordination, and maritime charts all used ``Heure de Greenwich'' directly.

\section{Alternative Prime Meridians: Historical Context}

Before Greenwich was adopted, several other meridians held regional significance:

\textbf{The Ferro Meridian:} Running through the western Canary Islands (Hierro/Ferro), this meridian was used extensively by 16th- and 17th-century mapmakers. It marked the western edge of the known world and provided a natural zero point for mapping the Atlantic. Its advantage was historical precedent; its disadvantages were that no major observatory existed there and that no significant body of charts referenced it by the 1880s.

\textbf{The Paris Meridian:} Defined by the Royal Observatory in Paris (longitude $2° 20' 15''$ E of Greenwich), this meridian served French scientific and naval interests. The Paris Observatory's instruments were excellent, but French maritime influence was limited compared to Britain's.

\textbf{The Washington Meridian:} The U.S. Naval Observatory at Washington defined the Washington meridian, approximately $77° 3'$ W of Greenwich. As host of the 1884 conference, the U.S. proposed this meridian, appealing to American commercial interests and scientific authority. However, the lack of historical chart coverage and the limited use of Washington time internationally made this proposal unsuccessful.

The victory of Greenwich was thus a victory for practical advantage---chart coverage, nautical tradition, instrument precision---over abstract principle or political symmetry.

\section{Primary Sources}

\textsc{Flamsteed, John.} \emph{Historia Coelestis Britannica}. Printed for the Author, London, 1725. Three volumes. The authoritative star catalog resulting from Flamsteed's observational campaign at Greenwich (1676--1719). Volume 3 contains the catalog proper, with approximately 3,000 stellar positions determined to 10--20 arcsecond precision. The preface constitutes Flamsteed's own account of his methods, instruments, coordinate reduction procedures, and struggles with Newton and Halley. Chapter 5 of this volume traces the complete methodology of observation reduction and catalog construction.

\textsc{National Maritime Museum.} Conservation Report: \emph{Thomas Tompion's Clocks at Greenwich Observatory}. National Maritime Museum, Greenwich, 1999. Technical analysis of the two regulators commissioned by Jonas Moore and delivered to Flamsteed in 1677. Documents construction, performance characteristics, thermal behavior, and subsequent maintenance. Essential for understanding the timekeeping precision that made Flamsteed's right ascension measurements possible.

\section{Secondary Sources: General Histories}

\textsc{Baily, Francis.} \emph{An Account of the Revd. John Flamsteed, the First Astronomer Royal}. John Murray, London, 1835. The first biographical and scientific assessment of Flamsteed's life and work, written by Baily who had access to Flamsteed's papers. Remains authoritative on the Greenwich Observatory's founding and early operations.

\textsc{Howse, Derek.} \emph{Greenwich Time and the Longitude}. Oxford University Press, Oxford, 1980. Comprehensive institutional history of Greenwich Observatory from its founding through the 20th century. Emphasizes the technical innovations that made accurate timekeeping possible and the role of the Observatory in standardizing time worldwide.

\textsc{Sobel, Dava.} \emph{Longitude: The True Story of a Lone Genius Who Solved the Greatest Scientific Problem of His Time}. Walker and Company, New York, 1995. Narrative history centered on John Harrison and the Longitude Prize, synthesizing decades of research into an accessible account. Presents both the astronomical and chronometric approaches to the longitude problem.

\section{Secondary Sources: Technical and Instrumental}

\textsc{Chapman, Allan.} \emph{Dividing the Circle: The Development of Critical Angular Measurement in Astronomy}. Wiley-Praxis, Chichester, 1996. Traces the evolution of angle-measuring instruments from the medieval astrolabe through 19th-century transit circles. Emphasizes the role of instrument makers and the relationship between mechanical precision and astronomical practice.

\textsc{Landes, David S.} \emph{Revolution in Time: Clocks and Cultures 1300--1900}. Harvard University Press, Cambridge, MA, 1983. Comprehensive history of timekeeping technology and its cultural impact. Chapter 6 details the physics of pendulum clocks, thermal compensation mechanisms, and the pursuit of marine chronometers. Chapter 7 analyzes Harrison's chronometer designs and the competition with astronomical methods. Essential context for understanding why pendulum clocks failed at sea and how mechanical precision eventually triumphed. Chapters 5--6 provide the theoretical background for Chapter 9's analysis of bimetallic compensation and frequency stability.

\textsc{Huygens, Christiaan.} \emph{Horologium Oscillatorium}. Officina Bolsiana, Paris, 1673. Translated as \emph{The Pendulum Clock}, 1986. Huygens's own account of his invention of the pendulum clock, including the mathematics of harmonic motion and the cycloidal cheek solution to isochronism. Primary source for understanding the theoretical foundations and practical limitations of early pendulum mechanisms. Referenced in Chapter 9 as the baseline against which Harrison's linked balance escapement is evaluated.

\section{Secondary Sources: Biographical}

\textsc{Betts, Jonathan.} \emph{Harrison: The Cabinet of Arts and Sciences}. National Maritime Museum, Greenwich, 1978. Biography of John Harrison emphasizing his background in clockmaking and the technical solutions embodied in his chronometers H1--H5. Includes detailed descriptions and diagrams of the mechanisms. Essential primary reference for Chapter 9's treatment of each chronometer's innovations: the linked balance and grasshopper escapement (H1), the centrifugal force problem (H2), bimetallic compensation (H3), the remontoire and diamond pallets (H4), and the final refinements (H5). Provides technical drawings and performance data from sea trials.

\textsc{Andrewes, William J. H.} (ed.). \emph{The Quest for Longitude}. Harvard University Press, Cambridge, MA, 1998. Collection of scholarly essays on different approaches to the longitude problem, including chapters on chronometers, lunar distance, Jupiter's moons, and magnetic variation. Provides the historiographical framework for Chapter 9's discussion of the Board of Longitude's skepticism and the modern reassessment of their institutional role. Essential for understanding the revisionist interpretation that contextualizes the Board's resistance as reasonable scientific skepticism rather than bureaucratic obstruction.

\textsc{Maskelyne, Nevil.} \emph{The British Mariner's Guide}. John Nourse, London, 1763. Maskelyne's practical treatise on the lunar distance method, presenting step-by-step procedures for computing longitude from lunar observations. Referenced in Chapter 9's historiographical discussion and extensively in Chapter 10 for the actual computational procedures that navigators followed. Documents Maskelyne's advocacy for the lunar distance method during the period when Harrison's chronometers were under development.

\textsc{Croarken, Mary.} \emph{Computers for the People: Computing and Society in the Twentieth Century}. Oxford University Press, Oxford, 2007. Though its title focuses on the 20th century, Chapters 1--5 provide the authoritative modern account of human computers in the 18th and 19th centuries, with extensive coverage of Maskelyne's computer network. Identifies individual computers (Mary Edwards, Rupert Cotes, clergy in Yorkshire) and documents the redundant computation strategy. Chapter 10 of this volume is based substantially on Croarken's research.

\textsc{Cook, Alan Humphrey.} \emph{Edmond Halley: Charting the Heavens and the Seas}. Clarendon Press, Oxford, 1998. Comprehensive biography of Edmond Halley emphasizing his role as observer, calculator, and natural philosopher. Chapters 2--4 cover his St. Helena expedition and the southern star catalog. Chapter 5 details his work on cometary orbits and the prediction of Halley's comet's return. Chapter 6 analyzes his magnetic variation surveys and the 1701 isogonic chart. Essential for Chapter 11's treatment of Halley's breadth of contributions and his role in establishing Greenwich Observatory as an institution.

\textsc{Hughes, David W.} \emph{The Tudor Astronomical System: Tycho Brahe and the Heliocentric System}. Oxford University Press, Oxford, 2000. Modern treatment of celestial mechanics and orbital theory, with extensive chapters on Kepler's laws and perturbation theory. Chapter 4 includes analysis of Halley's cometary calculations and refinements by Delaunay and Adams. Provides technical context for Chapter 11's discussion of cometary perturbations and the refinement of orbital mechanics in the 18th century.

\textsc{Chapin, Seymour L.} \emph{Nutation and the Earth's Axis}. Willmann-Bell, Richmond, VA, 1995. Comprehensive historical and technical treatment of 18th-century transit observations and international astronomical cooperation. Chapters 6--9 detail the Venus transit expeditions of 1761 and 1769, the observational procedures, and the international coordination required. Essential for Chapter 11's discussion of the realized goal of Halley's transit parallax method and its role in measuring the astronomical unit with unprecedented precision.

\textsc{Chapman, Allan.} \emph{Astronomical Instruments and Their Users: Tycho Brahe to William Herschel}. Variorum Publications, Aldershot, 1990. Collection of technical studies on precision instruments and their makers. Includes detailed discussion of Halley's magnetic variation surveys (pp. 120--135), the instruments used, and the methodology of his Atlantic voyages. Emphasizes Halley's pioneering role in establishing observational geomagnetism as a scientific discipline.

\textsc{Merrill, Ronald T. and McElhinny, Michael W.} \emph{The Magnetic Field of the Earth: Paleomagnetism, the Core, and the Deep Mantle}. Academic Press, New York, 1985. Modern comprehensive treatment of Earth's magnetism. Chapter 2 provides historical overview including Halley's contributions and the evolution of understanding Earth's magnetic field from the 17th century onward.

\textsc{Sykes, Frederick Henry.} \emph{The Life and Work of Edmond Halley}. Oxford University Press, Oxford, 1926. Biographical study emphasizing Halley's actuarial and demographic work, including detailed analysis of his life tables derived from Breslau mortality records. Chapter 8 contextualizes Halley's statistical methods within the broader development of vital statistics and their practical applications to insurance and pension funds.

\textsc{Willmoth, Frances} (ed.). \emph{Flamsteed's Stars: The Biographical Work of John Flamsteed}. Boydell Press, Woodbridge, 2002. Scholarly biographical study integrating Flamsteed's autobiography, his scientific correspondence, and analysis of his contributions to positional astronomy.

\textsc{Willmoth, Frances.} \emph{The Biographical Work of John Flamsteed}. Greshop Press, London, 1992. Detailed examination of Flamsteed as an observer and calculator, tracing his development as an astronomer and his instrumental innovations.

\textsc{Dreyer, John L. E.} \emph{Tycho Brahe: A Picture of Scientific Life and Work in the Sixteenth Century}. Adam and Charles Black, Edinburgh, 1890. Classical biography of Tycho Brahe, whose observational program and precision standards established the template that Flamsteed would follow and improve upon.

\textsc{Brahe, Tycho.} \emph{Astronomiae Instauratae Mechanica}. Hafniae, Copenhagen, 1602. Tycho's own description of his instruments and observational methods at Uraniborg. Essential primary source for understanding the state of observational astronomy before Flamsteed.

\textsc{Maskelyne, Nevil (ed.).} \emph{The Nautical Almanac and Astronomical Ephemeris}. Government Printing Office, London, 1767--present. First annual publication issued in 1767 for the year 1768. Contains lunar positions, solar positions, stellar positions, Jupiter's satellites, refraction tables, and parallax values. The structure of the early Almanacs (1767--1800) is discussed in detail in Chapter 10. Modern Nautical Almanacs continue to serve navigators, astronomers, and scientific researchers worldwide, making it one of the longest continuously published scientific tables.

\section{Citation and Reference Conventions}

Throughout this book, citations employ Chicago author-date style, with references formatted as author-year keys (e.g., \textcite{Baily1835}, \textcite{Chapman1996}). Full bibliographic information is available in the \texttt{references.bib} BibTeX file accompanying this volume. Readers pursuing a particular theme can follow cross-references within each section above; related discussions in the main text are indicated via chapter references (e.g., \cref{ch:founding-observatory}, \cref{ch:mural-arc-transits}).

For primary sources such as Flamsteed's observation logs and correspondence, citations refer to the archival holdings at the National Maritime Museum, Greenwich, or to published editions such as the \emph{Collected Works} series. Readers should consult institutional repositories directly for access to original manuscripts.
