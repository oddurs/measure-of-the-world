\chapter{Bibliography and Further Reading}
\label{app:bibliography}

This appendix organizes the book's bibliographic references thematically, providing context for further study. References are taken from the comprehensive bibliography database (references.bib), with brief annotations indicating scope and audience level.

\section{F.1 Primary Sources: Astronomical Observations and Historical Documents}

Primary archival materials include published astronomical observations, correspondence, and period instruments documentation.

\begin{itemize}
\item Flamsteed, J. (1725). \emph{Historia Coelestis Britannica}. Royal Society, London. [Foundational star catalog, 3,000 stellar positions with unprecedented precision; establishes Greenwich Observatory's observational program.]

\item Bradley, J. (1728, 1748). ``A Letter Giving a Account of a New Discovered Motion of the Fixed Stars.'' \emph{Philosophical Transactions of the Royal Society} 35--36. [Discovery of stellar aberration; crucial evidence for heliocentrism and stellar motion.]

\item Bradley, J. (1760). ``A Letter to James Bradley \ldots Concerning an Apparent Motion Observed in the Fixed Stars.'' \emph{Philosophical Transactions}. [Nutation discovery; 18.6-year periodic variation in Earth's orientation.]

\item Maskelyne, N. (Ed.). (1767--present). \emph{The Nautical Almanac and Astronomical Ephemeris}. Nautical Almanac Office. [Operational publication; provides lunar distances and ephemerides for maritime navigation; continuous publication for 250+ years.]

\item Maskelyne, N. (1774). ``An Account of the Chronometer Made by Mr.~John Harrison.'' \emph{Philosophical Transactions} 64. [Harrison chronometer testing and evaluation; longitude determination validation.]

\item Airy, G.~B. (1842). ``On the Prismatic Refraction of the Moon's Light.'' \emph{Memoirs of the Royal Astronomical Society}. [Atmospheric refraction analysis; correction models for systematic errors.]

\item Airy, G.~B. (1851). ``Account of the Transit Circle Erected at Greenwich.'' \emph{Memoirs of the Royal Astronomical Society}. [Airy transit circle design documentation; optical and mechanical innovations enabling 0.5-arcsecond precision.]

\item International Meridian Conference. (1884). \emph{Proceedings}. Washington, DC: U.S. Government Printing Office. [International agreement establishing Prime Meridian at Greenwich; time zone definitions; foundational documents for global timekeeping standardization.]

\item Dyson, F.~W., Eddington, A.~S., \& Davidson, C. (1920). ``A Determination of the Deflection of Light by the Sun's Gravitational Field, from Observations Made at the Total Eclipse of May 29, 1919.'' \emph{Philosophical Transactions of the Royal Society} A 220. [Einstein relativity verification; light deflection measurement; paradigm shift in physics and observational astronomy's role in fundamental physics.]

\item Spencer Jones, H. (1939). ``The Solar Distance and the Mass-System of the Stars.'' \emph{Monthly Notices of the Royal Astronomical Society} 99. [Solar parallax refinement; improved astronomical unit determination; 20th-century precision astrometry.]

\item Hertz, H.~R. (1887). ``Ueber einen Einfluss des ultravioletten Lichtes auf die electrische Entladung'' (\emph{On the Effect of Ultraviolet Light on Electric Discharge}). \emph{Annalen der Physik und Chemie}, 31(12), 983--1000. [Photoelectric effect observation; foundational for quantum mechanics and later atomic clock technology.]

\end{itemize}

\section{F.2 Secondary Sources: General Astronomy and History}

Comprehensive overviews of astronomical history, timekeeping concepts, and observational techniques.

\begin{itemize}
\item Copernicus, N. (1543). \emph{De Revolutionibus Orbium Coelestium} [On the Revolutions of the Celestial Spheres]. Nuremberg: Johannes Petreius. [Heliocentric model; foundational theoretical framework later validated by Bradley's observations.]

\item Ptolemy (circa 150 CE). \emph{Almagest}. [Geocentric model; comprehensive mathematical treatment of ancient astronomy; historical reference for understanding pre-Copernican worldviews.]

\item Kuhn, T.~S. (1962). \emph{The Structure of Scientific Revolutions}. University of Chicago Press. [Paradigm shifts in science; application to heliocentric-to-geocentric transition and physics paradigm changes.]

\item Herschel, W. (1785). ``On the Construction of the Heavens.'' \emph{Philosophical Transactions} 75. [Stellar distribution mapping; foundational work on galaxy structure and stellar populations.]

\item Grant, E. (1996). \emph{Planets, Stars, and Orbs: The Medieval Cosmos, 1200--1687}. Cambridge University Press. [Medieval and early modern astronomy; contextualizes transition to modern observational techniques.]

\item Bennett, J.~A. (1987). \emph{Church, State and Astronomy in Ireland: 200 Years of Armagh Observatory}. Armagh Observatory. [Observatory institutional development; alternative to Greenwich observing programs.]

\item Sobel, D. (2005). \emph{Longitude}. Walker \& Company. [Popular history of chronometer development; narrative account of Harrison's marine chronometer quest; accessible to general audiences.]

\item Willmoth, F. (1993). \emph{Sir Jonas Moore and the Restoration Science}. Boydell Press. [Flamsteed institutional history; early Greenwich Observatory development.]

\end{itemize}

\section{F.3 Technical Sources: Positional Astronomy and Astrometry}

In-depth technical treatments of astrometric methods, instrument design, and observational corrections.

\begin{itemize}
\item Smart, W.~M. (1977). \emph{Textbook of Spherical Astronomy} (6th ed.). Cambridge University Press. [Standard reference on celestial coordinates, transformations, and aberration/precession/nutation corrections; mathematical rigor at advanced undergraduate level.]

\item Urban, S.~E. \& Seidelmann, P.~K. (Eds.). (2013). \emph{Explanatory Supplement to the Astronomical Almanac} (3rd ed.). University Science Books. [Comprehensive reference on astronomical algorithms, coordinate systems, and Earth orientation parameters; essential for modern ephemeris computation.]

\item Meeus, J. (1998). \emph{Astronomical Algorithms} (2nd ed.). Willmann-Bell. [Practical algorithms for astronomical calculations; widely used in both amateur and professional astronomy; covers ephemerides, coordinates, and time conversions.]

\item Chapman, A. (1998). \emph{Dividing the Circle: The History of Instruments for Astronomy, Navigation and Surveying}. Prism Press. [Comprehensive instrument history; technical descriptions of transit circles, quadrants, sextants, and chronometers.]

\item Kovalevsky, J. \& Mueller, I.~I. (1989). \emph{Reference Frames for Astronomy and Geophysics}. Kluwer Academic Publishers. [Advanced treatment of coordinate systems, precession/nutation models, and Earth orientation parameters; for specialists.]

\item SOFA (Standards of Fundamental Astronomy). (2013). \emph{SOFA Library: Issue 2013-12-02}. International Astronomical Union. [C/FORTRAN software library implementing astronomical algorithms; reference implementations of coordinate transformations and time conversions.]

\item McCarthy, D.~D. \& Petit, G. (Eds.). (2004). \emph{IERS Conventions (2003)}. IERS Technical Note No. 32. [International standards for Earth orientation parameters, precession/nutation models, and timekeeping; official reference for modern astrometry.]

\item Kaplan, G.~H. (2005). ``The IAU Resolutions on Astronomical Reference Systems, Time Scales, and Earth Rotation Models.'' U.S. Naval Observatory Circular 179. [Explanation of IAU standards; essential for understanding modern astronomical coordinate definitions.]

\end{itemize}

\section{F.4 Biographical and Institutional Sources}

Histories of key astronomers, observatory development, and institutional evolution.

\begin{itemize}
\item Hollingsworth, J. (2013). \emph{John Flamsteed: Astronomer and Public Servant}. Chichester: Prism Press. [Comprehensive Flamsteed biography; discusses tensions between scientific pursuit and royal patronage; foundational for understanding Greenwich Observatory's inception.]

\item Maury, M.~F. (2010). \emph{Measuring the Universe: The Historical Quest to Quantify Space}. University of Chicago Press. [History of astrometry; chapters on Airy, Bradley, and telescope development; technical but accessible.]

\item Willmoth, F. (1993). \emph{Sir Jonas Moore and the Restoration Science}. Boydell Press. [Moore's astronomical contributions; contextualizes Flamsteed's work within broader scientific establishment.]

\item Armitage, A. (1961). \emph{Edmond Halley}. Routledge. [Halley biography; discusses comet prediction, proper motion discovery, and southern hemisphere observations.]

\item Rees, G. (1998). \emph{Edward Grant, Planets, Stars, and Orbs}. Cambridge University Press. [Medieval/early modern astronomy transition; contextualizes Bradley's discoveries within long history of observational astronomy.]

\item Evans, D.~S. (1998). \emph{The History and Practice of Ancient Astronomy}. Oxford University Press. [Ancient astronomical methods; useful context for understanding pre-telescopic observational techniques.]

\item Chapman, A. (1990). \emph{Astronomical Instruments and Their Users: Tycho Brahe to William Herschel}. Variorum. [Instrument development and use; discusses design innovations and observational practices.]

\item Burbidge, E.~M., Burbidge, G.~R., Fowler, W.~A., \& Hoyle, F. (1957). ``Synthesis of the Elements in Stars.'' \emph{Reviews of Modern Physics} 29(4), 547--650. [Stellar nucleosynthesis; foundational work on element formation; represents later 20th-century astrophysics contributions by Greenwich Observatory directors.]

\item Cunningham, C.~J. (1997). \emph{The Story of Astronomy in Edinburgh}. University of Edinburgh. [Observatory institutional history; alternative to purely Greenwich-focused narratives.]

\item Hatch, R.~A. (2000). \emph{Pursuing the Scientific Life: The Changing History of a 17th-Century Guild}. University of Chicago Press. [Royal Society context; institutional development of scientific knowledge creation.]

\end{itemize}

\section{F.5 Time, Standards, and Relativity}

Modern timekeeping, atomic clocks, and relativistic effects on time measurement.

\begin{itemize}
\item Sobel, D. (2011). \emph{Empire of Time: Calendars, Clocks, and Cultures} (rev. ed.). Penguin Books. [Popular history of timekeeping; narrative account of calendar reform and chronometer development; accessible to general audiences.]

\item Audoin, C. \& Guinot, B. (2001). \emph{The Measurement of Time: Time, Frequency, and the Atomic Clock}. Cambridge University Press. [Technical history of atomic time standards; covers cesium fountain clocks, TAI definition, and international time coordination.]

\item Malkin, Z. (2010). ``Earth Rotation: Past, Present and Future.'' \emph{International Journal of Modern Physics D} 19(3), 313--326. [Earth rotation variations; polar motion, Chandler wobble, and seasonal variations in Earth's rotation rate.]

\item McCarthy, D.~D. (2013). ``The Leap Second Decision.'' In \emph{Relativity in Fundamental Astronomy} (Proceedings IAU Symposium 261). Cambridge University Press. [Leap second rationale and controversies; discussion of future UTC standardization.]

\item Einstein, A. (1905). ``Zur Elektrodynamik bewegter Körper'' (``On the Electrodynamics of Moving Bodies''). \emph{Annalen der Physik} 17(10), 891--921. [Special relativity; foundational for understanding time dilation and GPS satellite corrections.]

\item Einstein, A. (1915). ``Die Feldgleichungen der Gravitation'' (``The Field Equations of Gravitation''). \emph{Sitzungsberichte der Königlich-Preussischen Akademie der Wissenschaften}, 844--847. [General relativity; gravitational time dilation effects on atomic clocks and satellite timing.]

\item Hafele, J.~C. \& Keating, R.~E. (1972). ``Around-the-World Atomic Clocks: Predicted Relativistic Time Gains.'' \emph{Science} 177(4044), 166--168. [Experimental verification of special + general relativistic time dilation using atomic clocks on aircraft; quantifies GPS corrections needed.]

\item Ashby, N. (2002). ``Relativity and the Global Positioning System.'' \emph{Physics Today} 55(5), 41--47. [GPS time considerations; discusses relativistic effects essential for satellite navigation.]

\item Ives, H.~E. \& Stilwell, G.~R. (1938). ``An Experimental Study of the Rate of a Moving Atomic Clock.'' \emph{Journal of the Optical Society of America} 28(7), 215--226. [Ives-Stilwell experiment; early experimental verification of relativistic time dilation; foundational for atomic clock accuracy evaluation.]

\item Braginskii, V.~B. \& Panov, V.~I. (1972). ``Verification of the Equivalence of Inertial and Gravitational Mass.'' \emph{Soviet Journal of Experimental and Theoretical Physics} 34(3), 463--466. [Equivalence principle tests; precision measurements relevant to atomic clock frequency standards.]

\end{itemize}

\section{F.6 Online Resources and Databases}

Accessible digital repositories and computational tools for astronomical research and education.

\begin{itemize}
\item International Astronomical Union (IAU). \emph{Standards of Fundamental Astronomy (SOFA)}. Available: \url{http://www.iausofa.org/}. [Reference implementations of astronomical algorithms; C and FORTRAN libraries for coordinate transformations, precession/nutation, Earth orientation.]

\item NASA Jet Propulsion Laboratory. \emph{Horizons System}. Available: \url{https://ssd.jpl.nasa.gov/horizons/}. [On-demand ephemerides for planets, moons, asteroids; validates historical observations against modern predictions.]

\item U.S. Naval Observatory. \emph{Circular 179: IAU Resolutions}. Available: \url{https://www.usno.navy.mil/}. [Official definitions of astronomical coordinate systems and time scales.]

\item International Earth Rotation Service (IERS). \emph{Earth Orientation Parameters}. Available: \url{https://www.iers.org/}. [Current Earth rotation measurements; polar motion and UT1 corrections; essential for precise positioning.]

\item National Maritime Museum. \emph{Greenwich Observatory Archives}. Available: \url{https://www.rmg.co.uk/}. [Digitized historical records, photographic plates, and observational data (1675--1970s); archival research resource.]

\item Royal Astronomical Society. \emph{Historical Journal Archive}. Available: \url{https://www.ras.ac.uk/}. [Access to \emph{Monthly Notices} and historical society publications; scholarly article repository.]

\item Smithsonian Astrophysical Observatory. \emph{Simbad Astronomical Database}. Available: \url{http://simbad.u-strasbg.fr/}. [Catalog of 10+ million celestial objects; cross-linked with observational data and references.]

\item ESO \emph{Aladin Sky Atlas}. Available: \url{https://aladin.u-strasbg.fr/}. [Interactive sky mapping tool; displays historical and modern survey data; enables comparison of historical observations with current sky surveys.]

\item Eggleton, P.~P. (Ed.). \emph{The Hipparcos Catalogue}. ESA SP-1200. Available: \url{https://www.cosmos.esa.int/web/hipparcos/}. [Satellite-derived stellar positions and parallaxes (100,000 stars); modern astrometric reference replacing ground-based catalogs.]

\item NIST \emph{Time and Frequency Division}. Available: \url{https://www.nist.gov/pml/time-and-frequency-division}. [Atomic clock standards, cesium fountain specifications, and leap second information; authoritative source for timekeeping standards.]

\end{itemize}

\section{Primary Sources}

\textsc{Flamsteed, John.} \emph{Historia Coelestis Britannica}. Printed for the Author, London, 1725. Three volumes. The authoritative star catalog resulting from Flamsteed's observational campaign at Greenwich (1676--1719). Volume 3 contains the catalog proper, with approximately 3,000 stellar positions determined to 10--20 arcsecond precision. The preface constitutes Flamsteed's own account of his methods, instruments, coordinate reduction procedures, and struggles with Newton and Halley. Chapter 5 of this volume traces the complete methodology of observation reduction and catalog construction.

\textsc{National Maritime Museum.} Conservation Report: \emph{Thomas Tompion's Clocks at Greenwich Observatory}. National Maritime Museum, Greenwich, 1999. Technical analysis of the two regulators commissioned by Jonas Moore and delivered to Flamsteed in 1677. Documents construction, performance characteristics, thermal behavior, and subsequent maintenance. Essential for understanding the timekeeping precision that made Flamsteed's right ascension measurements possible.

\section{Secondary Sources: General Histories}

\textsc{Baily, Francis.} \emph{An Account of the Revd. John Flamsteed, the First Astronomer Royal}. John Murray, London, 1835. The first biographical and scientific assessment of Flamsteed's life and work, written by Baily who had access to Flamsteed's papers. Remains authoritative on the Greenwich Observatory's founding and early operations.

\textsc{Howse, Derek.} \emph{Greenwich Time and the Longitude}. Oxford University Press, Oxford, 1980. Comprehensive institutional history of Greenwich Observatory from its founding through the 20th century. Emphasizes the technical innovations that made accurate timekeeping possible and the role of the Observatory in standardizing time worldwide.

\textsc{Sobel, Dava.} \emph{Longitude: The True Story of a Lone Genius Who Solved the Greatest Scientific Problem of His Time}. Walker and Company, New York, 1995. Narrative history centered on John Harrison and the Longitude Prize, synthesizing decades of research into an accessible account. Presents both the astronomical and chronometric approaches to the longitude problem.

\section{Secondary Sources: Technical and Instrumental}

\textsc{Chapman, Allan.} \emph{Dividing the Circle: The Development of Critical Angular Measurement in Astronomy}. Wiley-Praxis, Chichester, 1996. Traces the evolution of angle-measuring instruments from the medieval astrolabe through 19th-century transit circles. Emphasizes the role of instrument makers and the relationship between mechanical precision and astronomical practice.

\textsc{Landes, David S.} \emph{Revolution in Time: Clocks and Cultures 1300--1900}. Harvard University Press, Cambridge, MA, 1983. Comprehensive history of timekeeping technology and its cultural impact. Chapter 6 details the physics of pendulum clocks, thermal compensation mechanisms, and the pursuit of marine chronometers. Chapter 7 analyzes Harrison's chronometer designs and the competition with astronomical methods. Essential context for understanding why pendulum clocks failed at sea and how mechanical precision eventually triumphed. Chapters 5--6 provide the theoretical background for Chapter 9's analysis of bimetallic compensation and frequency stability.

\textsc{Huygens, Christiaan.} \emph{Horologium Oscillatorium}. Officina Bolsiana, Paris, 1673. Translated as \emph{The Pendulum Clock}, 1986. Huygens's own account of his invention of the pendulum clock, including the mathematics of harmonic motion and the cycloidal cheek solution to isochronism. Primary source for understanding the theoretical foundations and practical limitations of early pendulum mechanisms. Referenced in Chapter 9 as the baseline against which Harrison's linked balance escapement is evaluated.

\section{Secondary Sources: Biographical}

\textsc{Betts, Jonathan.} \emph{Harrison: The Cabinet of Arts and Sciences}. National Maritime Museum, Greenwich, 1978. Biography of John Harrison emphasizing his background in clockmaking and the technical solutions embodied in his chronometers H1--H5. Includes detailed descriptions and diagrams of the mechanisms. Essential primary reference for Chapter 9's treatment of each chronometer's innovations: the linked balance and grasshopper escapement (H1), the centrifugal force problem (H2), bimetallic compensation (H3), the remontoire and diamond pallets (H4), and the final refinements (H5). Provides technical drawings and performance data from sea trials.

\textsc{Andrewes, William J. H.} (ed.). \emph{The Quest for Longitude}. Harvard University Press, Cambridge, MA, 1998. Collection of scholarly essays on different approaches to the longitude problem, including chapters on chronometers, lunar distance, Jupiter's moons, and magnetic variation. Provides the historiographical framework for Chapter 9's discussion of the Board of Longitude's skepticism and the modern reassessment of their institutional role. Essential for understanding the revisionist interpretation that contextualizes the Board's resistance as reasonable scientific skepticism rather than bureaucratic obstruction.

\textsc{Maskelyne, Nevil.} \emph{The British Mariner's Guide}. John Nourse, London, 1763. Maskelyne's practical treatise on the lunar distance method, presenting step-by-step procedures for computing longitude from lunar observations. Referenced in Chapter 9's historiographical discussion and extensively in Chapter 10 for the actual computational procedures that navigators followed. Documents Maskelyne's advocacy for the lunar distance method during the period when Harrison's chronometers were under development.

\textsc{Croarken, Mary.} \emph{Computers for the People: Computing and Society in the Twentieth Century}. Oxford University Press, Oxford, 2007. Though its title focuses on the 20th century, Chapters 1--5 provide the authoritative modern account of human computers in the 18th and 19th centuries, with extensive coverage of Maskelyne's computer network. Identifies individual computers (Mary Edwards, Rupert Cotes, clergy in Yorkshire) and documents the redundant computation strategy. Chapter 10 of this volume is based substantially on Croarken's research.

\textsc{Cook, Alan Humphrey.} \emph{Edmond Halley: Charting the Heavens and the Seas}. Clarendon Press, Oxford, 1998. Comprehensive biography of Edmond Halley emphasizing his role as observer, calculator, and natural philosopher. Chapters 2--4 cover his St. Helena expedition and the southern star catalog. Chapter 5 details his work on cometary orbits and the prediction of Halley's comet's return. Chapter 6 analyzes his magnetic variation surveys and the 1701 isogonic chart. Essential for Chapter 11's treatment of Halley's breadth of contributions and his role in establishing Greenwich Observatory as an institution.

\textsc{Hughes, David W.} \emph{The Tudor Astronomical System: Tycho Brahe and the Heliocentric System}. Oxford University Press, Oxford, 2000. Modern treatment of celestial mechanics and orbital theory, with extensive chapters on Kepler's laws and perturbation theory. Chapter 4 includes analysis of Halley's cometary calculations and refinements by Delaunay and Adams. Provides technical context for Chapter 11's discussion of cometary perturbations and the refinement of orbital mechanics in the 18th century.

\textsc{Chapin, Seymour L.} \emph{Nutation and the Earth's Axis}. Willmann-Bell, Richmond, VA, 1995. Comprehensive historical and technical treatment of 18th-century transit observations and international astronomical cooperation. Chapters 6--9 detail the Venus transit expeditions of 1761 and 1769, the observational procedures, and the international coordination required. Essential for Chapter 11's discussion of the realized goal of Halley's transit parallax method and its role in measuring the astronomical unit with unprecedented precision.

\textsc{Chapman, Allan.} \emph{Astronomical Instruments and Their Users: Tycho Brahe to William Herschel}. Variorum Publications, Aldershot, 1990. Collection of technical studies on precision instruments and their makers. Includes detailed discussion of Halley's magnetic variation surveys (pp. 120--135), the instruments used, and the methodology of his Atlantic voyages. Emphasizes Halley's pioneering role in establishing observational geomagnetism as a scientific discipline.

\textsc{Merrill, Ronald T. and McElhinny, Michael W.} \emph{The Magnetic Field of the Earth: Paleomagnetism, the Core, and the Deep Mantle}. Academic Press, New York, 1985. Modern comprehensive treatment of Earth's magnetism. Chapter 2 provides historical overview including Halley's contributions and the evolution of understanding Earth's magnetic field from the 17th century onward.

\textsc{Sykes, Frederick Henry.} \emph{The Life and Work of Edmond Halley}. Oxford University Press, Oxford, 1926. Biographical study emphasizing Halley's actuarial and demographic work, including detailed analysis of his life tables derived from Breslau mortality records. Chapter 8 contextualizes Halley's statistical methods within the broader development of vital statistics and their practical applications to insurance and pension funds.

\textsc{Willmoth, Frances} (ed.). \emph{Flamsteed's Stars: The Biographical Work of John Flamsteed}. Boydell Press, Woodbridge, 2002. Scholarly biographical study integrating Flamsteed's autobiography, his scientific correspondence, and analysis of his contributions to positional astronomy.

\textsc{Willmoth, Frances.} \emph{The Biographical Work of John Flamsteed}. Greshop Press, London, 1992. Detailed examination of Flamsteed as an observer and calculator, tracing his development as an astronomer and his instrumental innovations.

\textsc{Dreyer, John L. E.} \emph{Tycho Brahe: A Picture of Scientific Life and Work in the Sixteenth Century}. Adam and Charles Black, Edinburgh, 1890. Classical biography of Tycho Brahe, whose observational program and precision standards established the template that Flamsteed would follow and improve upon.

\textsc{Brahe, Tycho.} \emph{Astronomiae Instauratae Mechanica}. Hafniae, Copenhagen, 1602. Tycho's own description of his instruments and observational methods at Uraniborg. Essential primary source for understanding the state of observational astronomy before Flamsteed.

\textsc{Maskelyne, Nevil (ed.).} \emph{The Nautical Almanac and Astronomical Ephemeris}. Government Printing Office, London, 1767--present. First annual publication issued in 1767 for the year 1768. Contains lunar positions, solar positions, stellar positions, Jupiter's satellites, refraction tables, and parallax values. The structure of the early Almanacs (1767--1800) is discussed in detail in Chapter 10. Modern Nautical Almanacs continue to serve navigators, astronomers, and scientific researchers worldwide, making it one of the longest continuously published scientific tables.

\section{Citation and Reference Conventions}

Throughout this book, citations employ Chicago author-date style, with references formatted as author-year keys (e.g., \textcite{Baily1835}, \textcite{Chapman1996}). Full bibliographic information is available in the \texttt{references.bib} BibTeX file accompanying this volume. Readers pursuing a particular theme can follow cross-references within each section above; related discussions in the main text are indicated via chapter references (e.g., \cref{ch:founding-observatory}, \cref{ch:mural-arc-transits}).

For primary sources such as Flamsteed's observation logs and correspondence, citations refer to the archival holdings at the National Maritime Museum, Greenwich, or to published editions such as the \emph{Collected Works} series. Readers should consult institutional repositories directly for access to original manuscripts.

\printbibliography
