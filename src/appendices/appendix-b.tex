\chapter{The Airy Transit Circle: Technical Specifications and Data Reduction}
\label{app:airy-transit-circle}

\section{Instrument Specifications}

The Airy transit circle installed at Greenwich Observatory in 1851 was constructed by Troughton \& Simms, the renowned London instrument makers. Table \ref{tab:airy-specs} summarizes its physical parameters and capabilities.

\begin{table}[htbp]
  \centering
  \caption{Physical specifications of the Airy transit circle.}
  \label{tab:airy-specs}
  \small
  \begin{tabular}{lr}
    \toprule
    \textbf{Parameter} & \textbf{Value} \\
    \midrule
    Objective aperture & 170 mm (6.7 inches) \\
    Focal length & 2.4 m (8 feet) \\
    Magnification & $\times 120$ (typical eyepiece) \\
    Vertical circle diameter & 1.37 m (4.5 feet) \\
    Circle graduation interval & 1 arcminute \\
    Reading microscope resolution & 1 arcsecond \\
    Reticule wires & 5 vertical, 1 horizontal \\
    Wire separation (vertical) & 30 arcseconds (central to adjacent) \\
    Main axis diameter (steel) & 25 mm (1 inch) \\
    Main bearing V-angle & 90° (approximately) \\
    Total weight & $\approx 3$ tons \\
    Mounting orientation & Meridian plane (true north-south) \\
    \bottomrule
  \end{tabular}
\end{table}

\section{Optical Design and Ray Tracing}

The objective lens is an achromatic doublet consisting of a concave lens of dense flint glass cemented to a convex lens of crown glass. The design corrects for chromatic aberration by bringing red and blue light to a focus at the same point, while green light focuses slightly off-axis. The focal plane contains the reticule—five vertical wires and one horizontal wire—mounted on a fixed frame.

The eyepiece is a Ramsden design or later Kellner design, providing approximately 120$\times$ magnification. The combination of $f=2.4$ m focal length and 120$\times$ magnification produces an exit pupil diameter of roughly 1.4 mm, placing the observer's eye directly at or beyond the exit pupil. This design minimizes vignetting and provides a large, accessible eye point.

The angular separation between the central wire and the adjacent wires is approximately 30 arcseconds. At the focal plane, this corresponds to:
\[
  \Delta h = 2.4 \text{ m} \times \tan(30'') \approx 2.4 \text{ m} \times (30/206265) \text{ rad} \approx 0.35 \text{ mm}
\]

A star's image, when in focus, is roughly 0.5 arcseconds in diameter (in good seeing conditions), corresponding to about 0.006 mm at the focal plane. The 0.35 mm separation between wires is therefore much larger than a stellar image, allowing the observer to unambiguously identify which wire the star crosses.

\section{Refraction Correction}

Refraction is the bending of light as it passes through Earth's atmosphere. The amount of refraction depends on the altitude angle $h$ of the observation. For the altitude range used in transit circle observations (typically $h > 30°$), the refraction can be approximated by:
\[
  R(\text{arcsec}) \approx 58.3 \cot(h)
\]
where $h$ is the observed altitude in degrees. At the zenith ($h = 90°$), the refraction is approximately 0 arcseconds; at $h = 45°$, it is about 58 arcseconds; at $h = 30°$, it is about 101 arcseconds.

A more accurate formula, accounting for temperature and pressure variations, is:
\[
  R = R_0 \frac{P}{P_0} \frac{T_0}{T}
\]
where $R_0$ is the standard refraction, $P$ and $T$ are the observed pressure and absolute temperature, and $P_0 = 101.325$ kPa and $T_0 = 288.15$ K are the standard sea-level values.

Airy maintained tables of refraction values computed for standard conditions and applied corrections based on the barometer and thermometer readings taken at the time of observation. For stars observed near the zenith, the refraction was typically a few arcseconds and well-constrained. For stars observed at low altitudes, the refraction could be large (tens of arcseconds) and uncertain, reflecting the variability of atmospheric conditions.

\section{Collimation Maintenance}

Collimation is the alignment of the optical axis of the telescope with the geometric axis of rotation. Any departure from perfect alignment introduces systematic errors in both right ascension (from the wire's tilting away from the meridian plane) and declination (from the optical axis tilting away from the vertical).

Airy maintained collimation through repeated observations of an artificial star created by a fixed illuminated slit placed at the focal point of a separate fixed telescope. The position of the artificial star image relative to the transit circle's reticule was measured, and any deviation from the expected position revealed a collimation error. The collimation error was then corrected either mechanically (by slightly tilting the optical tube) or computationally (by adding a correction term to all subsequent observations).

The collimation procedure was typically performed daily or every few days. By maintaining records of the collimation corrections, Airy could detect slow drift of the optical axis and maintain collimation accuracy to better than 1 arcsecond.

\section{Personal Equation Determination}

The personal equation of an observer is the systematic time offset between when an event occurs (the star crossing the wire) and when the observer records it. This offset reflects the observer's reaction time and varies from individual to individual.

Airy determined personal equations by having multiple observers watch the same series of stars and record their transit times. The differences between observers' times, when averaged over many stars, yielded the personal equations. For example, if Airy's times averaged 0.32 seconds earlier than an arithmetical mean of all observers, and his assistant's times averaged 0.18 seconds later, then:
\begin{align*}
  \text{Airy's personal equation} &= -0.32 \text{ s} \\
  \text{Assistant's personal equation} &= +0.18 \text{ s}
\end{align*}

(The negative sign for Airy indicates he records earlier than the mean; the positive sign for the assistant indicates he records later.)

In practice, personal equations were not perfectly constant—they varied with fatigue, lighting conditions, and the observer's state of alertness. Airy recalculated personal equations periodically (typically monthly) and applied the most recent values to all observations.

\section{Detailed Data Reduction: A Multi-Star Example}

To illustrate the full data reduction process, consider a set of observations from a single night in 1855. Three stars were observed, with both Airy and his assistant recording transit times and altitudes.

\textsc{Raw observation data (October 12, 1855):}

\begin{table}[htbp]
  \centering
  \caption{Raw observation data from October 12, 1855.}
  \label{tab:raw-obs-1855}
  \small
  \begin{tabular}{lrrrr}
    \toprule
    \textbf{Star} & \textbf{Airy Time} & \textbf{Asst Time} & \textbf{Airy Alt} & \textbf{Asst Alt} \\
    \midrule
    Polaris & $1^h 34^m 22^s$ & $1^h 34^m 22.4^s$ & $47° 22' 18''$ & $47° 22' 15''$ \\
    Vega & $18^h 58^m 12^s$ & $18^h 58^m 12.3^s$ & $69° 15' 8''$ & $69° 15' 10''$ \\
    Altair & $20^h 3^m 28^s$ & $20^h 3^m 28.2^s$ & $56° 48' 32''$ & $56° 48' 34''$ \\
    \bottomrule
  \end{tabular}
\end{table}

\textsc{Applying personal equation corrections:}

Using Airy's personal equation of $-0.32^s$ and the assistant's personal equation of $+0.18^s$:

\begin{table}[htbp]
  \centering
  \caption{Times corrected for personal equation.}
  \label{tab:corrected-times}
  \small
  \begin{tabular}{lrr}
    \toprule
    \textbf{Star} & \textbf{Airy Corrected} & \textbf{Asst Corrected} \\
    \midrule
    Polaris & $1^h 34^m 22.32^s$ & $1^h 34^m 22.22^s$ \\
    Vega & $18^h 58^m 12.32^s$ & $18^h 58^m 12.12^s$ \\
    Altair & $20^h 3^m 28.32^s$ & $20^h 3^m 28.02^s$ \\
    \bottomrule
  \end{tabular}
\end{table}

The mean times are:
\begin{align*}
  t_{\text{Polaris}} &= \frac{22.32 + 22.22}{2} = 22.27 \text{ s} \\
  t_{\text{Vega}} &= \frac{12.32 + 12.12}{2} = 12.22 \text{ s} \\
  t_{\text{Altair}} &= \frac{28.32 + 28.02}{2} = 28.17 \text{ s}
\end{align*}

\textsc{Converting to right ascension:}

Using the sidereal time at midnight GMT for October 12, 1855 ($\alpha_0 = 23^h 48^m 15^s$), we convert each mean time to sidereal time:
\begin{align*}
  \alpha_{\text{LST, Polaris}} &= \alpha_0 + 1.0027379 \times 1^h 34^m 22.27^s = 23^h 48^m 15^s + 1^h 34^m 45^s \\
  &= 1^h 23^m 0^s \\
  \alpha_{\text{LST, Vega}} &= 23^h 48^m 15^s + 1.0027379 \times 18^h 58^m 12.22^s = 23^h 48^m 15^s + 19^h 2^m 28^s \\
  &= 18^h 50^m 43^s \\
  \alpha_{\text{LST, Altair}} &= 23^h 48^m 15^s + 1.0027379 \times 20^h 3^m 28.17^s = 23^h 48^m 15^s + 20^h 7^m 51^s \\
  &= 19^h 56^m 6^s
\end{align*}

\textsc{Converting altitudes to declinations:}

For Polaris, the observed altitude is $47° 22' 18''$ (averaged: $47° 22' 16.5''$). Refraction correction at this altitude is approximately $42''$:
\begin{align*}
  h_{\text{true}} &= 47° 22' 16.5'' - 42'' = 47° 21' 34.5'' \\
  z &= 90° - 47° 21' 34.5'' = 42° 38' 25.5'' \\
  \delta_{\text{Polaris}} &= 51° 28' 40'' - 42° 38' 25.5'' = 8° 50' 14.5''
\end{align*}

For Vega ($h_{\text{obs}} \approx 69° 15' 9''$, refraction $\approx 21''$):
\begin{align*}
  h_{\text{true}} &= 69° 14' 48'' \\
  z &= 20° 45' 12'' \\
  \delta_{\text{Vega}} &= 51° 28' 40'' - 20° 45' 12'' = 30° 43' 28''
\end{align*}

For Altair ($h_{\text{obs}} \approx 56° 48' 33''$, refraction $\approx 31''$):
\begin{align*}
  h_{\text{true}} &= 56° 48' 2'' \\
  z &= 33° 11' 58'' \\
  \delta_{\text{Altair}} &= 51° 28' 40'' - 33° 11' 58'' = 18° 16' 42''
\end{align*}

\section{Quality Metrics and Internal Consistency Checks}

Airy employed several methods to check the quality of his observations:

1. \textsc{Residuals between observers:} The difference in the two observers' times (after correction) should scatter randomly around zero. Large systematic differences indicated that the personal equations needed recalibration.

2. \textsc{Replicate observations:} The same star was observed multiple times over weeks or months. The mean declination from all observations should scatter with scatter consistent with the expected random error (typically a few tenths of an arcsecond). Large outliers indicated either observational error or actual stellar motion (proper motion or parallax).

3. \textsc{Altitude range check:} Stars were intentionally observed at a range of altitudes to test whether refraction corrections and systematic effects were properly accounted for. If a star's declination appeared to depend on its altitude at observation, this indicated a problem with refraction modeling.

4. \textsc{Circle reversals:} The transit circle could be reversed end-for-end, so that the axis of rotation was swapped. Observations taken in both positions allowed systematic errors in the graduations to be detected. If the two reversals gave different positions for the same star, the difference pointed to a graduation error.

\section{The Azimuth Error: Detection and Correction}

An azimuth error—a small departure from true north-south orientation—manifests as a time error in transit observations that depends on the star's declination. If the instrument is tilted east or west by a small angle $\epsilon$, the transit time for a star at declination $\delta$ will be delayed by:
\[
  \Delta t = -\epsilon \sin(\delta) / (15°/\text{hour})
\]

Stars near the celestial equator ($\delta \approx 0°$) show little effect; stars near the celestial pole show the largest effect. By measuring transit times for a series of stars at known declinations and fitting a linear model $\Delta t = a + b \sin(\delta)$, Airy could extract the azimuth error. The fit yields $b$, which gives the azimuth error: $\epsilon = 15 \times b$ (degrees per hour $\times$ radians/hour).

A typical azimuth error might be a few arcseconds. Once detected, it could be corrected mechanically by loosening the mounting bolts and slightly tilting the entire instrument, then re-tightening. Alternatively, the azimuth error could be tracked and applied as a correction to all future observations.

\section{Long-Term Instrument Stability}

Over its decades of use, the Airy transit circle was monitored for long-term drift. Measurements of:
- The collimation (via artificial star observations)
- The level (via striding level measurements)
- The zero point of the graduated circle (via repeated observations of standard stars)

all showed slow but detectable drift over years. Temperature changes (seasonal and daily), gravitational settling of the mounting structure, and wear of the pivots all contributed. Airy and his successors maintained detailed records of these drifts and applied corrections to ensure that observations from different epochs could be meaningfully compared.

This attention to instrumental drift set a new standard for observational astronomy. Instruments were no longer viewed as static references but as dynamical systems requiring continuous monitoring and maintenance.
