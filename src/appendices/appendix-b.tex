\chapter{Instrument Specifications}
\label{app:instrument-specs}

This appendix provides comprehensive reference tables for major instruments discussed in the book, organized by functional category. Each table lists instruments historically documented, with key technical parameters enabling quantitative comparison across eras and techniques.

\section{Meridian Instruments}

Meridian instruments are transit circles, quadrants, and sectors designed to measure celestial positions along the observer's meridian (north-south line through the zenith). Table~\ref{tab:meridian-instruments} presents the key specifications for instruments spanning two centuries of positional astronomy at Greenwich and elsewhere. The progression shows increasing optical aperture (light-gathering), finer graduation intervals (angular resolution), and improved accuracy. The Airy transit circle represents the peak of mechanical precision before electronic recording.

\begin{sidewaystable}
  \centering
  \caption{Historical meridian instruments: date, maker, aperture, focal length, circle diameter, graduation, estimated accuracy, and location/notes.}
  \label{tab:meridian-instruments}
  \begin{tabularx}{\linewidth}{lllllllX}
    \toprule
    \textbf{Instrument} & \textbf{Date} & \textbf{Maker} & \textbf{Aperture} & \textbf{Circle} & \textbf{Grad.} & \textbf{Accuracy} & \textbf{Notes} \\
    \midrule
    Flamsteed mural arc & 1689 & Sharp & 130 mm & 2.1 m & $1'$ & $\pm 10''$ & Greenwich; sighted to zenith \\
    Halley transit instr. & 1710 & Instrument maker & 100 mm & 1.5 m & $1'$ & $\pm 15''$ & Greenwich; equatorial mounting \\
    Bradley zenith sector & 1727 & George Graham & 80 mm & — & — & $\pm 1''$ & Greenwich; discovery of aberration \\
    Bradley transit instr. & 1750 & Graham & 120 mm & 2.0 m & $1'$ & $\pm 5''$ & Greenwich; precise RA determination \\
    Bird 8-ft quadrant & 1750 & John Bird & 240 mm & 2.4 m radius & $1'$ & $\pm 8''$ & Oxford; portable design \\
    Airy transit circle & 1851 & Troughton \& Simms & 170 mm & 1.37 m & $1'$ & $\pm 0.5''$ & Greenwich; micrometers for readings \\
    Photographic ZT & 1900 & Various & 150 mm & — & digital & $\pm 0.3''$ & Greenwich; automated recording \\
    \bottomrule
  \end{tabularx}
\end{sidewaystable}

\section{Telescopes}

Refractors and reflectors used for both positional astronomy and spectroscopy, ranging from small refractors to the largest reflectors of the 20th century. Table~\ref{tab:telescopes} summarizes key optical parameters. Note that speculum metal (copper-tin alloy) mirrors required regular re-polishing; modern aluminum and silver coatings (post-1900) eliminate this maintenance burden.

\begin{sidewaystable}
  \centering
  \caption{Historical telescopes: aperture, focal length, mounting type, designer/maker, and location/era.}
  \label{tab:telescopes}
  \begin{tabularx}{\linewidth}{lllllX}
    \toprule
    \textbf{Telescope} & \textbf{Aperture} & \textbf{Focal Length} & \textbf{Type} & \textbf{Maker/Era} & \textbf{Location} \\
    \midrule
    Newton's reflector & 30 mm & 160 mm & Newtonian & Newton 1668 & Cambridge \\
    Herschel 20-ft reflect. & 190 mm & 6.1 m & Speculum metal & Herschel 1783 & Slough \\
    Herschel 40-ft reflect. & 490 mm & 12.2 m & Speculum metal & Herschel 1789 & Slough \\
    Grubb 28-inch refract. & 710 mm & 10.4 m & Achromatic & Grubb 1893 & Greenwich (Great Equatorial) \\
    Isaac Newton Telescope & 980 mm & 13.7 m & Reflector & Grubb/RGO 1967 & Herstmonceux, then La Palma \\
    \bottomrule
  \end{tabularx}
\end{sidewaystable}

\section{Navigational Instruments}

Sextants, quadrants, and octants used by navigators to measure altitude angles for celestial navigation are listed in Table~\ref{tab:navigation-instruments}. All reading precisions assume good atmospheric conditions and experienced observers. Actual shipboard accuracy was typically 3–5 times worse due to ship motion and rough seas.

\begin{sidewaystable}
  \centering
  \caption{Navigational instruments: type, approximate era of use, typical maker, arc range, and reading precision.}
  \label{tab:navigation-instruments}
  \begin{tabularx}{\linewidth}{lllllX}
    \toprule
    \textbf{Type} & \textbf{Era} & \textbf{Makers} & \textbf{Arc Range} & \textbf{Reading Precision} & \textbf{Notes} \\
    \midrule
    Hadley octant & 1730--1780 & Various & $0° – 90°$ & $2–3'$ & Double reflection; portable \\
    Hadley quadrant & 1740--1850 & Various & $0° – 90°$ & $2–3'$ & Larger frame, improved steadiness \\
    Ramsden sextant & 1770--1900 & Ramsden, others & $0° – 120°$ & $20–30''$ & Vernier scale; improved optics \\
    Troughton sextant & 1800--1950 & Troughton \& Simms & $0° – 120°$ & $10–20''$ & Micrometer screw; marine standard \\
    20th-c. sextant & 1900–present & Multiple & $0° – 120°$ & $10''$ & Bubble level; minimal maintenance \\
    \bottomrule
  \end{tabularx}
\end{sidewaystable}

\section{Chronometers}

Portable mechanical timepieces designed for sea trials, showing the progression from Harrison's experimental models to production marine chronometers. Table~\ref{tab:chronometers} documents the key specifications. The progression from H1 to H5 shows systematic improvements. Rates (error per day) decreased from $\pm 5$ seconds to $\pm 0.1$ seconds—a 50-fold improvement enabling reliable ocean navigation.

\begin{sidewaystable}
  \centering
  \caption{Historical chronometers: maker, date, escapement type, compensation method, and sea trial performance.}
  \label{tab:chronometers}
  \begin{tabularx}{\linewidth}{lllllX}
    \toprule
    \textbf{Chronometer} & \textbf{Date} & \textbf{Escapement} & \textbf{Compensation} & \textbf{Rate/Day} & \textbf{Notes} \\
    \midrule
    Harrison H1 & 1735 & Linked balance & None & $\pm 5$ s & Experimental; very accurate for land \\
    Harrison H2 & 1741 & Similar & Temperature trim & $\pm 3$ s & Improvement but heavy oscillation \\
    Harrison H3 & 1759 & Bi-metallic balance & Integral & $\pm 1$ s & Major breakthrough; complex \\
    Harrison H4 & 1759 & Detent & Diamond pallets & $\pm 0.2$ s & Nearly watches; highly accurate \\
    Harrison H5 & 1772 & Detent & Bi-metallic & $\pm 0.1$ s & Final refinement; production model \\
    Kendall K1 & 1769 & Copy of H4 & Diamond pallets & $\pm 0.3$ s & Copy for naval testing \\
    Arnold marine chron. & 1780+ & Chronometer & Bi-metallic & $\pm 0.5$ s & Commercial production; reliable \\
    Earnshaw chron. & 1790+ & Spring detent & Bi-metallic & $\pm 0.5$ s & Simpler design; widely used \\
    \bottomrule
  \end{tabularx}
\end{sidewaystable}

\section{Clocks}

Fixed observatory clocks and standard clocks used for timekeeping, spanning mechanical pendulums to atomic oscillators. Table~\ref{tab:clocks} summarizes the dramatic improvement in accuracy: from seconds per day (17th century) to seconds per million years (20th century)—a factor of $\sim 10^{10}$ improvement over 350 years.

\begin{sidewaystable}
  \centering
  \caption{Historical clocks: maker, date, escapement, compensation, rated accuracy, location, and era notes.}
  \label{tab:clocks}
  \begin{tabularx}{\linewidth}{lllllX}
    \toprule
    \textbf{Clock} & \textbf{Date} & \textbf{Escapement} & \textbf{Compensation} & \textbf{Accuracy} & \textbf{Notes} \\
    \midrule
    Tompion regulators & 1676 & Anchor & Gridiron pendulum & $\pm 1$ s/day & Greenwich; set standard for precision \\
    Graham regulator & 1715 & Anchor & Mercury comp. & $\pm 2$ s/week & Improved thermal stability \\
    Shepherd master & 1880 & Spring detent & Bi-metallic & $\pm 0.1$ s/day & Electric impulse drive; very stable \\
    Shortt free-pendulum & 1920 & Free pendulum & Vacuum chamber & $\pm 1$ s/month & Electromagnetic coupling; nearly ideal \\
    Cesium-133 clock & 1955+ & Hyperfine trans. & Atomic resonance & $\pm 1$ s/30M years & NIST-F1; primary standard \\
    Hydrogen maser & 1960+ & Maser resonance & Cavity tuning & $\pm 1$ s/1M years & Secondary standard; portable \\
    \bottomrule
  \end{tabularx}
\end{sidewaystable}

