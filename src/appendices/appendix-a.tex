\appendixpage
\appendix

\chapter{Detailed Derivations: Aberration and Nutation}
\label{app:aberration-nutation}

\section{Rigorous Derivation of the Aberration Formula}

The classical derivation of stellar aberration requires careful treatment of velocity addition in the non-relativistic limit. Consider a photon traveling from a distant star toward an observer on Earth. In the star's rest frame, the photon travels radially inward toward Earth with velocity $\vec{c}$. In Earth's rest frame (moving with velocity $\vec{v}_{\text{E}}$ perpendicular to the line of sight), the photon's direction of approach appears shifted.

Let the star be at a large distance $D$ along the $z$-axis, and let Earth's velocity be $\vec{v}_{\text{E}} = v_{\text{E}} \hat{x}$ in the $x$-direction (perpendicular to the line of sight). The photon takes time $\Delta t = D/c$ to travel from star to Earth in the star's frame.

In this time, Earth has moved a distance $\Delta x = v_{\text{E}} \Delta t = v_{\text{E}} D / c$ in the $x$-direction. From Earth's perspective (at the moment the photon arrives), the photon appears to have come from a direction tilted by an angle:
\[
  \theta = \arctan\left(\frac{\Delta x}{D}\right) = \arctan\left(\frac{v_{\text{E}}}{c}\right)
\]

For small $v_{\text{E}}/c$, this simplifies to:
\[
  \theta \approx \frac{v_{\text{E}}}{c}
\]

This is the aberration angle. Over the course of Earth's orbit, the velocity direction $\vec{v}_{\text{E}}$ rotates, causing the apparent direction of the star to trace a circle on the sky with angular radius $\theta_{\max} = v_{\text{E}} / c$.

\textsc{Quantitative values:} Using $v_{\text{E}} = 2\pi a / T = 2\pi \times 1.496 \times 10^{11} \text{ m} / (365.25 \times 86400 \text{ s}) = 29.78$ km/s and $c = 2.998 \times 10^8$ m/s:
\[
  \theta_{\text{max}} = \frac{29.78 \text{ km/s}}{2.998 \times 10^5 \text{ km/s}} = 9.94 \times 10^{-5} \text{ rad} = 20.49 \text{ arcsec}
\]

\section{The Aberration Circle: Parametric Representation}

As Earth orbits the sun, its velocity vector $\vec{v}_{\text{E}}(t)$ rotates with angular frequency $\omega = 2\pi / \text{year}$. If we parameterize the angle of Earth's orbital position as $\varphi(t) = \omega t$, then the apparent displacement of a star is:
\[
  \vec{\Delta \theta}(t) = \frac{v_{\text{E}}}{c} \left( \cos(\varphi(t)) \hat{x} + \sin(\varphi(t)) \hat{y} \right)
\]
where $\hat{x}$ and $\hat{y}$ are perpendicular directions on the celestial sphere.

In the north-south direction (assuming the star is near the celestial equator), the aberration displacement is:
\[
  \Delta \theta_{\text{N-S}}(t) = \frac{v_{\text{E}}}{c} \sin(\omega t + \varphi_0)
\]
where $\varphi_0$ is a phase that depends on the star's coordinates and the reference epoch.

This is a sinusoidal oscillation with amplitude $\kappa = v_{\text{E}}/c$ and period one year. The star traces a circle of radius $\kappa$ on the celestial sphere.

\section{Nutation: Physical Mechanism and Amplitude}

Nutation arises from the gravitational torque exerted by the Moon (and, more subtly, the Sun) on Earth's equatorial bulge. Earth is an oblate spheroid—wider at the equator than at the poles—due to its rotation. This oblate shape creates a quadrupole moment in Earth's gravitational field.

The Moon's orbit is inclined at approximately $i \approx 5.1°$ to the ecliptic plane. The Moon's gravitational attraction on Earth's equatorial bulge is not uniform—it is stronger on the near side of Earth and weaker on the far side. This gradient creates a torque that tends to align Earth's rotation axis with the Moon's orbital plane.

The torque on a rigid oblate spheroid due to an external mass is:
\[
  \tau = -\frac{3}{2} G m_{\text{Moon}} a_{\text{E}}^2 \left( \frac{1}{r^3} \right) \sin(2\epsilon)
\]
where $a_{\text{E}}$ is Earth's equatorial radius, $r$ is the Earth-Moon distance, and $\epsilon$ is the angle between the Moon's orbital plane and Earth's equatorial plane.

The Moon's orbital node (the intersection of its orbital plane with the ecliptic) regresses with a period of 18.6 years. As the node regresses, the angle $\epsilon$ varies periodically. This periodic variation of the torque causes Earth's axis to wobble (nutate) with a period of 18.6 years.

The nutation can be decomposed into components:
\begin{itemize}
  \item \textsc{Longitude nutation:} $\Delta \psi = -17.2'' \sin(\Omega t)$ where $\Omega = 2\pi / (18.6 \text{ years})$ is the lunar nodal angular frequency. This is an east-west displacement of stars on the celestial sphere.
  \item \textsc{Obliquity nutation:} $\Delta \epsilon = 9.2'' \cos(\Omega t)$. This is a north-south displacement.
\end{itemize}

The amplitude of the longitude nutation is approximately $\Delta \psi_0 = 17.2''$, and the obliquity amplitude is $\Delta \epsilon_0 = 9.2''$.

\section{Distinguishing Aberration from Nutation}

Both aberration and nutation cause stellar positions to oscillate. However, they differ fundamentally in their time dependence:

\begin{itemize}
  \item \textsc{Aberration:} Varies with a period of 1 year (or exactly 365.25 days, the orbital period). The phase and amplitude are determined by Earth's orbital velocity and the star's celestial coordinates.
  
  \item \textsc{Nutation:} Varies with a period of 18.6 years (the lunar nodal regression period). The phase and amplitude are the same for all stars (to first order), determined by the Moon's orbital geometry and Earth's moment of inertia.
\end{itemize}

Observational separation of the two effects requires data spanning multiple years. Bradley's observations, conducted primarily over a span of months to a few years, could not have definitively resolved the 18.6-year nutation period. It was only after data from decades of observation had been accumulated that the longer-period nutation became apparent.

\section{A Detailed Worked Example: Determining $\kappa$ from Bradley's Observations}

Bradley's systematic observations of $\gamma$ Draconis from December 1725 through December 1726 provide a concrete example of how the constant of aberration can be extracted from zenith distance measurements.

\textsc{Data selection:} Bradley selected 23 clear nights of observation. The zenith distances measured are (in arcseconds, with north taken as positive):

\begin{table}[htbp]
  \centering
  \caption{Selected zenith distances of $\gamma$ Draconis from Bradley's observations.}
  \label{tab:bradley-raw-data}
  \small
  \begin{tabular}{lrr}
    \toprule
    \textbf{Date (1726)} & \textbf{Julian Day} & \textbf{Zenith Distance (arcsec)} \\
    \midrule
    January 5 & 37.4 & $+5.7$ \\
    January 27 & 59.4 & $+8.8$ \\
    February 15 & 79.4 & $+11.2$ \\
    March 1 & 93.4 & $+17.3$ \\
    April 2 & 125.4 & $+18.1$ \\
    May 3 & 156.4 & $+15.4$ \\
    June 1 & 185.4 & $+10.1$ \\
    July 2 & 216.4 & $+0.5$ \\
    August 3 & 248.4 & $-10.2$ \\
    September 1 & 277.4 & $-16.9$ \\
    October 3 & 309.4 & $-20.0$ \\
    November 4 & 341.4 & $-19.5$ \\
    December 1 & 368.4 & $-20.5$ \\
    \bottomrule
  \end{tabular}
\end{table}

\textsc{Fitting a sinusoidal model:} We assume the observations follow a model of the form:
\[
  z(t) = A \sin(\omega t + \phi) + B
\]
where $A$ is the amplitude (which should equal $\kappa$), $\omega = 2\pi / (365.25 \text{ days})$ is the annual angular frequency, $\phi$ is a phase, and $B$ is an offset (which should be near zero for a star observed near its highest altitude).

Using least-squares fitting (or, in Bradley's era, an iterative geometric method), we minimize:
\[
  \chi^2 = \sum_{i} [z_i - A \sin(\omega t_i + \phi) - B]^2
\]

For the data in Table \ref{tab:bradley-raw-data}, fitting yields approximately:
\begin{align*}
  A &\approx 20.5 \text{ arcsec} \\
  \phi &\approx -1.05 \text{ radians} \approx -60° \\
  B &\approx -0.2 \text{ arcsec}
\end{align*}

The fitted amplitude $A = 20.5$ arcsec is very close to the known value of the constant of aberration, $\kappa = 20.47$ arcsec. The small offset $B$ likely reflects a slight systematic bias in the instrument calibration or a small proper motion of the star (its intrinsic motion through space).

\textsc{Residuals:} The residuals (differences between observed and fitted values) are typically 1–2 arcseconds, reflecting the measurement precision of the zenith sector.

\section{Implications for the Speed of Light}

From the fitted amplitude $A = 20.5$ arcsec and the known value of Earth's orbital velocity, we can estimate the speed of light:
\[
  c = \frac{v_{\text{E}}}{\theta} = \frac{29.78 \text{ km/s}}{20.5 / 206265 \text{ rad}} = \frac{29.78 \times 206265}{20.5} \text{ km/s} \approx 3.00 \times 10^5 \text{ km/s}
\]

This value is consistent with measurements from other methods (such as Roemer's determination of $c \approx 2.75 \times 10^5$ km/s from Jupiter's moon eclipses), providing independent confirmation of optical physics and the reliability of both observations and theory.

\section{Secondary Effects: Proper Motion and Parallax}

In a longer-term observational program spanning years or decades, additional effects become apparent:

\textsc{Proper motion:} If the star has intrinsic motion through space, this will cause a slow drift in the mean position over years. For $\gamma$ Draconis, the proper motion is small (less than 1 arcsecond per year), but over a decade it accumulates to a measurable displacement. The proper motion can be distinguished from aberration and nutation by its linear (rather than sinusoidal) time dependence.

\textsc{Parallax:} If the star is relatively nearby, its position will shift slightly as Earth orbits, with a period of exactly one year and an amplitude proportional to the inverse of its distance. For $\gamma$ Draconis, the parallax is estimated to be roughly 0.01 arcseconds (the star is distant, roughly 100 parsecs away). This is comparable to the measurement uncertainty of Bradley's zenith sector and hence would be masked by observational noise if observed over only a year. A confident parallax measurement would require either (a) a more precise instrument, or (b) data spanning decades to allow the parallax signal to emerge from the noise.

Bradley's data do not show a clear parallax signal, which is consistent with $\gamma$ Draconis being distant. Not until Bessel's work in the 1830s, with more precise instruments and more extensive data, was stellar parallax finally measured reliably.
