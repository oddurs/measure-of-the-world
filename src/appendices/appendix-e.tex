\chapter{Appendix E: Time Distribution—Technical Details and Calculations}
\label{app:time-distribution}

\section*{Time Ball Mechanics and Free Fall}

The time ball mechanism relies on precise timing of an electromagnetic release and the subsequent free fall of the ball. When the master clock reaches the designated moment (typically 1:00 PM), a relay closes a circuit, energizing the electromagnets that hold the ball. The magnets disengage, releasing the ball.

For a sphere of mass $m$ falling from rest under gravity, the distance fallen as a function of time is

\[
  h(t) = \frac{1}{2} g t^2,
\]

where $g \approx 9.8 \text{ m/s}^2$ is the acceleration due to gravity. For a ball suspended approximately 3 meters above the catch point, the time to fall is

\[
  t = \sqrt{\frac{2h}{g}} = \sqrt{\frac{2 \times 3}{9.8}} \approx 0.78 \text{ seconds}.
\]

Wait, this seems too long. Let me reconsider the height. Historical records indicate the ball drops from a height of approximately 1-1.5 meters, not 3 meters. For $h = 1.3 \text{ m}$:

\[
  t = \sqrt{\frac{2 \times 1.3}{9.8}} \approx 0.51 \text{ seconds}.
\]

The impact on the catch mechanism at the bottom is sudden, producing a distinctive sound that assists observers in timing. The final velocity just before impact is

\[
  v = gt = 9.8 \times 0.51 \approx 5 \text{ m/s}.
\]

This velocity is sufficient to produce an audible click when the ball strikes the catch, providing an acoustic signal to supplement the visual one—important for observers who might blink or become distracted at the exact moment of drop.

\section*{Light Travel Time and Angular Resolution}

For an observer at horizontal distance $d$ from the ball and viewing angle $\alpha$ above the horizon, the actual distance to the ball is

\[
  r = \sqrt{d^2 + h^2},
\]

where $h$ is the vertical height of the ball above the observer. The light travel time is

\[
  t_{\text{light}} = \frac{r}{c} = \frac{\sqrt{d^2 + h^2}}{c}.
\]

For $d = 500 \text{ m}$ and $h = 10 \text{ m}$ (a typical height of a ship's bridge):

\[
  r = \sqrt{500^2 + 10^2} = \sqrt{250100} \approx 500.1 \text{ m}.
\]

The light travel time is

\[
  t_{\text{light}} = \frac{500.1}{3 \times 10^8} \approx 1.67 \text{ microseconds}.
\]

This is utterly negligible compared to the 0.1-second accuracy of the time ball system.

However, angular resolution is a constraint. The ball's angular size as seen by the observer is approximately

\[
  \theta \approx \frac{D}{r} = \frac{1 \text{ m}}{500 \text{ m}} = 0.002 \text{ radians} \approx 0.11 \text{ degrees} \approx 400 \text{ arcseconds}.
\]

For an observer at 1 kilometer distance:

\[
  \theta \approx \frac{1 \text{ m}}{1000 \text{ m}} = 0.001 \text{ radians} \approx 0.057 \text{ degrees} \approx 200 \text{ arcseconds}.
\]

At distances beyond 3 kilometers, the ball becomes difficult to resolve with the naked eye, and ambiguity in the moment of drop increases dramatically.

\section*{Telegraph Signal Propagation Delays}

Telegraph signals propagate through copper wire at approximately 95-99\% of the speed of light in vacuum. In copper, the propagation velocity is approximately

\[
  v \approx 0.95c \approx 2.85 \times 10^8 \text{ m/s}.
\]

For a telegraph line running from Greenwich to Liverpool (approximately 200 kilometers), the signal propagation time is

\[
  t_{\text{prop}} = \frac{200 \times 10^3}{2.85 \times 10^8} \approx 0.70 \text{ milliseconds}.
\]

Telegraph relays, however, introduce additional delays. An electromagnetic relay must:
1. Receive the current pulse (microseconds)
2. Close its switch contacts (milliseconds)
3. Transmit the signal to the next relay or to the receiving device

The total relay delay is typically 5-20 milliseconds, depending on the relay design and the electrical characteristics of the circuit. Thus, a time signal from Greenwich to Liverpool would arrive with a total delay of approximately 5-20 milliseconds, plus the 0.7 millisecond propagation delay—roughly 6-21 milliseconds total.

For precise timekeeping, this delay must be calibrated and accounted for. Operators would observe known time signals at both ends of a telegraph line and compute the round-trip delay, then subtract half for the one-way delay. Once this correction is known, subsequent signals can be corrected.

\section*{Radio Signal Propagation and Multipath Effects}

Longwave radio signals (60 kHz) propagate both as ground waves (following Earth's surface) and as sky waves (reflecting off the ionosphere). The ground wave is reliable but attenuates over distance. The sky wave is long-distance but subject to propagation delay variations.

The ground wave propagates at the speed of light in air and through the ground:

\[
  v_{\text{ground}} \approx c / n \approx 2.9 \times 10^8 \text{ m/s},
\]

where $n$ is the refractive index of the propagation medium (approximately 1.03 for the air-ground interface). For the Rugby transmitter to a receiver 100 kilometers away:

\[
  t_{\text{ground}} = \frac{100 \times 10^3}{2.9 \times 10^8} \approx 0.34 \text{ milliseconds}.
\]

The sky wave propagates to the ionosphere and back, traveling a much longer path. At night, when the ionosphere is active and reflects the signal, the sky wave typically travels approximately 200-400 kilometers more than the ground wave, introducing a delay of

\[
  t_{\text{sky}} \approx \frac{300 \times 10^3}{2.9 \times 10^8} \approx 1.0 \text{ millisecond}.
\]

When both ground and sky waves are present, the receiver sees the signal as a composite—the ground wave arriving first, the sky wave following 0.5-1 millisecond later. If these two signals are in phase, they reinforce. If they are out of phase, they partially cancel. The resulting received signal amplitude can vary significantly, and the time of arrival becomes ambiguous.

To mitigate this, modern longwave time signals use encoding schemes where the time information is spread across multiple pulses. A receiver can integrate across several seconds of signal to extract the time accurately, reducing the impact of multipath distortion.

\section*{Error Budget for Time Ball Observation}

A comprehensive error budget for a time ball observation (measured in seconds):

\begin{center}
\begin{tabular}{lrr}
\hline
\textbf{Error Source} & \textbf{Magnitude} & \textbf{Description} \\
\hline
Electromagnetic release & 0.005 & Switching transient \\
Mechanical drag on shaft & 0.003 & Ball acceleration delay \\
Impact detection & 0.010 & Catch mechanism response \\
Light travel (parallax) & 0.000 & Negligible for distances $<$ 5 km \\
Atmospheric refraction & 0.003 & Bending of light rays \\
Observer parallax & 0.005 & Eye position relative to crosshairs \\
Observer perception & 0.030 & Time to recognize motion \\
Reaction time (central) & 0.100 & Neural processing and motor response \\
Reaction time variation & 0.050 & Standard deviation of personal reaction \\
\hline
\textbf{Root-sum-square total} & 0.13 & (Assuming independence) \\
\hline
\end{tabular}
\end{center}

The dominant error source is human reaction time variability. The best-trained observers, practicing repeatedly to minimize personal reaction time variation, could achieve consistent timing to approximately 0.1 seconds using the time ball.

\section*{GPS Time Distribution and Accuracy}

The Global Positioning System provides time distribution with accuracy to approximately 100 nanoseconds (0.0000001 seconds) to the general public (code phase). Encrypted military signals can achieve nanosecond-level accuracy. A GPS receiver determines time by measuring the propagation delay of signals from multiple satellites. Each satellite transmits its location and the current GPS time. The receiver calculates:

\[
  c \cdot \Delta t = \sqrt{(x_{\text{sat}} - x_{\text{rx}})^2 + (y_{\text{sat}} - y_{\text{rx}})^2 + (z_{\text{sat}} - z_{\text{rx}})^2},
\]

for each satellite, where $\Delta t$ is the signal propagation time and $c$ is the speed of light. With signals from four satellites, the receiver can solve for its position $(x_{\text{rx}}, y_{\text{rx}}, z_{\text{rx}})$ and the clock bias (which gives time).

The limiting factor in GPS time accuracy is atmospheric delay—principally the ionospheric delay, which causes electromagnetic waves to propagate more slowly through the ionosphere. Differential GPS techniques, using ground stations with known locations, can measure and correct for this delay, improving accuracy to better than 10 nanoseconds.

\section*{Historical Evolution of Time Distribution Accuracy}

\begin{center}
\begin{tabular}{lll}
\hline
\textbf{Method} & \textbf{Era} & \textbf{Accuracy} \\
\hline
Direct solar observation & Pre-1800 & $\pm 1$ minute \\
Chronometer at port & 1800–1850 & $\pm 5$ seconds \\
Time ball (visual) & 1833–1900 & $\pm 0.1$ seconds \\
Telegraph time signals & 1860–1950 & $\pm 10$ ms \\
Telephone time-of-day & 1950–1980 & $\pm 1$ ms \\
Longwave radio (MSF) & 1960–present & $\pm 1$ ms \\
GPS (civilian) & 1995–present & $\pm 100$ ns \\
GPS (military/corrected) & 2000–present & $\pm 10$ ns \\
\hline
\end{tabular}
\end{center}

The improvement by a factor of $10^6$ over two centuries reflects both advances in technology and a fundamental shift in the nature of distributed timekeeping, from local observation to globally coordinated signals.

\section*{The Telegraph Network Topology}

A simplified model of the British telegraph time distribution network (circa 1880) consists of:

\begin{itemize}
\item \textbf{Central node}: Greenwich Observatory (source of Greenwich Mean Time)
\item \textbf{First-level nodes}: Major cities (London, Edinburgh, Liverpool, Manchester) connected directly to Greenwich by telegraph wires
\item \textbf{Second-level nodes}: Railway stations and ports connected to first-level nodes
\item \textbf{Third-level nodes}: Individual railway stations, accessible via relays from second-level nodes
\end{itemize}

Each telegraph wire carried at least one time signal per minute (often once per hour for efficiency). Operators at receiving stations would record the time signal and use it to check and adjust local master clocks. This hierarchical distribution ensured that time was available at multiple levels of precision, with more frequent signals at major hubs and less frequent signals at peripheral stations.

The redundancy was intentional: if a single telegraph line failed, operators could still receive time signals via alternate routes. The system was robust and highly reliable, operating continuously for nearly a century before being superseded by radio and GPS.


