% =====================================================================
% FOREWORD
% =====================================================================
%
% The author's introduction to the book, establishing:
%   - The contemporary relevance of Greenwich's legacy
%   - The book's central argument (precision as human achievement)
%   - The book's approach (mathematics as argument, integrated narrative)
%   - The intended audience and what the book asks of readers
%   - The open source nature of the project
%   - An invitation to begin with Chapter 1 (the Scilly disaster)
%
% Layout: Unnumbered chapter, added to table of contents
% Running heads: "Foreword" on both verso and recto pages
%
% =====================================================================

\chapter*{Foreword}
\addcontentsline{toc}{chapter}{Foreword}
\markboth{Foreword}{Foreword}

Every time you glance at your phone to check the time, you are consulting the Royal Observatory, Greenwich. The satellites overhead, the atomic clocks they carry, the coordinate systems that locate you on Earth's surface---all descend from work done on a hill in southeast London, beginning in 1675 when a small building rose above the Thames and a solitary astronomer began measuring the heavens.

This book tells the story of that measuring.

It is a story of precision: not precision as an abstraction, but as a practical achievement built through centuries of patient labor. The astronomers who worked at Greenwich did not discover precision waiting in the sky. They constructed it, observation by observation, instrument by instrument, correction by correction. They learned to account for the bending of starlight through atmosphere, the wobble of Earth's axis, the expansion of brass in summer heat. Each source of error, once understood, became a correction; each correction brought the stars into sharper focus.

I wrote this book because I wanted to understand how we came to live in a world measured to nanoseconds---and because I suspected the answer would be more interesting than a simple tale of progress. It is. The path from Flamsteed's first observations to modern astrometry winds through shipwrecks and parliamentary battles, through rivalries that lasted decades and collaborations that spanned oceans. Precision emerged not from genius alone but from institutions that preserved knowledge across generations, from instruments that embodied hard-won understanding, from individuals who devoted their lives to work they knew they would not complete.

The mathematics in this book is real. I make no apology for this. Where a derivation illuminates, I include it. Where an equation reveals structure that words cannot, I let the equation speak. If you have wondered how Bradley detected the aberration of starlight, you will find not only the story of his observations but the geometry that made them interpretable. If you have asked how navigators determined longitude before GPS, you will work through the spherical trigonometry they used, with actual numbers from actual voyages. Mathematics is not decoration here; it is argument.

But mathematics alone cannot capture what happened at Greenwich. The equations Bradley derived emerged from years of observations made in freezing darkness, his eye pressed to an eyepiece, his fingers numb on the micrometer screw. The error analyses that made stellar positions meaningful were conducted by human beings with careers to advance, rivals to best, and doubts to overcome. This book holds both: the cold clarity of the mathematics and the human warmth of the people who made it. I believe they illuminate each other.

I have written for readers who want to understand, not merely to be told. I assume you are intelligent and willing to work. I will define terms precisely and then use them without apology. I will show you derivations step by step, trusting that if you have followed one step you can follow the next. In return, I promise not to simplify at the cost of accuracy. The astronomers of Greenwich took their readers seriously; I mean to do the same.

This book is open source. The complete text, all figures, and the code that produces the final document are freely available under a Creative Commons license. You may share it, adapt it, build upon it. If you find an error---and in a work of this scope, errors are inevitable---you can propose a correction. This feels appropriate for a book about precision. Science advances through open exchange; the best measurements invite verification; knowledge shared is knowledge multiplied. The source lives at \texttt{github.com/oddurs/measure-of-the-world}, and I welcome your contributions.

The story begins with disaster. On the night of October 22, 1707, four ships of the Royal Navy struck the rocks off the Scilly Isles. Nearly two thousand men drowned, including Admiral Sir Cloudesley Shovell, because no one aboard could determine how far west they had sailed. The fleet believed itself safely in open water. It was not. From this catastrophe came urgency; from urgency, the Longitude Act; from the Longitude Act, a century of innovation that transformed navigation, timekeeping, and our understanding of Earth's place in the cosmos.

Turn the page, and we begin.

\vspace{2em}
\noindent\textit{Oddur Sigurdsson}\\
\noindent\textit{Brooklyn, January 2026}
