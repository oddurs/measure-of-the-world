% =====================================================================
% PREAMBLE: Complete typographic configuration and customization
% =====================================================================
% This document uses principles from "The Elements of Typographic Style"
% by Robert Bringhurst. All spacing, sizing, and layout choices are
% intentional and can be customized via the variables defined below.
%
% Paper: US Letter (8.5" × 11")
% Font: 12pt serif body text
% Line length: ~65 characters per line (optimal readability)
% Margins: Harmonious proportions using baseline grid
% =====================================================================

% =====================================================================
% SECTION 1: DOCUMENT CLASS AND BASIC FORMATTING
% =====================================================================

% Package: geometry - Page layout and margins
% Provides \geometry{} command to set page dimensions, margins, and spacing
\usepackage{geometry}
\geometry{%
  paper=letterpaper,        % US Letter: 8.5" × 11"
  inner=1.75in,              % Inner margin (gutter): 1.75 inches
  outer=1.5in,               % Outer margin: 1.5 inches
  top=1.5in,                 % Top margin: 1.5 inches
  bottom=1.75in,             % Bottom margin: 1.75 inches
  headsep=0.5in,             % Distance from header to body text
  footskip=0.5in%            % Distance from baseline to footer (reduced for less space above)
}

% =====================================================================
% SECTION 2: MICROTYPE AND CHARACTER-LEVEL TYPOGRAPHY
% =====================================================================

% Package: microtype - Font expansion, protrusion, tracking, kerning
% Enables subtle adjustments to letter spacing and alignment
% for improved typographic color and readability
\usepackage{microtype}
\microtypesetup{%
  expansion=true,            % Font expansion (±2% of design size)
  protrusion=true,           % Hanging punctuation and margin kerning
  tracking=true,             % Adjust letterspacing by context
  kerning=true,              % Kerning adjustments
  spacing=true%              % Spacing adjustments
}

% Tracking context: applies tracking adjustment to proof-reading text
% Value 70 = 0.07em per 1000pt, very subtle
\SetTracking[context=proof]{encoding=*}{70}

% Microtype context spacing: prevents excessive spacing after frenchspaced
% punctuation. Use 'nonfrench' for standard English spacing
\microtypecontext{spacing=nonfrench}

% =====================================================================
% SECTION 3: LINE SPACING AND PARAGRAPH GEOMETRY
% =====================================================================

% Line spread: controls vertical line spacing (leading)
% 1.25 multiplier on 12pt font = 15pt line spread (1.25 × 12 = 15)
% This is the primary control for document density and readability
\linespread{1.25}\selectfont

% Paragraph indentation: first-line indent (16pt ≈ 1.3em at 12pt)
% Controls visual paragraph breaks in continuous text
\setlength{\parindent}{16pt}

% Paragraph skip: spacing between paragraphs (set to 0 to use indents only)
% If you want paragraph breaks instead of indents, set this to 0.5\baselineskip
\setlength{\parskip}{0\baselineskip}

% =====================================================================
% SECTION 4: MATHEMATICS AND SCIENTIFIC NOTATION
% =====================================================================

% Package: amsmath - AMS mathematical typesetting
% Provides align*, equation*, gather* and other math environments
\usepackage{amsmath,amssymb,mathtools}

% --------- GLOBAL MATH SPACING CONTROL ---------
% Fixed spacing: all display math formulas use consistent 12pt spacing
% This ensures visual height consistency across all equations
% Spacing is LOCKED (no stretch/shrink) to prevent variable heights
\setlength{\abovedisplayskip}{12pt plus 0pt minus 0pt}
\setlength{\belowdisplayskip}{12pt plus 0pt minus 0pt}
\setlength{\abovedisplayshortskip}{12pt plus 0pt minus 0pt}
\setlength{\belowdisplayshortskip}{12pt plus 0pt minus 0pt}

% IMPORTANT: For consistent formula heights across the document:
% 1. Use \[ ... \] for display math (preferred - uses fixed spacing above)
% 2. All equations will have exactly 12pt above and below
% 3. Formulas with varying content heights will still align consistently
% 4. If a specific formula needs different spacing, wrap it:
%    \vspace{-6pt}\[...\]\vspace{-6pt}  to reduce spacing by 6pt above/below

% Configure math spacing around display formulas
% Use these commands to control display math spacing:
%   \displaystyle     - Full-size math (default in display mode)
%   \textstyle        - Inline-size math (smaller, more compact)
%   \scriptstyle      - Even smaller (for sub/superscripts)
%   \scriptscriptstyle - Smallest (for nested scripts)

% Configure math spacing around display formulas
% Use these commands to control display math spacing:
%   \displaystyle     - Full-size math (default in display mode)
%   \textstyle        - Inline-size math (smaller, more compact)
%   \scriptstyle      - Even smaller (for sub/superscripts)
%   \scriptscriptstyle - Smallest (for nested scripts)

% Inline math spacing (control whitespace around $ ... $ math)
% \thinmuskip = space around binary operators in inline math (default 3mu)
% \medmuskip = space around relation operators (default 4mu)
% \thickmuskip = space around opening/closing delimiters (default 5mu)
\thinmuskip=3mu plus 1mu minus 1mu       % Fine control for inline spacing
\medmuskip=4mu plus 2mu minus 2mu        % Medium spacing for relations
\thickmuskip=5mu plus 5mu minus 5mu      % Thick spacing for delimiters

% Package: gensymb - Generic symbols (degree, micro, ohm, etc.)
% Provides \degree, \micro, \ohm, \celsius symbols
\usepackage{gensymb}

% Package: siunitx - SI units and quantities
% Provides \qty{}{}, \si{}, \num{} commands for proper scientific notation
% detect-all: auto-detects font settings and applies them to SI output
\usepackage{siunitx}
\sisetup{detect-all=true}

% =====================================================================
% SECTION 5: FIGURES, GRAPHICS, AND PLOTS
% =====================================================================

% Package: graphicx - Include and manipulate images
% Provides \includegraphics{} command for inserting figures
\usepackage{graphicx}
\graphicspath{{figures/}{figures/jpg/}{figures/png/}{figures/pdf/}}

% Package: tikz - Vector graphics drawing language
% Package: pgfplots - Data plotting library built on tikz
% Used for creating diagrams, plots, and mathematical visualizations
\usepackage{tikz,pgfplots}
\pgfplotsset{compat=1.18}

% =====================================================================
% SECTION 6: BIBLIOGRAPHY AND CITATIONS
% =====================================================================

% Package: biblatex - Modern bibliography management (requires biber backend)
% style=authoryear: Uses (Author Year) citation style
% Can change to 'numeric', 'alphabetic', 'authortitle', etc.
\usepackage[backend=biber,style=authoryear]{biblatex}

% =====================================================================
% SECTION 7: HYPERLINKS, REFERENCES, AND CROSS-REFERENCES
% =====================================================================

% Package: hyperref - Hyperlinks and PDF metadata
% colorlinks=true: Colored links instead of boxes
% linkcolor=black: Cross-references and internal links in black
% citecolor=black: Bibliography citations in black
% urlcolor=black: URLs in black (change to blue for web visibility)
\usepackage{hyperref}
\hypersetup{%
  colorlinks=true,
  linkcolor=black,
  citecolor=black,
  urlcolor=black,
  pdftitle={Measure of the World},
  pdfauthor={Oddur Snorrason}%
}

% Package: cleveref - Intelligent cross-referencing
% Provides \cref{} and \Cref{} for automatic reference labeling
% \cref{ch:intro} → "chapter 1" | \Cref{sec:method} → "Section 2"
\usepackage{cleveref}

% =====================================================================
% SECTION 8: TABLES AND LISTS
% =====================================================================

% Package: booktabs - Professional table typesetting
% Provides \toprule, \midrule, \bottomrule for high-quality tables
% Use instead of default table rules for much better appearance

% Package: longtable - Tables that span multiple pages
% Provides \begin{longtable} environment with headers/footers per page

% Package: enumitem - Fine-grained list control
% Provides [noitemsep], [nolistsep] and spacing customization

% Package: caption - Enhanced caption formatting
% Provides \captionsetup for customizing figure and table captions
\usepackage{booktabs,longtable,enumitem,caption}

% =====================================================================
% SECTION 9: GLOSSARIES AND ACRONYMS
% =====================================================================

% Package: glossaries - Create glossaries and lists of acronyms
% Requires \makeglossaries call in document and glossary term definitions
\usepackage{glossaries}

% =====================================================================
% SECTION 10: EPIGRAPHS AND DECORATIVE ELEMENTS
% =====================================================================

% Package: epigraph - Format quotations and attributions
% Used for chapter-opening quotes or thematic epigraphs
\usepackage{epigraph}
\setlength{\epigraphwidth}{0.6\textwidth}        % Width of epigraph block
\setlength{\epigraphrule}{0pt}                   % Rule below epigraph (0pt = none)
\setlength{\beforeepigraphskip}{1.5\baselineskip} % Space before epigraph
\setlength{\afterepigraphskip}{1.5\baselineskip}  % Space after epigraph

% =====================================================================
% SECTION 11: MEMOIR CLASS CONFIGURATION
% =====================================================================

% Memoir-specific numbering depth and table of contents depth
% setsecnumdepth: how deep to number sections (subsection = include \subsection{})
% settocdepth: how deep to list in table of contents
\setsecnumdepth{subsection}
\settocdepth{subsection}

% Configure chapter opening pages to use the 'chapter' page style
% instead of memoir's default 'plain' style (which centers page numbers)
\aliaspagestyle{chapter}{chapter}

% Disable memoir's built-in paragraph break decorations
% By default, memoir adds a small ornamental rule between paragraphs
% This command disables it for a cleaner appearance
\renewcommand{\pfbreakdisplay}{}

% =====================================================================
% SECTION 12: HEADING HIERARCHY AND FONTS
% =====================================================================
% Typographic scale: 1.2× (12pt → 14.4pt → 17.28pt)
% This creates visual hierarchy through size while maintaining harmony
%
% Hierarchy:
%   Chapter:        18pt (small caps, dominant)
%   Section:        14pt (small caps, primary)
%   Subsection:     12pt (italic, secondary)
%   Subsubsection:  12pt (italic, tertiary)
%   Body text:      12pt (roman)

% CHAPTER HEADING
% \chapnamefont controls "Chapter X" label (14pt italic dark gray, like H2)
% \chapnumfont controls the chapter number (14pt italic dark gray)
% \chaptitlefont controls "Chapter Title" text (small caps, dominant size, black)
\renewcommand{\chapnamefont}{\normalfont\itshape\fontsize{14}{16.8}\selectfont\color[HTML]{333333}}
\renewcommand{\chapnumfont}{\normalfont\itshape\fontsize{14}{16.8}\selectfont\color[HTML]{333333}}
\renewcommand{\chaptitlefont}{\normalfont\scshape\fontsize{18}{21.6}\selectfont\color{black}}

% Chapter spacing:
% \beforechapskip = space before chapter heading (relative to text)
% \afterchapskip = space after chapter heading (before body text)
% \midchapskip = space between "Chapter X" and title (affects alignment)
\setlength{\beforechapskip}{4\baselineskip}      % Plenty of space before
\setlength{\afterchapskip}{1.5\baselineskip}     % Normal space after
\setlength{\midchapskip}{6pt}                     % Tight spacing between label and title

% SECTION HEADING
% 14pt small caps - primary hierarchical level below chapter
% \setsecheadstyle sets font and sizing for \section{}
\setsecheadstyle{\normalfont\scshape\fontsize{14}{16.8}\selectfont}

% Section spacing:
% \setbeforesecskip = space before section heading (can be negative)
% \setaftersecskip = space after section heading (before body text)
\setbeforesecskip{1.25\baselineskip}
\setaftersecskip{0.75\baselineskip}

% SUBSECTION HEADING
% 12pt italic - secondary hierarchical level (same size as body)
% Italic creates visual distinction without increasing size
\setsubsecheadstyle{\normalfont\itshape\fontsize{12}{14.4}\selectfont}
\setbeforesubsecskip{1.1\baselineskip}
\setaftersubsecskip{0.6\baselineskip}

% SUBSUBSECTION HEADING
% 12pt italic - tertiary level, minimal visual emphasis
\setsubsubsecheadstyle{\normalfont\itshape\fontsize{12}{14.4}\selectfont}

% =====================================================================
% SECTION 13: PAGE STYLES, HEADERS, AND FOOTERS
% =====================================================================
% Memoir's page style system provides fine-grained control over
% headers, footers, and page numbers. Two styles are defined:
% - mainmatter: Used for chapters (no headers, page numbers in outer corners)
% - frontmatter: Used for title/dedication/TOC (centered page numbers)

% MAIN MATTER PAGE STYLE (chapters)
% Page numbers appear only in bottom outer corners:
%   Recto (right, odd) pages: bottom right
%   Verso (left, even) pages: bottom left
\makepagestyle{mainmatter}

% Header configuration: no header text, no rule line
\makeheadrule{mainmatter}{\textwidth}{0pt}  % Rule thickness = 0pt (no rule)
\makeoddhead{mainmatter}{}{}{}              % Right page: empty header
\makeevenhead{mainmatter}{}{}{}             % Left page: empty header

% Footer configuration: page numbers in outer corners
% \makeoddfoot{style}{left}{center}{right}  → Recto (right) page
% \makeevenfoot{style}{left}{center}{right} → Verso (left) page
\makeoddfoot{mainmatter}{}{}{\footerpagenumber{\thepage}}   % Right corner for recto
\makeevenfoot{mainmatter}{\footerpagenumber{\thepage}}{}{}  % Left corner for verso

% Chapter mark configuration (kept for reference, not currently displayed)
% \createmark{division}{position}{number format}{before}{after}
% These define what content appears in running heads (if enabled)
\createmark{chapter}{left}{shownumber}{}{. \ }
\createmark{section}{right}{shownumber}{}{. \ }

% Set mainmatter as the default page style for the entire document
\pagestyle{mainmatter}

% FRONT MATTER PAGE STYLE (title page, dedication, TOC)
% Centered page numbers at bottom of page
\makepagestyle{frontmatter}
\makeheadrule{frontmatter}{\textwidth}{0pt}  % No rule
\makeoddhead{frontmatter}{}{}{}              % No header (right page)
\makeevenhead{frontmatter}{}{}{}             % No header (left page)
\makeoddfoot{frontmatter}{}{\footerpagenumber{\thepage}}{} % Centered (recto)
\makeevenfoot{frontmatter}{}{\footerpagenumber{\thepage}}{}% Centered (verso)

% CHAPTER OPENING PAGE STYLE
% Chapter opening pages should have page numbers in the outer corner,
% not centered. This overrides memoir's default centered behavior.
\makepagestyle{chapter}
\makeheadrule{chapter}{\textwidth}{0pt}     % No rule
\makeoddhead{chapter}{}{}{}                 % No header (right page)
\makeevenhead{chapter}{}{}{}                % No header (left page)
\makeoddfoot{chapter}{}{}{\footerpagenumber{\thepage}}   % Right corner (recto)
\makeevenfoot{chapter}{\footerpagenumber{\thepage}}{}{}  % Left corner (verso)

% =====================================================================
% SECTION 14: LIST FORMATTING
% =====================================================================
% Fine-grained control over itemize and enumerate environments
% Spacing parameters:
%   topsep:  space before list starts
%   itemsep: space between list items
%   parsep:  space between paragraphs within a list item
%   leftmargin: indentation from left margin

\setlist[itemize]{%
  topsep=0.75\baselineskip,
  itemsep=0.5\baselineskip,
  parsep=0pt,
  leftmargin=2em%
}

\setlist[enumerate]{%
  topsep=0.75\baselineskip,
  itemsep=0.5\baselineskip,
  parsep=0pt,
  leftmargin=2em%
}

% =====================================================================
% SECTION 15: TABLE TYPOGRAPHY AND STYLING
% =====================================================================
% Tables require careful spacing and alignment for readability.
% All table dimensions are controlled by the variables defined below,
% allowing global adjustment of table appearance without editing
% individual tables.

% --------- TABLE ROW HEIGHT ---------
% Controls vertical spacing within table cells
% Use: \renewcommand{\arraystretch}{<factor>} in table
% 1.0 = normal, 1.3 = 30% taller, 1.5 = 50% taller
\renewcommand{\arraystretch}{1.3}

% --------- TABLE RULE WIDTHS ---------
% booktabs provides three rule types: \toprule, \midrule, \bottomrule
% These are thinner and more elegant than standard \hline

% Heavy rule thickness (top and bottom of table)
\setlength{\heavyrulewidth}{0.08em}

% Light rule thickness (horizontal rules within table, \midrule)
\setlength{\lightrulewidth}{0.05em}

% Cmidrule thickness (partial rules, \cmidrule)
\setlength{\cmidrulewidth}{0.05em}

% --------- TABLE TEXT SIZING ---------
% Body text size for table content (smaller than body text for compactness)
\newcommand{\tabletextsize}{\small}

% --------- TABLE SPACING ---------
% Before/after table spacing (relative to baseline)
\newcommand{\tablebeforeskip}{0\baselineskip}
\newcommand{\tableafterskip}{0\baselineskip}

% Space between table caption and table itself
\newcommand{\tablecaptionspacing}{0.5\baselineskip}

% Longtable spacing (tables that span multiple pages)
\setlength{\LTpre}{\tablebeforeskip}
\setlength{\LTpost}{\tableafterskip}

% --------- TABLE ENVIRONMENT MACRO ---------
% \begin{styledtable}[placement] ... \end{styledtable}
% Automatically applies all table typography variables
% Default placement: [htbp] (here, top, bottom, page)
% Usage: \begin{styledtable}[h] \caption{...} \begin{tabular}...
\newenvironment{styledtable}[1][htbp]{%
  \setlength{\parskip}{\tablecaptionspacing}%
  \begin{table}[#1]%
  \centering%
  \tabletextsize%
  \renewcommand{\arraystretch}{1.3}%
}{%
  \end{table}%
}

% --------- TABLE CAPTION STYLING ---------
% Captions are set in small font (like footnotes) for visual hierarchy
% Font is slightly smaller than body text to visually subordinate the caption
\captionsetup{font=small,labelfont=small,justification=centering}

% =====================================================================
% SECTION 16: HEADER AND FOOTER TYPOGRAPHY
% =====================================================================
% Controls the appearance of headers, footers, and running heads

% --------- HEADER TEXT STYLING ---------
% 11pt italic for headers (slightly smaller than body text)
% Used with \versoheader{} and \rectoheader{} commands
\newcommand{\headerfont}{\normalfont\itshape\fontsize{11}{13.2}\selectfont}

% --------- FOOTER TEXT STYLING ---------
% 10pt for footer content (minimal information, needs small size)
\newcommand{\footerfont}{\normalfont\fontsize{10}{12}\selectfont}

% --------- PAGE NUMBER STYLING ---------
% 11pt for page numbers (same size as headers for consistency)
\newcommand{\pagenumberfont}{\normalfont\fontsize{11}{13.2}\selectfont}
\newcommand{\pagenumberformat}{\pagenumberfont}

% --------- MARGINS AND SPACING ---------
% Control indentation and positioning of header/footer elements
\newcommand{\headerindent}{0.25in}          % Indent for header content
\newcommand{\headermargin}{0.1875in}        % Margin for header rules
\newcommand{\footerindent}{0.25in}          % Indent for footer content
\newcommand{\footermargin}{0.1875in}        % Margin for footer rules

% --------- SEPARATOR RULES ---------
% Control thickness of rules above headers and below footers
\newcommand{\headerrulethickness}{0.5pt}    % Rule above running head
\newcommand{\footerrulethickness}{0.5pt}    % Rule below running head

% --------- RUNNING HEAD SPACING ---------
% Vertical skip between page margin and running head text
\newcommand{\runningheadskip}{0.5\baselineskip}

% --------- VERSO/RECTO HEADER MACROS ---------
% Use these to apply consistent styling to left and right page headers
\newcommand{\versoheader}[1]{\headerfont #1}    % Verso (even) pages
\newcommand{\rectoheader}[1]{\headerfont #1}    % Recto (odd) pages

% --------- PAGE NUMBER FORMATTING MACRO ---------
% Centralized control over page number appearance
\newcommand{\footerpagenumber}[1]{\pagenumberformat #1}

% =====================================================================
% SECTION 17: MARGIN NOTES AND SIDENOTES
% =====================================================================
% Configuration for margin annotations and sidenotes
% Useful for supplementary information without disrupting flow

% Smaller font size for margin notes (9pt)
\newcommand{\marginnotesize}{\fontsize{9}{11}\selectfont}

% Maximum width of margin note text (1.25 inches fits in outer margin)
\newcommand{\marginnotewidth}{1.25in}

% Horizontal gap between body text and margin note
\newcommand{\marginnotesep}{0.25in}

% =====================================================================
% SECTION 18: FOOTNOTE STYLING
% =====================================================================
% Footnotes provide expanded commentary and context without disrupting narrative
% All spacing, colors, and dimensions are configurable below

% --------- FOOTNOTE RULE STYLING ---------
% Horizontal divider between body text and footnotes
\newcommand{\footnoterulecolor}{333333}          % Rule color (hex, no # prefix)
\newcommand{\footnoterulewidth}{2in}             % Width of divider rule
\newcommand{\footnoterulethickness}{0.5pt}       % Thickness of divider line
\newcommand{\footnoterulekernabove}{6pt}         % Space above divider
\newcommand{\footnoterulekernbelow}{6pt}         % Space below divider

% Apply configurable footnote rule with spacing
% Color is scoped to the rule only (doesn't affect following text)
\renewcommand{\footnoterule}{%
  \kern \footnoterulekernabove%
  {\color[HTML]{\footnoterulecolor}\hrule width \footnoterulewidth height \footnoterulethickness}%
  \kern \footnoterulekernbelow}

% --------- FOOTNOTE SPACING ---------
% Space between footnote area and body text
\newcommand{\skippagefootins}{1\baselineskip}   % Gap before first footnote
\setlength{\skip\footins}{\skippagefootins}

% Space between individual footnote items
\newcommand{\footnoteseparation}{0.5\baselineskip}  % Gap between footnotes
\setlength{\footnotesep}{\footnoteseparation}

% --------- FOOTNOTE FONT SIZING ---------
% Font size for footnote text (smaller than body, but readable)
\renewcommand{\footnotesize}{\fontsize{10}{12}\selectfont}

% =====================================================================
% SECTION 19: FOOTER SPACING
% =====================================================================
% Control space above footer to create breathing room

% --------- FOOTER GAP CONFIGURATION ---------
% Distance from body text baseline to footer line (higher = more space)
% Currently set in \geometry{} command (Section 1)
% Default: 0.5in (increased from 0.75in for compact spacing)
\newcommand{\customfootskip}{0.5in}

% Note: To adjust, modify the geometry footskip parameter in Section 1
% OR change \customfootskip above and apply in geometry settings
% =====================================================================
% All document-wide settings are now complete. The document class,
% typography, spacing, colors, and styling are now ready for use.
% Modify the variables above to globally adjust appearance.
%
% Key customization points:
% - Line spacing:         \linespread value (line 43)
% - Margins:              \geometry settings (line 20-25)
% - Heading sizes:        Font sizes in Section 12
% - Table spacing:        Section 15 variables
% - Math spacing:         Section 4 (abovedisplayskip, belowdisplayskip)
% - Colors:               Section 7 (\hypersetup{})
% =====================================================================
