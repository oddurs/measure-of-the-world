\chapter{Telescope Optics and Mountings}
\label{ch:telescope-optics-mountings}

In 1668, Isaac Newton stood before the Royal Society in London with an instrument barely six inches long. The reflecting telescope he presented was a marvel of miniaturization and optical cleverness. Its single mirror, an inch in diameter, fashioned from speculum metal—a bronze alloy of copper and tin—focused light with remarkable clarity. The mirror was curved into a parabolic shape, its surface finished to a precision that few had imagined possible. Newton had solved a fundamental problem of refraction that had plagued opticians for decades: the rainbow fringing that plagued traditional refracting telescopes. Here was a telescope that saw the world without chromatic aberration, with power equivalent to a refractor ten times its length.\footnote{\textcite{Newton1704} describes his reflecting telescope in the \textit{Opticks}. \textcite{Chapman1990} provides detailed historical analysis of Newton's optical work and its impact on telescope design.} Yet Newton's triumph was fleeting. Within months, the speculum metal mirror tarnished, its reflective coating degrading in the damp London air. The reflector's superiority was theoretical; its practical dominance would have to wait two centuries, until better mirror technologies emerged. This chapter traces both the physics that made telescopes work and the engineering compromises that constrained their use.

\section{Refraction and Chromatic Aberration}

The refractor—a telescope using a glass lens—was the first practical telescope design, appearing in the early 17th century. Its principle is elegant: a large front lens (the objective) collects light, forming an image at its focal point. An eyepiece lens magnifies this image. The optical path is straightforward; the engineering challenge is precision.

But glass is not neutral to light. The refractive index depends on wavelength. Blue light bends more than red light. A single glass lens, therefore, focuses different colors at different distances—an effect called \textbf{chromatic aberration}. The mathematics is straightforward. The refractive index $n(\lambda)$ of glass depends on wavelength according to an empirical relation (the Cauchy equation):

\[
  n(\lambda) = A + \frac{B}{\lambda^2} + \frac{C}{\lambda^4} + \cdots
\]

where $A$, $B$, and $C$ are material constants, and $\lambda$ is wavelength. For a single converging lens of focal length $f_0$ at wavelength $\lambda_0$, the focal length at another wavelength $\lambda$ is:

\[
  f(\lambda) = f_0 \left[ 1 + \frac{B(n(\lambda_0) - 1)}{A}\left(\frac{1}{\lambda_0^2} - \frac{1}{\lambda^2}\right) \right]^{-1}.
\]

For visible light, the spread between red and blue focal points is roughly $\Delta f \approx f_0 / 100$ for a typical glass lens. For a telescope with a 1-meter focal length, the red and blue images are separated by a centimeter—a devastating blur.\footnote{\textcite{Born1999} provides rigorous treatment of chromatic aberration and dispersion in optical systems. \textcite{Smith2006} traces the history of color correction in telescope design.}

To compensate, early telescope makers built enormously long instruments—refractors 40 feet or more in focal length, where the separation, while absolute, was small enough relative to the eyepiece's field of view to be tolerable. These unwieldy tubes, supported by rickety scaffolding, were the curse of 17th-century astronomy.

\section{The Achromatic Doublet}

The solution was the achromatic lens, a combination of two glass elements of different types that cancel each other's chromatic aberration at two wavelengths. The idea appeared first in patents by Chester Moore Hall in 1730, and was independently developed and commercialized by John Dollond in the 1750s.\footnote{\textcite{Dollond1758} describes his achromatic designs. \textcite{Bennett1999} examines the history of the achromat's development and its impact on telescope construction.}

The principle: combine a converging lens of crown glass (low dispersion, weak color spread) with a diverging lens of flint glass (high dispersion, strong color spread). If the two lenses are chosen correctly, the red and blue focal points can be made coincident, and intermediate wavelengths fall in between.

Consider a crown glass lens of power $\Phi_c = 1/f_c$ and refractive index $n_c$, placed in contact with a flint glass lens of power $\Phi_f = 1/f_f$ and refractive index $n_f$. The combined focal length is:

\[
  \frac{1}{f} = \frac{1}{f_c} + \frac{1}{f_f} = \Phi_c + \Phi_f.
\]

To achieve achromatism—equal focal lengths for red and blue light—we require that the combined system's focal length at red ($\lambda_{\text{red}}$) equals its focal length at blue ($\lambda_{\text{blue}}$):

\[
  \frac{1}{f_{\text{red}}} = \frac{1}{f_{\text{blue}}}.
\]

Expanding, this becomes:

\[
  \Phi_c(\lambda_{\text{red}}) + \Phi_f(\lambda_{\text{red}}) = \Phi_c(\lambda_{\text{blue}}) + \Phi_f(\lambda_{\text{blue}}).
\]

Using the Cauchy dispersion relation and rearranging:

\[
  \frac{\Phi_f}{\Phi_c} = - \frac{n_c(\lambda_{\text{red}}) - n_c(\lambda_{\text{blue}})}{n_f(\lambda_{\text{red}}) - n_f(\lambda_{\text{blue}})}.
\]

Since crown glass has lower dispersion than flint glass, the magnitudes of the denominators ensure that $\Phi_f < 0$ (the flint lens is diverging), and its power is weaker than the crown lens's. The result is a net converging system.\footnote{\textcite{Kingslake1978} provides detailed design theory for achromatic objectives. The mathematical condition for achromatism is discussed thoroughly in \textcite{Hecht2002}.}

With an achromatic doublet, telescopes could be shortened dramatically. A 1-meter focal length refractor became practical where previously a 40-foot tube was required. By the 19th century, the achromatic refractor dominated observatory work, and giant refractors—the Great Refractor at Greenwich (28 inches), the Yerkes refractor (40 inches)—became prestige instruments.

\section{Reflectors: Newton and Beyond}

Newton's reflecting design avoided chromatic aberration entirely: mirrors reflect all wavelengths identically. A parabolic mirror, unlike a lens, needs no color correction. The challenge was not optics but metallurgy and mechanics.

Newton's reflector used speculum metal, a brittle bronze alloy. Its reflectivity at visible wavelengths exceeds 60%, but it tarnishes rapidly. The optical surfaces require grinding and polishing to extraordinary precision—departures from a perfect parabola of more than a few wavelengths of light degrade image quality. And the metal must be supported against the flexure induced by its own weight, particularly when the telescope moves to observe different parts of the sky.

For more than a century, reflectors remained curiosities, easier to make in theory than in practice. But in the late 18th century, William Herschel began constructing large reflectors, using improved speculum metal alloys and innovative mounting designs. Herschel's 20-foot reflector (40-inch aperture) and later his 40-foot (48-inch aperture) became the largest telescopes in the world, far exceeding any refractor.

The breakthrough for reflectors came in 1857, when Léon Foucault developed the silver-on-glass mirror: a thin layer of silver deposited on the back of a glass substrate.\footnote{\textcite{Foucault1857} describes the silver-on-glass mirror technique and its advantages. \textcite{Wilson1996} provides comprehensive modern treatment of reflector optics.} Glass is easy to shape and polish; the silver layer is replaced occasionally; and the reflectivity exceeds 90% initially and remains above 80% even after tarnishing. Within decades, silver-on-glass reflectors dominated astronomy, from small amateur instruments to the great observatories.

\section{Aberrations Beyond Chromatic}

Even a perfectly achromatic lens or a perfect parabolic mirror introduces distortions to an image. These \textbf{optical aberrations} arise from the geometry of focusing light through a finite aperture.

\textbf{Spherical Aberration:} A spherical surface naturally focuses different zones of an incoming plane wave at slightly different distances. Light rays near the edge of the lens focus closer to the lens than rays near the optical axis. For a parabolic mirror, spherical aberration is eliminated by design (the parabolic surface is specifically shaped to converge all rays to a single focus). For a lens, spherical aberration can be reduced but not eliminated by a single element; it must be corrected through careful combination of positive and negative lenses.

\textbf{Coma:} When observing off-axis (away from the optical axis), even a perfect parabolic mirror or achromatic lens introduces a comet-shaped image distortion. The off-axis point sources blur into a trailing asymmetry. Coma increases with aperture size and field of view.

\textbf{Astigmatism:} Off-axis points are focused differently in the meridional (in-plane) and sagittal (perpendicular) directions. Astigmatism becomes severe for wide fields of view and large apertures.

\textbf{Field Curvature:} Even if on-axis aberrations are corrected, the focal plane may be curved rather than flat. A flat detector (photographic plate or CCD) cannot be in focus everywhere across the field of view simultaneously.

\textbf{Distortion:} The magnification varies with angle from the optical axis, causing straight lines at the edge of the field to appear curved. Distortion is rarely a limiting factor for astronomical instruments but can matter for wide-field surveys.

These aberrations limit the useful field of view and resolution. Modern telescope designs employ complex combinations of elements—three, four, or more lenses or mirror groups—to minimize these errors across a wide field. But there is always a trade-off: wide field of view versus aperture size, cost, and mechanical complexity.

\section{Mountings: Altazimuth and Equatorial}

A telescope points at an object in the sky. As Earth rotates, the object's position changes continuously. The observer must track, adjusting the telescope's pointing to follow the star across the sky. Two mounting systems evolved: the \textbf{altazimuth} mount and the \textbf{equatorial} mount.

The altazimuth mount rotates about two perpendicular axes: altitude (up-down) and azimuth (left-right). From any location on Earth's surface, altitude and azimuth specify a unique direction in the sky. Altazimuth mounts are mechanically simple and can be compact. But to track a star, the mount must continuously adjust both altitude and azimuth as the star moves.

The equatorial mount is oriented so that one axis is parallel to Earth's rotation axis (pointing toward the celestial pole). Rotating about this axis once per sidereal day (23 hours 56 minutes) keeps the telescope pointed at the same star indefinitely. The second axis, perpendicular to the polar axis, is used to set the initial declination. For steady tracking, only the polar axis needs to rotate—a constant, uniform motion easily driven by a mechanical clock.

The advantage of the equatorial mount is profound: a single constant rotation rate produces perfect tracking. For this reason, equatorial mounts dominated observatory design for centuries. The largest refracting telescopes—those requiring precision tracking for extended observations—were mounted equatorially.

An altazimuth mount, by contrast, must continuously adjust both axes in a complex, non-uniform pattern to track. However, altazimuth mounts have two compensating advantages: they can be shorter and more compact than equatorial mounts of comparable aperture, and they place heavy instruments (primary mirrors or objectives) lower and closer to vertical, reducing the mechanical stress on the structure.

\section{Field Rotation and the Equatorial Advantage}

There is a subtle but important effect that long favored equatorial mounts: field rotation. When an astronomical object is observed in an altazimuth mount, the orientation of the field of view rotates as the object moves across the sky. If a star is being tracked for a long-exposure photograph, the image blur caused by imperfect tracking, combined with the rotating field, can smear a point source into an arc.\footnote{\textcite{Malin1979} discusses field rotation in altazimuth telescopes and its effect on long-exposure imaging.}

The field rotation angle depends on the altitude $h$ of the observed object and the observer's latitude $\phi$. For an object at altitude $h$ transiting due south, the field rotation rate is:

\[
  \frac{d\theta_{\text{rot}}}{dt} = \sin(\phi) \tan(h) \frac{d\text{Az}}{dt},
\]

where Az is the azimuth. This rotation rate becomes large for objects near the horizon or at high altitudes when moving rapidly in azimuth. For deep, long-exposure observations—critical for discovering faint objects—field rotation limited the practical exposure duration with an altazimuth mount.

An equatorial mount has no field rotation: as the telescope rotates about the polar axis, the entire field of view rotates, but the celestial objects within the field maintain fixed positions relative to the image detector. This lack of field rotation was a powerful reason observatories adopted equatorial mounts for their largest, most sensitive instruments.

In recent decades, computer control and adaptive optics have made altazimuth mounts practical even for precision work. Modern large telescopes—the Very Large Telescope, the Keck Observatory—use altazimuth mounts with computerized tracking to correct for field rotation and other effects. But for most of the 19th and 20th centuries, equatorial mounts were essential for serious astronomical work.

\section{Clock Drives}

An equatorial mount's advantage is only realized with a reliable clock drive. Without it, the observer must manually track, a task that becomes exhausting after hours of precise work. A clock drive mechanically connects a constant-speed rotator (traditionally a pendulum clock) to the telescope's polar axis.

The tracking rate must be precise. Earth rotates once every sidereal day, which is 86164.0905 seconds (23 hours 56 minutes 4.0905 seconds), not 86400 seconds as in a solar day. The sidereal rotation rate is:

\[
  \omega_{\text{sidereal}} = \frac{2\pi \text{ radians}}{86164.0905 \text{ seconds}} = 7.2921 \times 10^{-5} \text{ rad/s}.
\]

A clock drive gears the telescope's polar axis to rotate at exactly this rate. Any deviation causes the tracked object to drift across the field of view. For observational work requiring precise positioning, the clock drive must be accurate to better than a few arcseconds per hour—a tolerance demanding high-quality clock mechanics.

Traditional observatory clocks were temperature-compensated pendulum systems, as described in Chapter~\ref{ch:harrison-chronometers}, refined to extraordinary precision. By the 20th century, some observatories used multiple synchronized clocks or quartz oscillators to drive their largest telescopes.

\section{Aperture and Performance: The 19th-Century Race}

As optical engineering improved and aberrations were better controlled, observatories competed to build larger telescopes. Aperture determines the light-gathering power (proportional to area) and the angular resolution (limited by diffraction, inversely proportional to aperture for a given wavelength). Larger is better, up to limits set by atmospheric seeing, optical quality, and mechanical stability.

Table~\ref{tab:telescope-apertures} summarizes the major telescopes constructed from 1668 to 1900, showing the steady increase in aperture and the gradual shift from refractors to reflectors as mirror technology improved.

\begin{table}[ht]
\centering
\begin{tabularx}{\textwidth}{>{\raggedright\arraybackslash}X c >{\raggedright\arraybackslash}X >{\raggedright\arraybackslash}X}
\toprule
\textbf{Telescope} & \textbf{Year} & \textbf{Type \& Aperture} & \textbf{Notable Features} \\
\midrule
Newton reflecting & 1668 & Reflector, 1 inch & First practical reflector; speculum metal \\
Huygens refractor & 1680 & Refractor, 2.2 inches & Long focal length; aerial telescope \\
Dollond refractor & 1758 & Refractor, 2.5 inches & First achromatic lens; commercialized \\
Herschel reflector & 1785 & Reflector, 40 inches & Largest telescope of its era; speculum metal \\
Fraunhofer refractor & 1820 & Refractor, 9.6 inches & Improved achromat; Munich Observatory \\
Rosse reflector & 1845 & Reflector, 72 inches & Leviathan of Parsonstown; largest for 30 years \\
Greenwich refractor & 1859 & Refractor, 28 inches & Great Refractor; Greenwich Observatory \\
Yerkes refractor & 1897 & Refractor, 40 inches & Largest refractor ever built \\
Lick refractor & 1888 & Refractor, 36 inches & Lick Observatory; San Francisco Bay \\
Mount Wilson reflector & 1908 & Reflector, 60 inches & Modern silver-on-glass mirror; equatorial \\
\bottomrule
\end{tabularx}
\caption{Major Telescopes, 1668–1900. Aperture growth and the transition from refractors to reflectors reflect improving optical theory and mirror technology.}
\label{tab:telescope-apertures}
\end{table}

The race to build larger telescopes drove innovation in precision manufacturing, optical theory, and mechanical engineering. Yet the success of larger apertures revealed fundamental limits: atmospheric turbulence (seeing), which scrambles light as it passes through Earth's atmosphere, ultimately limits resolution for ground-based telescopes to roughly an arcsecond—independent of aperture for telescopes larger than about a meter. This realization, emerging in the early 20th century, shifted focus from aperture alone to complementary technologies: adaptive optics, interferometry, and eventually space-based observation.

\section{The Limits of Optical Design}

By 1900, the full range of optical aberrations was understood, and design techniques existed to minimize them. Yet fundamental physical limits remained. No lens could eliminate all aberrations; no mirror could be made infinitely rigid against gravity and temperature change; no mount could track with perfect precision.

Moreover, atmospheric seeing remained an insurmountable obstacle for ground-based telescopes. Starlight, passing through turbulent air, is randomly refracted and scattered. Even a perfect telescope, pointed at a star on a calm night, produces a light distribution spread over many arcseconds due to atmospheric blurring. Only a fraction of the theoretical diffraction-limited resolution is ever achieved in practice.

These limits drove the next great innovations: the development of interferometry (combining light from multiple telescopes to achieve resolution as if the telescopes were much larger), the invention of adaptive optics (using deformable mirrors to correct for atmospheric distortion), and ultimately the construction of space-based telescopes (above the atmosphere). But these advances lay in the 20th century and beyond the scope of this history. By 1900, the optical and mechanical principles that would guide telescope design for a century had been established, refined, and embodied in the great observatories that mapped the universe.
