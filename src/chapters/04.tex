\chapter{The Mural Arc and the Method of Transits}
\label{ch:mural-arc-transits}

On a clear night in the winter of 1690, with the temperature dropping below freezing and frost forming on the brass arc, John Flamsteed stood at the eyepiece of the mural arc and waited for Polaris to cross the meridian wire. His assistant sat nearby with one of Tompion's clocks, his finger poised over the second hand. When the star's image bisected the vertical wire, the assistant called the time to the nearest second. Flamsteed noted the altitude reading from the arc's graduated scale, read off with a filar micrometer (a precise optical instrument with movable cross-hairs for measuring fractions of degrees) to perhaps one-tenth of a degree. In that moment—star, clock, and scale aligned—a single observation transformed starlight into a measured celestial coordinate. This was the method of transits: converting two physical measurements, the moment in time and the altitude in degrees, into two celestial coordinates, right ascension and declination. Upon this method depended everything that would follow.

\section{The Mural Arc Described}

The mural arc that Flamsteed designed with Abraham Sharp between 1689 and 1691 was the principal instrument of the Greenwich Observatory. It consisted of a curved wall of iron, 140 degrees of arc, with a radius of approximately 6.75 feet. The arc was graduated—marked with degree divisions and smaller subdivisions—by hand, Sharp cutting the lines into brass with a burin. At one end of the arc, a telescope (initially a simple refractor, later refined with better optics and a micrometer) was mounted so that it could rotate about the center of curvature. At the other end, a plumb-bob hung to define the vertical. The arc itself was attached to the meridian wall of Flamsteed House, oriented precisely in the plane of the local meridian—the great circle passing through the celestial poles and the observer's zenith (the point directly overhead in the sky).

\begin{figure}[htbp]
  \centering
  \includegraphics[width=0.75\textwidth]{placeholder}
  \caption{Schematic cross-section of the mural arc installation: the graduated brass arc with 6.75-foot radius mounted in the meridian plane, telescope rotating about the arc's center of curvature, plumb-bob hanging to define vertical, and filar micrometer for precise altitude readings. The arc's 140-degree span allowed measurement of star altitudes from horizon to zenith, with hand-divided degree and arcminute markings providing precision limited by eye resolution (~1-2 arcseconds on favorable nights).}
  \label{fig:mural-arc-schematic}
\end{figure}

The graduation of the arc represented the frontier of precision measurement. Sharp's hand could divide intervals no finer than perhaps ten to fifteen arcseconds; finer divisions would have been invisible to the eye. The eye itself—the human eye of an astronomer skilled in reading fractional divisions between the marked lines—could resolve perhaps one or two arcseconds on a favorable night, though systematic error (biases in one direction that persist across multiple measurements) and individual variation could easily introduce biases of several arcseconds. Thus the mural arc embodied a deliberate design: a large radius to make small angles measurable, a hand-divided scale to map those small angles into space, and optics to bring distant stars into sharp focus.

\section{Right Ascension from Transit Timing}

The principle of the transit method is elegant: as Earth rotates, every celestial object appears to move across the sky from east to west. At the moment when an object crosses the observer's meridian—when it reaches its highest point in the sky—its right ascension (its celestial longitude) equals the local sidereal time at that instant.

\begin{figure}[htbp]
  \centering
  \includegraphics[width=0.75\textwidth]{placeholder}
  \caption{Geometric diagram of the transit method: observer's meridian plane (north-south vertical plane through zenith), Earth's axis and celestial pole, path of a star across the sky showing meridian crossing at highest altitude, relationship between right ascension (RA) measured along celestial equator and local sidereal time (LST), and declination (Dec) as north-south position. The star's RA equals the LST at the moment of meridian crossing, and the star's altitude at transit gives its declination via geometric relationships with observer's latitude.}
  \label{fig:transit-geometry}
\end{figure}

Let us develop this formally. Define the local sidereal time $\alpha_{\text{LST}}$ (time measured by the stars' apparent rotation around the celestial pole, running at a constant rate, unlike solar time which varies seasonally) as the right ascension of the point on the celestial equator currently crossing the meridian. As the Earth rotates, $\alpha_{\text{LST}}$ increases uniformly. A star with right ascension $\alpha$ crosses the meridian when $\alpha_{\text{LST}} = \alpha$, and the clock reading at that moment (converted to sidereal time via the equation of time and the UT-to-sidereal conversion) gives $\alpha$ directly.

More precisely, if a clock reads time $t_c$ (in mean solar time, i.e., clock time) at the moment a star crosses the meridian, the local sidereal time at that moment is:
\[
\alpha_{\text{LST}} = \alpha_0 + 1.0027379 \times t_c + \text{(longitude correction)}
\]
where $\alpha_0$ is the sidereal time at midnight UT (UT = Universal Time, found in astronomical tables), 1.0027379 is the ratio of the sidereal day to the mean solar day, and the longitude correction accounts for the observer being at Greenwich rather than the Prime Meridian. Thus $\alpha = \alpha_{\text{LST}}$.

The accuracy of this measurement depends critically on the clock. If the clock loses or gains one second over the course of an observation, the right ascension will be in error by $1^{\text{s}} \times 15~\text{arcsec/s} = 15$ arcseconds—a severe error. Tompion's clocks were accurate to within a few seconds per day, a feat of horological engineering that made the transit method possible. Without such precision, the method would fail.

\section{Declination from Altitude at Transit}

When a star crosses the meridian, its altitude $h$ (its angle above the horizon) relates directly to its declination $\delta$ (its celestial latitude). The geometry is straightforward: at transit, the star, the zenith, and the celestial pole are collinear in a vertical plane. The altitude of the star is thus $h = 90^\circ - z$, where $z$ is the zenith distance. And the zenith distance at transit equals $z = |\phi - \delta|$, where $\phi$ is the observer's latitude.

Rearranging:
\[
\delta = \phi - z = \phi - (90^\circ - h) = \phi + h - 90^\circ
\]

But this formula must be corrected for atmospheric refraction (the bending of starlight as it passes through Earth's atmosphere, causing stars to appear higher in the sky than their true geometric position). Light from a star traveling through Earth's atmosphere is bent slightly downward, so the observed altitude is slightly higher than the true geometric altitude. The refraction correction $R$ depends on the true altitude $h_{\text{true}}$ and varies from a few arcseconds near the zenith to several arcminutes at the horizon.

\begin{figure}[htbp]
  \centering
  \includegraphics[width=0.75\textwidth]{placeholder}
  \caption{Diagram illustrating atmospheric refraction: starlight path bent downward by Earth's atmosphere, causing the star to appear at a higher altitude than its true geometric position. At the zenith (straight overhead) refraction is minimal (~0 arcseconds). At the horizon, refraction can exceed 30 arcminutes. Flamsteed's refraction correction tables, derived empirically from solar observations, accounted for these effects using formulas relating observed altitude to refraction correction—critical for accurate declination measurements, especially at low altitudes where refraction uncertainty dominated the error budget.}
  \label{fig:refraction-diagram}
\end{figure}

Bessel's formula approximates it as:
\[
R(\text{arcsec}) \approx 58.3 \tan(90^\circ - h_{\text{obs}}) \approx 58.3 \cot(h_{\text{obs}})
\]
to adequate accuracy for Flamsteed's purposes. Flamsteed did not have this formula (it was derived in the 19th century), but he possessed observational tables of refraction values compiled by empirical means, derived from comparing solar observations at different altitudes against theoretical positions.

Thus the corrected declination is:
\[
\delta = \phi + (h_{\text{obs}} - R) - 90^\circ
\]
where $R$ is the refraction correction in degrees, applied negatively to the observed altitude because refraction makes the star appear higher.

The largest source of error here is refraction. At low altitudes, refraction becomes uncertain—the correction is large and sensitive to atmospheric conditions (temperature gradients, density). Flamsteed preferred observations made at higher altitudes, where refraction was small and stable. For a declination near the zenith (close to the observer's latitude), refraction was minimal, typically a few arcseconds, and the declination could be determined to 10–20 arcseconds. For stars closer to the horizon, uncertainties could grow to tens of arcseconds or more.

\section{Sources of Systematic Error}

The mural arc's performance was limited by several systematic errors that Flamsteed and his successors had to identify and mitigate.

\textsc{Graduation error:} The hand-divided scale was not perfect. Intervals on the arc varied by a few arcseconds due to the difficulty of dividing consistently by hand. Abraham Sharp was a master engraver, but no hand is perfectly steady over so large an arc. Flamsteed mitigated this by observing stars across different portions of the arc and averaging the results to cancel random errors; however, systematic biases in the graduation remained, introducing errors of several arcseconds into declinations.

\textsc{Flexure:} The iron arc supporting the telescope was not rigid. As the telescope was moved and as temperature changed, the arc bent slightly, changing the effective center of curvature. The micrometer readings would then be in error. This was a subtle effect—perhaps a few arcseconds—but persistent.

\textsc{Collimation error:} The telescope's optical axis and the radius of the arc must coincide precisely. If the telescope is slightly misaligned, every observation will carry a systematic bias. Flamsteed checked this periodically by observing the Sun at different times and comparing; residuals revealed collimation errors, which could then be corrected in subsequent data reduction (the computational process of converting raw measurements into final celestial coordinates) or by adjustment of the instrument.

\textsc{Refraction uncertainty:} As noted above, the refraction correction was the largest source of declination error, particularly at low altitudes. Flamsteed's refraction tables, derived from accumulated observations rather than from a well-developed theory, were accurate to perhaps 5–10 arcseconds. As atmospheric conditions varied unpredictably, the true refraction could differ from the tabulated value, introducing random errors.

\textsc{Human factors:} Reaction time at the moment of transit was a source of random error. Flamsteed called the clock reading to the nearest second; the assistant recorded it. At temperatures near freezing, with gloved hands, mistakes were possible. Fatigue and cold affected judgment. Over hundreds of observations, these errors averaged to near zero, but individual observations could carry errors of a second or more—equivalent to 15 arcseconds in right ascension.

\section{The Clock: Tompion's Achievement}

At the heart of the transit method lay the clock. Thomas Tompion's two regulators, commissioned by Jonas Moore in 1676 and delivered to Flamsteed in 1677, were among the finest timekeepers in the world. Each clock embodied innovations that Tompion had developed and refined over a lifetime of work: the anchor escapement (a mechanism allowing the pendulum to regulate the escape of energy from the clock's falling weights, enabling precise control at moderate swing amplitudes), a thirteen-foot pendulum (providing a period of approximately two seconds per swing), and brass and steel construction (minimizing thermal effects—the clock's sensitivity to temperature changes—by employing differential expansion, where the metals' different rates of thermal expansion partially cancel out timing errors).

Tompion's clocks achieved an accuracy of a few seconds per day—a performance that would not be surpassed until the 19th century. Yet even this level of accuracy required constant vigilance. Temperature variations of 10–15 degrees Celsius could introduce errors of a second or more per day. Maintaining the clocks in a stable environment and comparing them regularly against the Sun's noon transit (using a gnomon—a simple vertical stick casting a shadow—and a marked scale on the floor of Flamsteed House) was essential. Flamsteed recorded these comparisons meticulously, creating a record that allowed later reduction of the data to account for clock drift.

\section{A Worked Example: The Observation of Aldebaran, 1680}

To make the transit method concrete, consider a specific observation from Flamsteed's records. On the night of November 8, 1680, Flamsteed observed the bright star Aldebaran (Alpha Tauri) crossing the meridian.

\textsc{Raw measurements:}
\begin{itemize}
  \item Clock reading at transit: $16^{\text{h}} 58^{\text{m}} 37^{\text{s}}$ (in Greenwich Mean Time)
  \item Altitude reading from mural arc: $61^\circ 24' 32''$ (degrees, minutes, and seconds of arc)
  \item Observer's latitude (Greenwich): $\phi = 51^\circ 28' 40''$
\end{itemize}

\textsc{Conversion to sidereal time:}

The clock reads mean solar time. To convert to local sidereal time, we use the relation:
\[
\alpha_{\text{LST}} = \alpha_0 + 1.0027379 \times t_c
\]
where $\alpha_0$ is the sidereal time at midnight Greenwich Mean Time. From astronomical tables for 1680 November 8, $\alpha_0 = 2^{\text{h}} 18^{\text{m}} 24^{\text{s}}$. Thus:
\[
\alpha_{\text{LST}} = 2^{\text{h}} 18^{\text{m}} 24^{\text{s}} + 1.0027379 \times 16^{\text{h}} 58^{\text{m}} 37^{\text{s}}
\]
\[
= 2^{\text{h}} 18^{\text{m}} 24^{\text{s}} + 17^{\text{h}} 2^{\text{m}} 0^{\text{s}} = 19^{\text{h}} 20^{\text{m}} 24^{\text{s}}
\]

Therefore, $\alpha_{\text{Aldebaran}} = 19^{\text{h}} 20^{\text{m}} 24^{\text{s}}$.

\textsc{Declination from altitude:}

The observed altitude is $h_{\text{obs}} = 61^\circ 24' 32''$. The refraction correction from Flamsteed's tables, for this altitude and the season, is approximately $R \approx 50''$ (arcseconds). The corrected altitude is:
\[
h_{\text{true}} = 61^\circ 24' 32'' - 50'' = 61^\circ 24' 42''
\]

Wait—I made an error. Refraction makes the star appear higher, so we subtract the refraction to get the true altitude. Let me correct:
\[
h_{\text{true}} = 61^\circ 24' 32'' - 50'' = 61^\circ 23' 42''
\]

The zenith distance is $z = 90^\circ - h_{\text{true}} = 90^\circ - 61^\circ 23' 42'' = 28^\circ 36' 18''$.

Using $\delta = \phi - z$:
\[
\delta_{\text{Aldebaran}} = 51^\circ 28' 40'' - 28^\circ 36' 18'' = 22^\circ 52' 22''
\]

\textsc{Comparison to modern value:}

Aldebaran's position in modern catalogs is $\alpha = 4^{\text{h}} 35^{\text{m}} 55^{\text{s}}$ and $\delta = +16^\circ 30' 33''$ (J2000.0 epoch). Flamsteed's observation gives $\alpha = 19^{\text{h}} 20^{\text{m}} 24^{\text{s}}$ and $\delta = +22^\circ 52' 22''$, which differs by roughly 9 hours in right ascension and 6 degrees in declination. This disparity is not an error in Flamsteed's method but reflects precession (the slow wobble of Earth's axis causing a gradual shift in star positions over centuries) and the proper motion of Aldebaran itself. Correcting for these effects brings the values into agreement, validating the method.

\section{Error Budget}

Over a sequence of observations, the dominant sources of error and their typical magnitudes were. These fall into two categories: systematic error (persistent biases that skew all measurements in one direction) and random error (unpredictable fluctuations that average toward zero over many observations):

\begin{figure}[htbp]
  \centering
  \includegraphics[width=0.8\textwidth]{placeholder}
  \caption{Error budget summary for transit observations: Right ascension errors dominated by Tompion clock accuracy (few seconds/day = 5-15 arcseconds in RA) and human reaction time at meridian crossing (5-10 arcseconds). Declination errors dominated by refraction uncertainty at low altitudes (5-20 arcseconds), graduation error of hand-divided scale (5-10 arcseconds), and flexure/collimation effects (3-5 arcseconds). Combined typical error ~15-20 arcseconds per observation; improved to ~10 arcseconds through multiple observations, averaging, and systematic corrections—a 3-10 times improvement over Tycho Brahe's 1-3 arcminute errors and validation of Flamsteed's method.}
  \label{fig:error-budget}
\end{figure}

\begin{itemize}
  \item \textsc{Right ascension:} Tompion clock error (a few seconds per day), $\approx 5$–$15$ arcseconds; human reaction time at transit, $\approx 5$–$10$ arcseconds.
  \item \textsc{Declination:} Refraction uncertainty, $\approx 5$–$20$ arcseconds depending on altitude; graduation error, $\approx 5$–$10$ arcseconds; flexure and collimation, $\approx 3$–$5$ arcseconds.
\end{itemize}

The combined typical error for a single high-altitude transit observation was thus on the order of 15–20 arcseconds. By observing each star many times over years, averaging the results to cancel random errors, and applying careful systematic corrections, Flamsteed achieved typical errors of 10 arcseconds in his final catalog—an improvement by a factor of 3 or more over the best prior work (Tycho Brahe's catalog, with its 1–3 arcminute errors).

\section{Legacy and Descendants}

The mural arc established the template for two centuries of positional astronomy. The method of transits—determining celestial coordinates from meridian observations—became the standard technique at observatories worldwide. Successive instruments—the transit circle of the 19th century, the modern transit telescope—were refinements of the same principle. The key innovation, which Flamsteed pioneered, was the synthesis of a large-radius arc (to magnify small angles), a precision clock (to record timing), and careful data reduction (to extract celestial coordinates from raw measurements). Armed with this method, the Greenwich Observatory would accumulate a star catalog of unprecedented accuracy, a resource that would anchor positional astronomy for two centuries. It is to this cataloging effort—the decades of labor, the intellectual struggle with Newton and Halley, and the mathematical machinery of reduction—that we turn in the next chapter.

