\chapter{The 1884 Meridian Conference}
\label{ch:meridian-conference}

In October 1884, Washington was unseasonably warm. Delegates from twenty-five nations gathered in the U.S. Board of Trade building to solve a problem that had bedeviled commerce, science, and diplomacy for centuries: which meridian should mark the prime?\footnote{\textcite{InternationalMeridianConference1884} provides the official proceedings of the conference, documenting each nation's position, technical arguments, and voting records.}

France's delegate rose to argue for a meridian through the Atlantic Ocean---neutral territory, belonging to no nation, marked only by the mathematics of geography. The gallery fell silent. Then Britain's delegate noted a fact: seventy-two percent of the world's merchant shipping already used charts referenced to Greenwich. Practical advantage defeated principle. By vote---22 in favor, 1 against (San Domingo), 2 abstentions (France and Brazil)---Greenwich became the world's prime meridian. France would adopt the time anyway, calling it ``Paris Mean Time diminished by 9 minutes 21 seconds'' for decades, a petulant footnote in history. But from October 13, 1884 onward, a single meridian became the reference for all the world's charts, all its time zones, all its coordinated human endeavor.

\section{Why a Single Prime Meridian?}

For most of history, nations used their own meridians. Paris observed from the Paris Observatory. Greenwich observed from Greenwich. Washington had its own reference line. The Paris meridian ran through the Royal Observatory, just as the Greenwich meridian ran through Greenwich's transit circle. These meridians differed: London-to-Paris is about 2° 20' of longitude. For a sailor on a ship with limited instruments, using the wrong reference could mean a ten-nautical-mile error in position at the equator alone.

The practical problem emerged in the mid-19th century as four technologies converged: accurate marine chronometers (making it possible to carry time across oceans), extensive hydrographic surveys (creating millions of detailed charts), telegraph networks (allowing coordination across continents), and expanding global commerce. A merchant vessel carrying cargo between London and Bombay might consult British charts (referenced to Greenwich), Indian charts (referenced to various Madras or Calcutta conventions), and French colonial charts (referenced to Paris). Reconciling these required conversion between different meridian systems---an added source of error.

Railways made coordination even more urgent. A train running from London to Dover at 30 miles per hour traverses 15 nautical miles per hour, equivalent to 15 minutes of arc per hour---one minute of arc per four seconds of travel. If stations along the line used different time references, schedules were ambiguous. Should the Dover terminus use London time or Paris time? If a passenger train and a freight train were scheduled to meet, which clock determined simultaneity?

Telegraph networks added a final pressure. When messages traveled along wires at the speed of light, global coordination was theoretically instant---but only if sender and receiver agreed on time. A telegraph operator in Mumbai needed to know the precise time in London to schedule coordinated transmissions. A single, universally accepted prime meridian would solve this.

\section{The Candidates}

The International Meridian Conference of 1884 heard arguments for five primary candidates:

\textbf{Greenwich:} The Greenwich meridian passed through Airy's transit circle---the instrument that defined the Observatory's observational practice and produced the Nautical Almanac used by the majority of the world's navies. Britain, the dominant naval power, had built an extensive library of charts and tables referenced to Greenwich. Seventy-two percent of merchant vessels carried charts with longitude referenced to Greenwich.

\textbf{Paris:} The Paris meridian ran through the Royal Observatory at Paris, one of the world's premier astronomical institutions. French scientists argued that Paris, not Britain, should define the world's reference. France could point to substantial scientific contributions---the Academy of Sciences, the precision of French instruments---and to the mathematical elegance of a meridian belonging to no single nation-state but to the institution of science itself.

\textbf{Washington:} The United States, as host of the conference, proposed the Washington meridian through the Naval Observatory on Observatory Hill. The argument was appealing to American self-interest: a neutral western hemisphere location, reasonably central to the continental United States, and a symbol of American scientific authority.

\textbf{Ferro:} The Ferro meridian, running through the Canary Island of Hierro (also spelled Ferro), had historical precedent. Mapmakers in the 16th and 17th centuries had used Ferro as a zero meridian, in part because it marked the edge of the known world---the westernmost point of European cartography. Reviving Ferro would have honored that history while avoiding the appearance of favoring any modern nation.

\textbf{The ``Neutral'' Meridian:} France's compromise proposal suggested an artificial meridian bisecting the Atlantic Ocean, equidistant from all inhabited continents. This meridian would mark no nation's territory and would require all countries equally to measure longitude from a mathematical abstraction rather than a physical observatory. From a political standpoint, this proposal had merit: it avoided favoritism. From a practical standpoint, it had fatal flaws. No observatory existed there. No charts referenced it. No existing nautical tradition supported it.

\section{The Technical Arguments}

Beneath the diplomatic posturing lay genuine technical considerations. A prime meridian needed to satisfy three criteria: it must be defined by an instrument of indisputable precision; existing charts and tables referencing it must be numerous enough to justify conversion costs; and the meridian must be accessible for ongoing refinement as precision improved.

Greenwich satisfied all three. Airy's transit circle, installed in 1850, was one of the most precise zenith-distance instruments ever built. Its defining circle, illuminated and visible through the eyepiece, allowed observers to measure star positions with precision exceeding 0.5 arcseconds---better than one part in 100,000 of a full circle. The Nautical Almanac, published annually by the Nautical Almanac Office at Greenwich, gave the positions of celestial bodies calculated to Greenwich time and Greenwich longitude. Every ship in the British navy carried this almanac. Many merchant vessels did as well, particularly those trading with British ports. Charts published by the Hydrographic Office of the British Admiralty, the most extensive hydrographic surveying organization in the world, referenced Greenwich.

Paris had comparable precision---the Paris Observatory's instruments were excellent---but fewer ships carried French charts. France's own navy was no match for Britain's, nor was the French merchant marine its equal. French maps and tables existed, but their market penetration was lower.

Washington was new, precise, but isolated. American naval charts existed, but the Naval Observatory's role in international navigation was minimal. Few merchant vessels consulted Washington time.

The technical advantage was Greenwich's.

\section{The Vote}

On October 13, 1884, the delegates voted. The resolution, formally titled ``On the Adoption of a Prime Meridian to be Common to All Nations,'' asked whether Greenwich should be this meridian.

Twenty-two nations voted in favor: Austria-Hungary, Brazil (despite Brazil's abstention on the final vote), Chile, Colombia, Costa Rica, Denmark, France (surprisingly, in favor despite the subsequent abstention), Germany, Great Britain, Haiti, Italy, Japan, Mexico, Netherlands, Paraguay, Portugal, Russia, San Salvador, Spain, Switzerland, Turkey, and Uruguay.

One nation voted against: San Domingo (the Dominican Republic), whose delegate argued that the conference was premature and that scientific consensus had not been sufficiently established.

Two nations abstained: France reconsidered and abstained on the final vote (despite voting in favor on the resolution itself, indicating internal French conflict), and Brazil abstained, perhaps from political alignment with France or from genuine indecision.

The voting breakdown appears in the following table:

\begin{center}
\begin{tabular}{lc}
\hline
\textbf{Position} & \textbf{Nations} \\
\hline
In Favor (22) & Austria-Hungary, Brazil, Chile, Colombia, Costa Rica, \\
 & Denmark, France, Germany, Great Britain, Haiti, Italy, \\
 & Japan, Mexico, Netherlands, Paraguay, Portugal, Russia, \\
 & San Salvador, Spain, Switzerland, Turkey, Uruguay \\
Against (1) & San Domingo \\
Abstaining (2) & France (reconsideration), Brazil (final vote) \\
\hline
\end{tabular}
\end{center}

The motion carried overwhelmingly. Greenwich was adopted as the prime meridian. But adoption and compliance are not synonymous. France, stung by the outcome, would use ``Paris Mean Time minus 9 minutes 21 seconds'' as its official time reference for decades---a mathematically cumbersome way of maintaining pride while acknowledging the Greenwich standard. Eventually, convenience triumphed, and France adopted Greenwich time directly.

\section{The Universal Day}

The conference's second major decision concerned the universal day---should the world's calendar day begin at midnight (civil convention) or noon (astronomical convention)?

Astronomers, by convention stretching back to the 18th century, began their observational day at noon. An astronomer observing on the night of December 24--25 would record that observation as occurring on December 24 (the astronomical day beginning at noon on December 24 and ending at noon on December 25). This convention avoided splitting a night's observations across two calendar dates.

Civil timekeeping, by contrast, began the day at midnight. A person going to bed at 11 PM on December 24 and waking at 6 AM on December 25 experienced only one ``day,'' despite crossing the midnight threshold.

The conference voted to adopt the civil convention for the universal day: beginning at midnight, ending at midnight. Astronomers would need to adjust their practices. Subsequent astronomical convention adopted ``Julian Day Number''---a continuous count of days beginning in 4713 BCE---as a compromise, eliminating ambiguity about where one day ends and another begins.

\section{Time Zones and the 15° Rule}

With a single prime meridian established, the question of time zones became tractable. If all time is to be referenced to Greenwich, but local civil practice demands time to roughly match solar noon at the observer's location, then the solution is bands of longitude, each maintaining a fixed offset from Greenwich time.

The interval chosen was 15 degrees of longitude---the Earth rotates 360 degrees in 24 hours, so 15 degrees corresponds to exactly one hour. A location at 15° east of Greenwich observes local solar noon one hour earlier than Greenwich's noon; it therefore uses Greenwich time plus 1 hour (``1 hour ahead'').

This 15° standard, while mathematically clean, obscures the actual geography. Not all nations aligned their time zones with multiples of 15°. Political boundaries, economic ties, and geographic factors led to irregular zones. India, for instance, uses a single time zone 5 hours and 30 minutes ahead of Greenwich---not a multiple of 15 degrees. But the 15° rule became the default, with exceptions noted. A useful derivation is provided in Appendix F.

\section{Adoption and Delay}

Adoption of the Greenwich meridian at the conference did not produce immediate universal compliance. Government agencies, railway companies, and maritime authorities around the world adopted Greenwich-based systems at different speeds.

British railways had already begun coordinating via Greenwich time in the 1850s. The British Post Office telegraph system used Greenwich time. France, despite its abstention and subsequent stubbornness about ``Paris Mean Time diminished by ...,'' adopted Greenwich time for most purposes by the 1920s. Germany adopted Greenwich meridian for surveying by 1900. Japan, as a modernizing nation eager to integrate into global commerce, adopted Greenwich rapidly. The United States was slower: American railways used a patchwork of local times until 1883, when they adopted four time zones offset by whole hours from a reference meridian.

By 1920, Greenwich time was the de facto standard for international navigation, telecommunications, and railway coordination. By 1960, it was universal.

\section{What Was Resolved and What Was Not}

The 1884 conference clarified that a single meridian would serve as the world's geographic reference. It established that this meridian would be Greenwich, and that time zones would be derived from Greenwich Mean Time, each offset by an integer number of hours (with exceptions made for political and economic convenience).

What the conference did not resolve was the nature of the day itself. Was the universal day to be measured by Earth's rotation (solar time, subject to seasonal and orbital irregularities) or by some more abstract standard? This question would not be answered until 1967, when the SI second was defined not by Earth's rotation, but by the cesium-133 atom. For now, Greenwich Mean Time was what it claimed: the mean (smoothed average) of the actual solar time at Greenwich, corrected for the equation of time.

The conference also left unresolved the question of how time information would be transmitted. The telegraph existed, the time ball existed, but radio was not yet invented. Distribution mechanisms would evolve, but the reference---Greenwich---was now fixed.

\section{The Prime Meridian Today}

The prime meridian marked by Airy's transit circle remains visible at Greenwich Observatory, now a museum. Tourists stand astride a brass line set into the ground, with one foot in the Eastern Hemisphere, one in the Western. They photograph the moment. But the actual prime meridian, defined by the most precise geodetic measurements, lies 102 meters to the east of this tourist marker. The WGS84 geodetic reference frame, used by GPS and modern surveying, defines the prime meridian not by a transit circle, but by a statistical adjustment of thousands of survey measurements. Airy's circle is historically and culturally significant; it is not technically the prime meridian anymore.

Yet the choice made in 1884 endures in its essence. Every time zone on Earth is defined relative to Greenwich. Every map, every GPS coordinate, every timestamp in coordinated universal time traces back to that October vote in Washington and to the practical decision to use the observatory that had best solved the problem of measuring the stars.
