\chapter{The Deadly Ignorance of Position}
\label{ch:deadly-ignorance}

\section{The Scilly Disaster}
\label{sec:opening-vignette}

The fog rolled across the Atlantic in October 1707, thick as wool.\index{Scilly disaster (1707)} \textsc{hms} Association, the flagship of Admiral Sir Cloudesley Shovell,\index{Shovell, Cloudesley} cut through black water with the English fleet returning from Gibraltar---weeks at sea, the navigators' calculations checked and rechecked, the officers confident they had a safe margin west of the Scilly Isles. At eight o'clock in the evening, the Western Rocks rose out of the darkness without warning. The Association struck first, breaking apart instantly; within minutes, Eagle, Romney, and Firebrand followed her onto the reefs. The sea boiled white around the wrecks as the ships shattered on stone and the men struggled in water too cold for survival. Between fourteen hundred and two thousand men died that night---officers and ordinary sailors indistinguishable in the violence of water and rock. By morning, bodies washed onto black sand beaches. Shovell himself, waterlogged and unrecognizable until local women found his ring, lay among them. The court of inquiry that followed found nobody obviously at fault: the officers had computed their position with care, the calculations were correct by the standards of the time, yet the fleet had been more than forty nautical miles off course in precisely the direction that put them on the rocks.\footnote{The most detailed account of the disaster is from the \emph{Weekly Journal} of 3 November 1707 and the court of inquiry transcript, preserved in Admiralty Records. Sobel's \emph{Longitude} provides a modern narrative synthesis.}

\begin{figure}[htbp]
  \centering
  \includegraphics[width=0.45\textwidth]{photos/ch01-admiral-shovell}
  \caption{Admiral Sir Cloudesley Shovell (1650--1707), portrait by Michael Dahl, c.~1702. Five years after this portrait was painted, Shovell would perish with his fleet on the Western Rocks.}
  \label{fig:admiral-shovell}
\end{figure}

\section{What Is Longitude?}
\label{sec:definition}

The problem of position on Earth is geometrically elementary.\index{longitude!definition}\index{latitude!definition} The rotating sphere requires two coordinates: one measuring angle north or south from the equator (latitude, $\phi$), the other measuring angle east or west from an arbitrary reference meridian (longitude, $\lambda$). For three centuries, mariners could determine the first coordinate with reasonable accuracy but had no practical means to determine the second. This asymmetry---one problem solved, one unsolvable---was not an accident of geography but a fundamental consequence of how the Earth rotates and how the heavens appear.

\subsection{Latitude: Celestial Geometry Cooperates}

Latitude admits a direct geometric solution. Stand anywhere on Earth and look toward the celestial pole (south toward the South Celestial Pole if you are in the Southern Hemisphere, north toward the North Celestial Pole if in the north). The altitude of that pole above the horizon, measured in degrees, is your latitude. This relationship is absolute:

\begin{equation}
\phi = h_{\text{pole}}
\end{equation}

where $h_{\text{pole}}$ is the altitude of the celestial pole above the horizon.\footnote{Strictly, the altitude of the true pole; the observation uses Polaris, which lies within about $1^{\circ}$ of the true pole in the epoch of the 17th century, introducing a small but measurable correction.}

The geometry cooperates because the celestial poles lie on the axis of the Earth's rotation. An observer standing on the rotating surface naturally stands on a tilted coordinate system, and that system writes its inclination onto the sky. The observer's latitude is literally the angle between their horizon and the celestial equator.

\begin{figure}[htbp]
  \centering
  \includegraphics[width=0.6\textwidth]{generated/ch01-latitude-geometry}
  \caption{The celestial pole altitude equals the observer's latitude. An observer at latitude $\phi$ sees the pole at altitude $h = \phi$ above the horizon. At the equator the pole is at the horizon ($h = 0^{\circ}$); at either terrestrial pole it is directly overhead ($h = 90^{\circ}$).}
  \label{fig:latitude-geometry}
\end{figure}

In practice, three observational methods converge to the same result. First, the pole star method: observe the altitude of \emph{Polaris} and apply a small correction for its offset from the true pole. Second, the noon Sun method: measure the Sun's maximum altitude at solar noon, determine its declination from ephemeris tables (tabulated astronomical position predictions), and apply spherical geometry. The declination $\delta$ is the Sun's angular distance north ($+$) or south ($-$) of the celestial equator, measured in degrees; at the moment of meridian transit (the instant when the Sun crosses the observer's meridian, also called true solar noon), the relationship is:

\begin{equation}
\phi = \delta + (90^{\circ} - h_{\text{sun}})
\end{equation}

where $h_{\text{sun}}$ is the observed altitude. Third, circumpolar star observations: circumpolar stars are those that never set below the horizon; they orbit the celestial pole without disappearing. For any such star, observe its altitude at upper culmination (the moment when it reaches its highest point in the sky) and at lower culmination (when it reaches its lowest point), then latitude follows from the average of these two altitudes.\footnote{The precise formula involves spherical trigonometry; the simple form given here is approximate for stars at small polar distances.}

A careful observer with an adequate instrument---a quadrant or astrolabe with graduated arc and an ability to read to the nearest minute of arc---can achieve latitude accurate to within one degree, the apparent diameter of the full Moon. Better observers in better conditions achieve half that error.\footnote{Chapman, \emph{Dividing the Circle}, provides an authoritative technical account of instrument performance and observational precision in the late 17th century.}

\subsection{Longitude: The Hidden Coordinate}

Longitude has no corresponding celestial marker. No star, no planet, no celestial feature sits directly above the Prime Meridian (or any other reference meridian). The sky looks essentially identical to an observer in London and an observer in Gibraltar, except for one crucial difference: the time at which the stars rise and set.

And here lies the essential insight that unlocks the entire problem. The difference in longitude between two places equals the difference in their local solar times, converted to angle:

\begin{equation}
\lambda_{\text{observer}} - \lambda_{\text{reference}} = \Delta t \times \left(\frac{360^{\circ}}{24 \text{ hours}}\right) = \Delta t \times 15^{\circ}/\text{hour}
\end{equation}

The conversion factor $15^{\circ}$ per hour follows directly from Earth's rotation: the planet rotates $360^{\circ}$ in $24$ hours, or equivalently $15^{\circ}$ per hour of rotation. If two observers experience a one-hour difference in local solar time (the time when the Sun reaches maximum altitude), they are separated by a longitude difference of $15^{\circ}$ of arc. If you know the time at a reference meridian (Greenwich, say) and you know your local time (determinable from the Sun's altitude), the difference between them gives your longitude.

This is a statement of pure geometry. It is also a statement of impossibility: to determine your longitude at sea, you must simultaneously know two times separated by a vast distance. You can determine your local time by measuring the Sun's altitude at noon. But how do you know the time at Greenwich while standing in the middle of the Atlantic Ocean? There is no signal that travels faster than a ship. There is no mechanism that preserves time across a rolling deck for months. This is the core of the longitude problem, and it has no solution in astronomy alone.

\subsection{Dead Reckoning and Cumulative Error}

Without an absolute time reference, the navigator's only tool was \emph{dead reckoning}---the calculation of position by integrating estimates of velocity and direction over elapsed time. The method was ancient and crude. Each watch, the officer on deck would estimate the ship's speed by dropping a wooden chip attached to a knotted rope into the water ahead and counting how many knots passed through his hand in a measured time interval (\emph{chip log}). He would note the compass heading. These estimates---speed and direction---were recorded in the ship's logbook. When summed over hours and days, they became the calculated position.

The mathematics of dead reckoning is straightforward in principle. If the ship travels at speed $v(\tau)$ on heading $\theta(\tau)$ over elapsed time $t$, the change in position is the integral of velocity components along the longitude and latitude directions:

\begin{align}
\Delta \lambda &= \int_0^t v(\tau) \cos(\theta(\tau)) \, d\tau \quad \text{(eastward component)} \\
\Delta \phi &= \int_0^t v(\tau) \sin(\theta(\tau)) \, d\tau \quad \text{(northward component)}
\end{align}

In practice, the navigator approximates these integrals by summing the discrete speed and heading estimates recorded in the ship's logbook for each watch period. The execution is catastrophic. The chip log is crude; speed estimates are commonly off by twenty percent. The magnetic compass varies in declination (the angle between magnetic north and the direction to Earth's true geographic pole) in ways that were not fully predictable in the seventeenth century. Ocean currents are invisible and unknown. A ship sailing through fog for days accumulates errors that grow in all directions simultaneously, amplifying and interacting.

\begin{table}[htbp]
  \centering
  \caption{Cumulative error in dead reckoning over a typical transatlantic crossing, 1650--1750.}
  \label{tab:dead-reckoning-error}
  \begin{tabular}{lll}
    \toprule
    \textsc{Days at Sea} & \textsc{Typical Error (nm)} & \textsc{Characteristics} \\
    \midrule
    5 & 10--20 & Random direction \\
    10 & 30--50 & Emerging systematic bias \\
    20 & 60--100 & Strongly westward \\
    30 & 100--150 & Westward systematic error \\
    40 (typical Atlantic) & 150--250 & Westward, rapidly growing \\
    \bottomrule
  \end{tabular}
  \tablenote{Values are approximate estimates from analysis of 17th--18th century navigation records. Actual errors varied widely depending on weather, crew skill, and instrument calibration. See Howse, \emph{Greenwich Time}, for compiled analysis.}
\end{table}

A longitude error of one degree at the latitude of the English Channel corresponds to about forty nautical miles---the distance between safe harbor and a reef. A crew could accumulate that error invisibly and discover it only when land appeared where no land was expected, or when the ship struck rock that the chart showed wasn't there.

\begin{figure}[htbp]
  \centering
  \includegraphics[width=0.85\textwidth]{generated/ch01-dead-reckoning-error}
  \caption{Cumulative position error in dead reckoning over a transatlantic voyage. The shaded region shows the typical range of error; the dashed line marks the critical threshold of approximately forty nautical miles---the margin between safe passage and disaster at English Channel latitudes.}
  \label{fig:dead-reckoning-error}
\end{figure}

\section{A Catalog of Maritime Disasters}
\label{sec:maritime-losses}

The Scilly disaster did not appear from nowhere. It was the culmination of a long history of losses, each traceable to the same invisible enemy: the impossibility of knowing where you were when the horizon had vanished.

\begin{enumerate}
  \item \textsc{1591: the São Thomé.} Portuguese galleon bound for India struck the coast near Sumatra and sank. The crew believed themselves to be six hundred nautical miles to the east. The ship carried nine hundred and forty-four people; fewer than two hundred survived.\footnote{Reported in contemporary Portuguese naval records; see Parry, \emph{Age of Reconnaissance}, for synthesis.}
  
  \item \textsc{1615: the Eendracht.} Dutch East Indiaman, separated from her convoy in the Indian Ocean, made unexpected landfall on the coast of Western Australia. The ship was lost, but the accidental discovery added an entire continent to European geographic knowledge.\footnote{\emph{Journal of the Eendracht}, published in Dutch; modern edition in translation by Masselman.}
  
  \item \textsc{1656: the Tryall.} English East Indiaman struck rocks off Western Australia, having misjudged her longitude severely. Forty men survived on a desolate island; only a handful were ever rescued.\footnote{Documented in East India Company records; Sobel provides narrative account in \emph{Longitude}.}
  
  \item \textsc{1691.} A squadron of English ships, attempting to make port at Plymouth in fog, struck rocks near the English coast. Five ships lost; the incident provoked outrage in Parliament and naval circles.\footnote{Pepys's naval correspondence discusses the incident and its political aftermath; see Howse, \emph{Greenwich Time}, chap. 2.}
  
  \item \textsc{1707: the Scilly Disaster.} \textsc{hms} Association, Eagle, Romney, and Firebrand, with Admiral Shovell commanding the fleet. Deaths estimated between fourteen hundred and two thousand. The court of inquiry concluded that no one was obviously at fault.
\end{enumerate}

Each loss was, by the standards of the time, inexplicable and blameless. The officers had followed established procedures, checked their calculations carefully, and used the finest instruments available. The instruments had been as good as the age could provide, and the methods were those taught in every maritime academy. Yet the ships had gone down anyway. The maritime powers of Europe found themselves staring at a problem that appeared to be fundamentally unsolvable with existing tools and knowledge.\footnote{Sobel's \emph{Longitude} provides an accessible synthesis of these disasters and their political consequences. Howse's \emph{Greenwich Time} offers more technical detail on the navigational failures. Willmoth's \emph{Flamsteed's Stars} traces the Observatory's role in the institutional response.}

\section{The Political Response}
\label{sec:longitude-act}

Pressure had been accumulating for decades before the crisis came to a head. The disaster of the Association crystallized vague anxiety into political urgency. In 1714, less than seven years after Shovell's death, Parliament passed the Longitude Act, offering a prize of \textsc{£20,000}---a sum equivalent to the cost of a large warship or the annual salary of several thousand working people---to anyone who could devise a method of determining longitude at sea to within thirty nautical miles.\footnote{The Longitude Act of 1714 (12 Anne c. 15) established the Board of Longitude in perpetuity. The full text is available in parliamentary records; modern analysis appears in Howse, \emph{Measure of All Things}, and Sobel, \emph{Longitude}.}

The immediacy and scale of the act reflected desperation. The Scilly disaster was still raw memory---the grief and anger not yet cooled. The accumulation of maritime losses over a century had produced a consensus among naval and political leaders: something had to be done, and done urgently.

Two competing visions emerged immediately. The astronomers believed the answer lay in the heavens---that careful observation of the Moon's motion against the stars, or the motion of Jupiter's moons, could provide a time signal that would propagate across the ocean in the form of ephemerides and tables. The clockmakers believed the answer lay in the machine---that a sufficiently accurate clock could be carried aboard a ship and would keep reference time, even in the midst of salt spray and the ship's violent motion.

The Royal Observatory at Greenwich, founded in 1675 by Charles II, had been established in part to lay the groundwork for astronomical solutions.\footnote{Charles II's warrant of 4 March 1675 is the founding document; excerpts appear in Howse, \emph{Greenwich Observatory: A History}, vol. 1. Flamsteed's *Historia Coelestis Britannica* (1725) was the first major output to feed the astronomical solution to longitude.} The observations that would enable lunar distance tables were only just beginning to accumulate, gathered by John Flamsteed with laborious care. And the battle between the two schools---the astronomers and the mechanical philosophers---would consume the next century, driving precision upward and enriching both institutions and nations.

\section{Forward to the Solutions}
\label{sec:bridge}

The Scilly disaster and its predecessors had revealed the magnitude of the problem and the inadequacy of existing methods. But before any solution could be attempted, two things had to be understood: what instruments and knowledge existed to pursue solutions, and what precision was already achievable with the best tools the age had developed. The astronomers and clockmakers who would compete for the Longitude Prize needed to know where the ceiling was, what constraints they faced, and what foundation of knowledge they could build upon. \cref{ch:instruments-methods} surveys the instruments and astronomical methods available when the prize was offered, explains why none was yet sufficient to the problem's demands, and traces the precision ceiling that the coming century would gradually push upward.
