\chapter{The Meridian Instruments}
\label{ch:meridian-instruments}

It is 1850 at the workshop of Troughton \& Simms on Fleet Street in London. An instrument maker—let us call him Friedrich—sits at a grinding wheel, polishing a steel pivot the size of a pencil's tip. The pivot will support a five-foot-diameter brass circle, weighing several hundred pounds, and must rotate with such smoothness and precision that its wobble is less than one-tenth of an arcsecond. Friedrich laps the pivot against a cast-iron spindle with progressively finer abrasives—first coarse emery, then fine, then jeweler's rouge on soft felt. The rotation must be true; the surface finish must be perfect. A full day's work to reduce the irregularity by a fraction of an arcsecond. This pivot will be part of an instrument for the Greenwich Observatory, used to determine the positions of thousands of stars. Its precision, though invisible, will propagate through century of navigation. The instrument maker's craft—hand labor, intuition, and the accumulated knowledge of a thousand small refinements—is the foundation upon which astronomical precision rests.\footnote{\textcite{Repsold1908} documents the instrument-making craft of this era in detail. \textcite{Chapman2003} provides modern analysis of the technical requirements.} This chapter surveys the meridian instruments:\index{meridian instruments}\index{instruments!meridian} the transit telescopes, the mural circles, and the combined meridian circles that were designed to observe stars as they crossed the meridian, the imaginary line running north-south through the zenith.

\section{The Meridian Instrument Concept}

An astronomical observation requires determining a star's position relative to a reference frame.\index{astrometry} For astrometry---the measurement of stellar positions---the most precise reference is the meridian itself: the great circle of the celestial sphere that passes through the observer's zenith and the celestial poles.

Any star, as Earth rotates, traces an arc across the sky. When that star crosses the meridian, its altitude reaches an extremum: either maximum (if the star is south of the zenith) or minimum (if north). At that instant of transit, the star's hour angle is zero—its right ascension equals the local sidereal time at that moment. By recording the clock time of transit and the altitude at transit, one can determine the star's right ascension and declination with precision.

The meridian instruments are designed to constrain observation to this single moment and direction. They are not general-purpose telescopes but specialized devices that allow rotation only in the meridian plane, eliminating degrees of freedom and thus eliminating sources of systematic error.

\section{The Transit Instrument}

The transit instrument\index{transit instrument}\index{instruments!transit} (or transit telescope) is the simplest meridian instrument. It is a telescope fixed to rotate only about an east-west axis. The axis is horizontal and lies in the meridian plane; the telescope, when rotated about this axis, can point anywhere from the northern horizon through the zenith to the southern horizon, all within the meridian.

The transit instrument requires several key features:

\textbf{The pivots}: The east-west axis is supported at two ends by pivots—precision bearings that allow the telescope to rotate smoothly. The pivots must be as nearly cylindrical as possible; any wobble in the pivot introduces a wobble in the pointing direction.

\textbf{The optical system}: The telescope must have sufficient magnification to see stars clearly and sufficient light grasp to observe faint stars. A typical transit telescope in the eighteenth century had a 2-4 inch aperture and a 6-10 foot focal length.

\textbf{The crosshairs}: At the focal plane of the eyepiece, a system of crosshairs allows the observer to determine the star's position on the meridian. The central vertical hair defines the meridian; if this hair is truly vertical and truly in the meridian plane, a star that aligns with this hair has zero hour angle.

\textbf{The level}: A sensitive level indicates whether the east-west axis is horizontal. Any tilt of this axis from horizontal introduces a systematic error into the transit timing.

\textbf{The collimation}: Before observing stars, the transit instrument must be collimated—its optical axis must be verified to align with the direction of true north and south. Collimation is done using a collimator, an external optical bench containing a light source positioned so that light rays traveling through the collimator's objective emerge parallel. This simulates an infinitely distant star. By observing this artificial star through the transit instrument, one can verify the instrument's alignment.

The transit instrument's great virtue is simplicity: it has only one degree of freedom (rotation about the east-west axis), and thus only one opportunity to be out of alignment. Its weakness is that it cannot measure altitude directly. To determine declination, one needs to know the altitude at transit, which requires either a second instrument or an auxiliary level.

\section{The Mural Quadrant and Mural Circle}

The mural quadrant is fixed to a wall, constrained to measure altitude in the meridian plane. The instrument consists of a large graduated arc—literally a quadrant of a circle, hence 90 degrees—fixed vertically with one edge aligned north-south and the other pointing upward (zenith).

A telescope, mounted on an arm pivoting at the center of this arc, can rotate from the northern horizon (0°) through the zenith (90°) to the southern horizon (back to 0°). As the telescope rotates, an index mark traces across the graduated scale, indicating the altitude.

The mural quadrant evolved into the mural circle—a full circle (360°) rather than a quadrant (90°). A full circle has several advantages: it provides redundancy (the altitude can be read at multiple positions on the circle), and the instrument can measure altitudes below the horizon, which is useful for correcting refraction.

Both instruments measure altitude and thus are capable of determining declination, but they do not directly provide right ascension. They must be paired with a clock or combined with a transit telescope to give time information.

\section{Airy's Transit Circle: The Defining Instrument}

George Biddell Airy, as Astronomer Royal, designed and oversaw the construction of the transit circle—an instrument combining the functions of both transit and mural circle. The Airy transit circle, completed in 1851, became the defining instrument for positional astronomy and the physical realization of the Prime Meridian.

The Airy transit circle consists of:

\textbf{The telescope}: An 8-inch achromatic refractor with a focal length of 11 feet 6 inches, providing sufficient light grasp and magnification for observing stars down to about magnitude 7.

\textbf{The circles}: Two graduated circles, each 6 feet in diameter, rigidly attached to the telescope. One circle is for reading right ascension (via the clock time); the other is for reading declination (via the altitude). Both circles are divided into arcminutes, and each circle is read using six microscopes placed around its circumference, minimizing the effect of any local irregularities in the graduation.

\textbf{The pivots}: The instrument rotates about an east-west axis supported by two large pivots, one on each end. These pivots are the critical precision elements; their quality directly determines the instrument's accuracy.

\textbf{The level}: A sensitive spirit level, placed on the east-west axis, allows the observer to verify that the axis is horizontal to within a fraction of an arcsecond.

\textbf{The collimator}: An external light source positioned to project parallel rays through the transit circle's objective. By observing this artificial star, the observer confirms that the instrument is properly aligned.

\section{Optical Requirements and Collimation}

The collimation of a meridian instrument is crucial. The optical axis must be in the meridian plane, and it must point in a direction that is truly north-south. Any deviation from north-south introduces a systematic error into the declination measurement.

The collimator works by producing a parallel beam of light from a point source. If the telescope is perfectly aligned, the collimator's light source appears to be infinitely far away, on the meridian, at a particular declination. By focusing the transit circle on the collimator's light source and noting its position on the circle, one can determine any deviation of the instrument's optical axis from north-south.

The collimation error can be split into two components: a misalignment in the meridian plane (causing an error in declination) and a misalignment perpendicular to the meridian plane (causing an error in the altitude measurement, which propagates to declination).

To minimize collimation errors, observations of the collimator are taken regularly—typically daily or weekly—and any systematic shift is noted and corrected. Airy's detailed records of collimation checks show how much the instrument's alignment varied due to thermal expansion, mechanical wear, and other causes.

\section{Mechanical Requirements: Pivots, Bearings, and Flexure}

The mechanical precision of a meridian instrument depends critically on the quality of its pivots and bearings. A pivot is the boundary between the rotating part of the instrument and the fixed support. It must be as nearly cylindrical as possible; any deviation—a flat spot, a depression, a bulge—will cause the instrument to rock or wobble slightly as it rotates.

The tolerance is remarkable. For a 6-foot-diameter circle rotating on a pivot a few centimeters in diameter, a deviation of just $0.001$ inches ($0.025$ mm) in the pivot shape can introduce an arcsecond or more of error.

The pivot is supported by a bearing, typically a V-shaped groove in a brass or bronze block. The bearing's shape is critical: it must be truly V-shaped, and its apex must be sharp enough that the rounded pivot rests at a single line of contact.

Bearings suffer friction, and friction introduces a problem: as the instrument rotates, the friction force is not perfectly constant. At times the instrument wants to rotate freely; at other times it sticks slightly. This produces a systematic error in the timing of transits—the observer cannot record the exact moment when the star crosses the meridian if the instrument is momentarily sticky.

To address this problem, early meridian instruments used special bearing designs. One innovation was to use bearings machined with absolute precision and maintained with the finest lubricants. Another was to reduce friction by using ball or roller bearings, though these were difficult to manufacture to the required precision in the eighteenth and nineteenth centuries.

Flexure is a related problem: when the telescope, circles, and eyepiece are loaded with the weight of the optical train, the supporting structure may sag slightly. The east-west axis may flex, tilting slightly. The circles may distort. These changes are small—a few arcseconds at most—but they are systematic and must be corrected for.

\section{The Level: Detecting Departure from Horizontal}

The level is a simple but crucial instrument. A sealed glass tube contains a liquid (mercury or alcohol) and an air bubble. When the tube is horizontal, the bubble settles to the center. When tilted, the bubble moves off-center. The tube is divided into equal divisions, so the observer can determine the degree of tilt.

For a meridian instrument, the level is placed on the east-west axis. If the axis is not horizontal, the level's bubble will be off-center. The observer can read how far the axis has tilted and apply a correction to the observed altitudes.

The sensitivity of a level is characterized by its "division value"—the angle (in arcseconds) corresponding to one division of the bubble's travel. A high-precision level might have a division value of 0.1 arcseconds; thus moving the bubble by one division means the axis has tilted by 0.1 arcseconds.

A related instrument is the striding level—a level mounted on a frame that can straddle the instrument's pivots. The striding level is used to measure directly the heights of the two pivots, allowing one to check that they are at the same height and thus that the axis is horizontal. The striding level is typically used at intervals to verify the instrument's level and to detect any changes over time.

\section{The Collimator: Producing an Artificial Star}

A collimator consists of a small objective lens (a few inches in diameter) focused on a small point source of light (a candle or, later, an electric light) positioned at the focal point of the objective. Light from this source, diverging slightly, is refracted by the objective into a parallel beam—light rays that appear to come from an infinitely distant source.

The collimator is positioned at a fixed distance from the transit circle—typically a hundred feet or more—and carefully aligned north-south. An observer using the transit circle, focusing the telescope on the collimator's parallel beam, sees the light source appear to be infinitely far away, on the meridian, at a particular altitude (the altitude depending on the collimator's height above the transit circle's axis).

By recording the collimator's apparent position as seen through the transit circle, the observer can detect any change in the instrument's collimation. If the collimator's position drifts over time, it indicates the instrument is shifting.

The collimator's effectiveness depends on its stability. If the collimator itself moves—due to thermal expansion, wind, or seismic activity—it will introduce errors into the collimation check. For this reason, carefully designed observatories position the collimator in a dedicated enclosure far from sources of vibration.

\section{Azimuth Determination: Orienting the Instrument}

Even with the transit circle carefully leveled and collimated, there remains a critical task: determining the instrument's azimuth—confirming that it is oriented exactly north-south.

The ideal method uses observations of stars at different hour angles. By observing a star as it approaches the meridian, at the meridian, and as it leaves the meridian, the observer can determine the moment of true transit (when the hour angle is zero) by fitting the observed times to a mathematical model. Any east-west misalignment of the instrument will cause the transit time to shift, and this shift can be measured and corrected.

A simpler method uses the Sun. At solar noon (when the Sun reaches its maximum altitude), the Sun is on the meridian. By observing the Sun and noting when it reaches maximum altitude, one can verify that the instrument points south.

Still another method uses Polaris, the North Star. Polaris orbits the north celestial pole with a period of roughly 24 hours (though the orbit is not perfectly circular). By observing Polaris at different times of night and measuring its hour angle, one can determine the direction to the north celestial pole and thus verify the instrument's north-south orientation.

\section{The Nadir Observation: A Horizontal Reference}

The nadir is the point directly below the observer, on the horizon. Unlike the zenith, which is directly overhead, the nadir cannot be observed directly through a telescope pointed upward. However, one can observe the nadir's reflection in a horizontal mirror.

A mercury pool serves as such a mirror. Mercury, being liquid, naturally forms a horizontal surface (by gravity). By positioning a telescope to look downward at the mercury surface and then observing a star's reflection in that surface, one can determine the star's zenith distance—the angle from the zenith.

The reflected light path is vertical when the telescope is pointed at the horizon. This provides an independent reference for measuring altitude, free of the instrument's mechanical errors. By observing both a star directly (through the telescope pointed upward) and its reflection in mercury (through the telescope pointed downward), one can determine the star's true altitude and, by comparing the two measurements, detect systematic errors in the instrument.

The mercury pool observation was used at Greenwich and other major observatories as a consistency check on the transit circle data.

\section{The Photographic Zenith Tube}

In the twentieth century, photographic methods began to supplement visual observation. The photographic zenith tube is a telescope pointed vertically at the zenith, with a photographic plate at its focal plane. Stars passing through the zenith are photographed automatically; the clock time of exposure records the transit timing.

The advantage of the photographic method is that it removes the observer's reaction time from the measurement. The disadvantage is that photographic plates require calibration, and the image of a star on a plate has finite size, introducing some ambiguity about its exact position.

Despite these limitations, photographic techniques became increasingly important in the twentieth century, eventually dominating astrometric observations. The photographic zenith tube represented a transition from classical observing methods to modern automated instrumentation.

\section{The Danjon Astrolabe}

A later development, the Danjon astrolabe (1955), refined the mercury pool technique. The astrolabe uses a mercury surface to create a horizontal reference, and a special optical system to measure stellar positions. The instrument is small, portable, and sufficiently precise for many applications. While it did not replace the traditional transit circle, it became a useful tool for establishing or verifying the positions of reference stars.

\section{Worked Example: Complete Meridian Circle Observation Reduction}

Suppose an observer at Greenwich on the evening of 15 March 1850 observes the star Vega (right ascension $\approx 18^{\mathrm{h}}36^{\mathrm{m}}56^{\mathrm{s}}$, declination $+38°47'$) using the transit circle.

At the moment Vega crosses the meridian, the clock reads $18^{\mathrm{h}}36^{\mathrm{m}}42^{\mathrm{s}}$ mean solar time. The observer's reaction time adds a systematic delay; from previous collimation checks, this is known to be $+0.3^{\mathrm{s}}$. The clock is known to be $+1.2^{\mathrm{s}}$ fast. Correcting:

Transit time (corrected) $= 18^{\mathrm{h}}36^{\mathrm{m}}42^{\mathrm{s}} - 0.3^{\mathrm{s}} - 1.2^{\mathrm{s}} = 18^{\mathrm{h}}36^{\mathrm{m}}40.5^{\mathrm{s}}$

Converting to sidereal time: $18^{\mathrm{h}}36^{\mathrm{m}}40.5^{\mathrm{s}} \times 1.00273791 = 18^{\mathrm{h}}37^{\mathrm{m}}04.3^{\mathrm{s}}$ (sidereal)

This is the local sidereal time at transit. The right ascension thus determined is $18^{\mathrm{h}}37^{\mathrm{m}}04.3^{\mathrm{s}}$.

Comparing to the cataloged value: $18^{\mathrm{h}}36^{\mathrm{m}}56^{\mathrm{s}}$ (Greenwich Star Catalog 1850), the difference is $+8.3^{\mathrm{s}}$, or approximately $2.1'$ of arc. This residual must be investigated: Is it an error in the catalog, an error in the observation, or an indication of stellar proper motion?

For the declination: The level on the east-west axis shows the axis is tilted $0.2$ divisions, corresponding to $+0.02°$. The observed altitude at transit is $51°43.2'$. The declination is computed as:

$\delta = 90° - \text{zenith distance} = 90° - (\text{altitude correction})$

The computed declination is $\delta = 38°47.1'$, which agrees with the catalog value to within 0.1 arcminute. The observation is consistent.

\section{Precision and Systematic Error}

The best meridian instruments of the nineteenth century—the Airy transit circle at Greenwich, the Repsold circle at other major observatories—achieved precisions of $\pm 0.3$ arcseconds for well-observed, well-reduced transit observations. This precision was limited not by the instrument's mechanical capability but by atmospheric refraction variations, collimation drift, and the observer's reaction time.

The catalog of star positions derived from these observations—the most precise astrometric data available in the nineteenth century—formed the foundation for modern stellar astronomy. With these positions, stellar proper motion could be measured, parallax searches could be conducted, and the structure of the Milky Way could be mapped.

\begin{table}[h]
\centering
\begin{tabularx}{\textwidth}{XXXX}
\toprule
\textbf{Instrument} & \textbf{Era} & \textbf{Key Features} & \textbf{Achievable Precision} \\
\midrule
Flamsteed's Mural Arc & 1689–1712 & 140° arc, hand-divided, 7 ft radius & $\pm 10-15$ arcsec \\
Bradley's Zenith Sector & 1725–1750 & Vertical telescope, star tracking & $\pm 1-2$ arcsec \\
Ramsden Vertical Circle & 1790s & Full circle, mechanical divisions & $\pm 2-3$ arcsec \\
Airy Transit Circle & 1851–1900 & 6 ft circles, 6 microscopes, 8 inch refractor & $\pm 0.3-0.5$ arcsec \\
Repsold Circle & 1870–1950 & Similar design, German construction & $\pm 0.4-0.5$ arcsec \\
Modern CCD Astrometry & 1980–present & Electronic detection, computer reduction & $\pm 0.001$ arcsec \\
\bottomrule
\end{tabularx}
\caption{Evolution of meridian instruments from Flamsteed to the modern era, showing the progression from mechanical to electronic precision.}
\end{table}

\section{The Transition to Astrographic Methods}

By the end of the nineteenth century, visual observation with meridian circles was being supplemented and eventually replaced by photographic methods. The Carte du Ciel (Photographic Star Chart) project, initiated in 1887, aimed to photograph the entire sky systematically. These photographic plates could reach fainter stars than visual observation and were free of the observer's personal equation.

Photographic astrometry requires careful calibration: each photographic plate must be measured against known reference stars, and the image positions converted back to celestial coordinates. The precision achievable depends on the photographic plate's quality, the telescope's optics, and the sophistication of the measurement and reduction procedures.

Despite the transition to photographic and later to electronic methods, the fundamental principle established by the meridian instruments remained: precision requires constraining the degrees of freedom, removing sources of systematic error, and maintaining meticulous attention to the instrument's calibration and stability. The meridian instrument was not the endpoint of astrometry but rather the foundation upon which all subsequent methods were built.
