\chapter{Building the Historia Coelestis Britannica}
\label{ch:historia-coelestis}

\section{Flames and Theft}
\label{sec:flames-and-theft}

It was October 1712 when Flamsteed learned what had been done to him. His hands trembled as he held the letter. Isaac Newton, together with Edmond Halley, had taken his unpublished observations---decades of work, incomplete, raw, still containing errors he had not yet corrected---and arranged for their publication without his permission. They had done it while he was still living, still observing, still working to perfect what he had gathered. The pretext was scholarly urgency. The truth was impatience.

Flamsteed was then seventy-four years old. He had been observing steadily since 1676. His health was failing. He knew this would be his only lifetime to build the catalog. Now, without his consent, without his final revisions, flawed data bearing his name were entering the permanent record of astronomy. He understood at once: if these positions became standard, if other astronomers built upon them, the errors would propagate. One corrupt calculation multiplies through every dependent work.

He purchased three hundred of the four hundred printed copies that had been distributed. He burned them. All of them. The gesture was as clear as it was futile. The damage was done. But the act itself became legendary: Flamsteed, in his fury, destroying the unauthorized publication of his life's labor.

The true Historia Coelestis Britannica would come later, in 1725, published finally with his final revisions, at his instruction. It contained three thousand stars, each position distilled from perhaps two hundred individual observations, each observation reduced through careful corrections for instrument drift, refraction variation, and systematic errors. This was the work that mattered---not the pirated edition, but the deliberate culmination.

\section{The Scale of the Campaign}
\label{sec:scale-of-campaign}

From 1676 to 1719, Flamsteed observed. Forty-three years. Nearly fifty thousand individual instrument readings passed through his hands. Each reading consisted of two components: an altitude (from the mural arc) and a time (from Tompion's clock). Each pair of numbers, combined with knowledge of the observer's position and the date, could be transformed into a celestial coordinate: right ascension and declination.

But a single observation is never sufficient. Thermal expansion of the brass arc changed its scale. The pendulum of the clock drifted. Atmospheric refraction varied. Personal reaction time was never zero. To achieve final positions accurate to ten or fifteen arc-seconds required taking many observations of the same star, separated in time and under varying conditions, then subjecting them to careful statistical treatment. The best stars in the final catalog were observed two hundred or three hundred times. Some were observed nearly every clear night for decades.

The final catalog contained three thousand stars. But the observational base was enormous: between forty and fifty thousand individual measurements. A single observer, aided by assistants, could record no more than perhaps ten bright stars per clear night. With weather, mechanical problems, and illness, clear nights at Greenwich averaged perhaps a hundred per year. Over forty years, fifty thousand observations was the natural result of such discipline.

\section{Reducing an Observation}
\label{sec:reducing-observation}

The journey from raw instrument reading to catalog position followed a well-defined path. Consider a single observation. Flamsteed has measured the altitude of the star Sirius at the moment it crossed the meridian. His reading from the arc was altitude 51 degrees, 21 arcminutes, 42 arcseconds. His clock read 18 hours, 34 minutes, 52 seconds.

The first step is to correct the clock time. Tompion's clock was accurate, but not perfect. It gained or lost a few seconds per day. Flamsteed calibrated his clock by observing stars of known position and checking whether the recorded time was consistent. These calibrations allowed him to determine the clock's rate, and thus apply a correction to each observation. Suppose the correction on this night is $+3$ seconds---the clock was running 3 seconds fast. The corrected time becomes 18 hours, 34 minutes, 49 seconds.

Next, this solar time must be converted to sidereal time. The relationship is:

\[
\text{Sidereal Time} = \text{Solar Time} + \text{Equation of Time} + 12\text{h}
\]

The equation of time accounts for the fact that the solar day (relative to the Sun) is not constant: it varies throughout the year between about $-14$ and $+16$ minutes. Tables for this were available. Suppose the equation of time on this date is $-5$ minutes. Then:

\[
\text{Sidereal Time} = 18\text{h}34\text{m}49\text{s} + (-5\text{m}) + 12\text{h} = 30\text{h}29\text{m}49\text{s}
\]

Since there are only 24 hours in a day, we subtract 24 hours: the sidereal time is 6 hours, 29 minutes, 49 seconds. This is the right ascension of any star on the meridian at that moment.

But we also need the declination. The altitude of $51\degree 21' 42''$ must be corrected for refraction. Near the meridian, refraction is small and well-behaved. Using tables based on the atmospheric conditions and assuming a standard temperature and pressure, the refraction correction is roughly $1$ arcminute. So the true altitude is $51\degree 20' 42''$. The zenith distance is:

\[
z = 90\degree - h = 90\degree - 51\degree 20' 42'' = 38\degree 39' 18''
\]

Now, Flamsteed's latitude at Greenwich is approximately $51\degree 29'$. The declination of the star is:

\[
\delta = \phi - z = 51\degree 29' - 38\degree 39' 18'' = 12\degree 49' 42''
\]

This is the declination. Combined with the right ascension, we have a catalog position:
\[
\text{RA} = 6\text{h}29\text{m}49\text{s}, \quad \delta = -16\degree 42' 15''
\]

(The declination is given a negative sign if the star is south of the celestial equator. Sirius is indeed a southern star.)

But this is a single observation, on a single night, under specific conditions. The real procedure involved averaging many such observations, weighting them by their quality, and then checking the residuals---the difference between the final position and each individual observation. If one observation disagreed badly with the others, it might be discarded as corrupted by some transient error. The final position is the weighted mean of the good observations, and the scatter of the individual observations provided an estimate of the uncertainty.

\section{Celestial Coordinates}
\label{sec:celestial-coordinates}

The work Flamsteed was doing required fluency in coordinate systems. Astronomers work with several. The most natural for observational astronomy is the \emph{horizon system}: altitude and azimuth, measured from the observer's horizon. But this system is local. A star's altitude and azimuth are different in London than in Cairo, even at the same instant. This makes it useless for comparison.

Instead, astronomers use the \emph{equatorial system}: right ascension and declination. Right ascension is like longitude on the celestial sphere, measured eastward from the vernal equinox (the point where the Sun crosses the celestial equator in spring). Declination is like latitude, measured north or south from the celestial equator. This system is absolute: it applies to any observer anywhere on Earth (as long as one accounts for small corrections like aberration and nutation, which Flamsteed's contemporary knowledge could not quantify). Two observers on opposite sides of Earth can now compare observations.

But the older tradition of Western astronomy used the \emph{ecliptic system}: measuring positions relative to the plane of the Earth's orbit around the Sun. This system is natural for planets, which orbit in or near this plane. An older star catalog, or a table of planetary positions, might be given in ecliptic coordinates: ecliptic longitude and ecliptic latitude.

Converting between these systems requires spherical trigonometry. Consider a star with equatorial coordinates (RA, Dec). We want its ecliptic coordinates (ecliptic longitude, ecliptic latitude). The ecliptic plane is tilted $23\degree 26'$ from the equatorial plane (this tilt is the obliquity of the ecliptic, denoted $\varepsilon$). Using the spherical law of cosines, we can write:

\[
\cos(b) = \sin(\delta) \cos(\varepsilon) - \cos(\delta) \sin(\varepsilon) \sin(\alpha)
\]

where $\alpha$ is right ascension, $\delta$ is declination, $b$ is ecliptic latitude, and $\varepsilon$ is the obliquity. Rearranging gives the ecliptic latitude directly. Similarly, we can derive the ecliptic longitude. The algebra is straightforward but tedious.

Flamsteed understood these transformations implicitly. He needed them to compare his catalog with older ones, and to understand how a star's position would shift over time due to precession---the slow wobble of Earth's axis, which causes the coordinates of stars to drift very gradually.

\section{Precession}
\label{sec:precession-correction}

Tycho Brahe had compiled a catalog of stellar positions around 1600. When Flamsteed compared his 1700 positions with Tycho's, he found systematic shifts. A star that Tycho had recorded at right ascension 7 hours was now at 7 hours and 6 minutes. The shift was not random. It was systematic: all stars showed a drift in approximately the same direction.

The explanation is precession. Earth's axis is not fixed in space. It wobbles, like a spinning top that is not perfectly spun. The wobble has a period of about 26,000 years. Over a century (the century separating Tycho and Flamsteed), the vernal equinox---the zero point from which right ascension is measured---drifts westward by approximately one degree. This means all right ascensions increase by the same amount.

To compare catalogs separated in time, one must correct for precession. The transformation is not simple. The precession drift is not quite uniform; it includes small periodic terms (nutation). Modern formula are complex. But Flamsteed, in the early 18th century, had access to tables computed by Jean Picard and others, which allowed him to correct observed positions to a standard epoch and compare them reliably.

The precession constant (the amount the vernal equinox drifts per year) was not accurately known. Tycho had placed it at about $47$ arcseconds per year. By Flamsteed's time, it was estimated at about $50$ arcseconds per year, somewhat more accurate. (The true value is close to $50.3$ arcseconds per year). The uncertainty in this constant was a significant source of error when comparing historical catalogs, but within a single epoch, it was stable enough.

\section{Data Quality and Conflict}
\label{sec:data-quality}

The publication of the pirated 1712 edition by Newton and Halley was theft, but it was also a scientific crisis. Newton needed accurate star positions for his lunar theory. He had pressured Flamsteed for years to release data. Flamsteed had refused, believing the data were still too rough, still contaminated by errors he hadn't fully understood.

Newton, then over seventy and focused on finishing his work, had been impatient. He thought Flamsteed was being obstructionist. In fact, Flamsteed was right. The 1712 edition contained positions that differed from the final 1725 catalog by 20 to 30 arcseconds for many stars. This difference is small by the standard of naked-eye observation a century earlier, but it is large compared to what Flamsteed ultimately achieved, and it is large enough to introduce significant errors into lunar calculations.

The conflict was not personal animosity alone, though Newton and Flamsteed disliked each other. It was a disagreement about readiness. Newton wanted publishable data; Flamsteed wanted to publish only polished data. Newton believed science progresses through publication and community refinement; Flamsteed believed science progresses through careful completion before release. Both positions had merit. In this case, Flamsteed's caution was vindicated. The pirated edition has been largely forgotten. The 1725 Historia is the catalog that shaped astronomy for the next century.

\section{The Published Catalog}
\label{sec-the-published-catalog}

The Historia Coelestis Britannica finally published in 1725 contained three thousand stars. Each star had:
\begin{itemize}
\item A designation (eventually called the Flamsteed number: 1 Andromedae, 2 Andromedae, etc.)
\item Right ascension to the nearest second of time
\item Declination to the nearest arcminute
\item Magnitude (brightness estimate on Tycho's scale)
\item Position angle and distance of any visible companion star
\end{itemize}

The astrometric precision achieved was remarkable. For the brightest stars, typically observed dozens or hundreds of times, the internal consistency of the positions (comparing multiple observations after all corrections) was of order 10 to 15 arcseconds in both coordinates. This was a tenfold improvement over Tycho Brahe's positions, and it remained the standard reference for stellar astrometry for nearly a century, until the 1830s when fixed-telescope micrometers finally surpassed Flamsteed's accuracy.

\begin{table}[htbp]
  \centering
  \caption{Comparison of stellar positions: Tycho Brahe (1600) vs. Flamsteed (1725), corrected for precession to epoch 1700.}
  \label{tab:tycho-vs-flamsteed}
  \small
  \begin{tabular}{lcccc}
    \toprule
    \textsc{Star} & \textsc{Tycho RA} & \textsc{Flamsteed RA} & \textsc{RA Difference} & \textsc{Improvement} \\
    \midrule
    $\alpha$ Andromedae & $0\text{h}0\text{m}11\text{s}$ & $0\text{h}1\text{m}14\text{s}$ & $63\text{s}$ (945 arcsec) & (precession) \\
    Aldebarau ($\alpha$ Tauri) & $4\text{h}26\text{m}27\text{s}$ & $4\text{h}33\text{m}22\text{s}$ & $55\text{s}$ (825 arcsec) & (precession) \\
    Altair ($\alpha$ Aquilae) & $19\text{h}44\text{m}4\text{s}$ & $19\text{h}47\text{m}3\text{s}$ & $59\text{s}$ (885 arcsec) & (precession) \\
    \midrule
    \multicolumn{5}{l}{Note: Large RA differences are due to precession (125 years). Flamsteed's positions}\\
    \multicolumn{5}{l}{are reproducible to $\pm 10''$ to $\pm 15''$, compared to Tycho's $\pm 2'$ to $\pm 3'$.}\\
    \bottomrule
  \end{tabular}
\end{table}

The true measure of the catalog's value was not how it compared to Tycho in 1700, but how it enabled the next century of discovery. Flamsteed's positions provided the reference frame for Bradley's discovery of aberration and nutation. They provided the fixed stars against which the planets' motions could be precisely charted. They showed, definitively, that the stars were not fixed---small proper motions could be detected for some stars. These discoveries, built directly on Flamsteed's foundation, transformed astronomy from a static catalog of positions into a dynamic science of motion and change.

\section{The Labor}
\label{sec:the-labor}

Flamsteed lived to 1719 and saw his work published. He died before the Historia was fully distributed. The catalog was already becoming standard. But the effort it cost was immense: forty-three years of nearly every clear night, calibrating instruments, taking observations, performing calculations, mentoring assistants, and enduring the institutional struggles---Newton's pressure, Halley's support-then-betrayal, the financial precarity of the Royal Society, the need to fight for his own data's integrity.

This is often invisible in histories of science. We count discoveries, derive equations, trace ideas. We rarely quantify the human labor that lies beneath each datum: the cold nights, the manual calculation, the repeated checking, the refusal to publish before understanding. Flamsteed's catalog is perhaps the last great act of naked-eye astronomy, and its existence depended utterly on this unflinching commitment to completeness. The next chapter turns to the instruments that would finally enable greater precision---to the mechanical clock and its limitations on the ocean. \cref{ch:clock-problem-1} continues the story of how timekeeping, not stellar positions, became the constraint.
