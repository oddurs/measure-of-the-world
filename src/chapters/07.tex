\chapter{The Longitude Act and Its Incentives}
\label{ch:longitude-act}

In the spring of 1714, the House of Commons was seized by an unusual urgency. Day after day, witnesses appeared to testify about maritime disasters caused by navigational uncertainty. Sea captains described near-misses and disasters: ships landing miles from their reckoned positions, convoys scattered in fog, valuable cargoes lost. William Whiston and Humphry Ditton had submitted a proposal: use signal fires and rockets visible from heights of forty miles to communicate time across the ocean. It was impractical and absurd, yet even this proposal gained serious consideration. The political pressure was intense. In the summer, with remarkable speed and unusually broad support, Parliament passed An Act for providing a Public Reward for such Person or Persons as shall discover the Longitude at Sea (12 Anne, c. 15). The Act offered the enormous sum of £20,000—equivalent to roughly twenty years' wages for a skilled tradesman—to anyone who could devise a method to determine longitude at sea to within 30 nautical miles. Smaller prizes of £10,000 and £15,000 were offered for less stringent requirements. This was the Longitude Act of 1714, and it inaugurated a new era in which the British state explicitly offered incentives for technological innovation.

\section{The Precision Thresholds and Their Physical Meaning}

The Act established three levels of achievement, each with corresponding rewards:

\begin{itemize}
  \item \textsc{£10,000} for accuracy to 60 nautical miles at the end of a 6-week voyage
  \item \textsc{£15,000} for accuracy to 40 nautical miles at the end of a 6-week voyage
  \item \textsc{£20,000} for accuracy to 30 nautical miles at the end of a 6-week voyage
\end{itemize}

These distance thresholds correspond to angular and temporal precisions. A nautical mile is defined as 1/60th of a degree of latitude, or equivalently 1/21,600 of the circumference of the Earth. Converting to longitude:

At the equator, where a degree of longitude spans exactly 60 nautical miles:
\[
60 \text{ nm} = 1^{\circ} = 4 \text{ minutes of time (at equator)}
\]

At higher latitudes $\phi$, the conversion is:
\[
\text{distance (nm)} = 60 \cos(\phi) \text{ nm per degree}
\]

Thus at 30 nautical miles:
\[
30 \text{ nm} = 0.5^{\circ} = 2 \text{ minutes of time (at equator)}
\]

For the lunar distance method (the astronomical approach), 30 nautical miles corresponds to requiring the Moon's position to be known to approximately 1 arcsecond precision. For the chronometer approach, 30 nautical miles at the equator corresponds to a clock error of 2 minutes of time accumulated over a 6-week voyage—an error rate of roughly 0.5 seconds per day.

These thresholds were not chosen arbitrarily. Parliamentary testimony had established that navigational errors beyond 60 nautical miles posed unacceptable risks to merchant vessels and naval squadrons. The 30-nautical-mile threshold (most worth pursuing, as it offered the £20,000 prize) represented the limits of what contemporary navigation techniques could reliably achieve. The Act thus crystallized into measurable objectives what had been vague aspirations.

\section{The Four Competing Approaches}

From the outset, multiple methods were proposed to the Board of Longitude. Four major approaches emerged, each grounded in different physics and each with adherents among scientists and practitioners.

\textsc{The Lunar Distance Method:} The astronomer's preferred solution. The Moon moves approximately 0.5 degrees per hour through the celestial sphere. At any given moment, the Moon's angular distance from reference stars (or from the Sun) is a function of time at Greenwich. If an accurate lunar position table were available, a navigator could observe the Moon-star distance with a sextant, consult the table to find what Greenwich time corresponds to that distance, and thus determine his longitude. The method is elegant, requiring only a sextant and a table—no moving parts vulnerable to wear or temperature change. Its drawback is computational burden: a complete calculation from raw sextant reading to longitude determination required 30 minutes of mental arithmetic using logarithms. Advocates included the astronomers of the Board, particularly Edmond Halley.

\textsc{Jupiter's Moons:} When Galileo discovered the four large moons of Jupiter orbiting the planet, he recognized them as a celestial clock: the moons undergo regular eclipses, and the times of these eclipses depend only on Jupiter's true position, not on the observer's location. An observer who could accurately predict when Jupiter's moons would be eclipsed could compare the predicted time (in Greenwich time) with the observed time (at his current location) to determine his longitude. The method requires a telescope and detailed tables of Jupiter's motion. Its advantage is that it avoids the computational burden of lunar distances. Its disadvantages are severe: Jupiter is faint and difficult to observe from a pitching ship; the telescopes needed were large and fragile; and the method works only in dark twilight, when Jupiter is visible but the sky is dark enough for precise timing. Few serious investigators pursued this method, though it remained on the board's agenda.

\textsc{Magnetic Variation:} The compass needle does not point exactly north; it deviates by an angle called magnetic variation or declination. Magnetic variation changes with location—in London it might be 12 degrees west of true north, while in Jamaica it might be 3 degrees east. If the variation at every location on Earth were known, a navigator could measure the local variation and look it up to determine his longitude. Henry Gellibrand and others had attempted to map magnetic variation, but the magnetic field proved to be non-uniform and changing with time, making the method unreliable. Despite its limitations, magnetic variation attracted attention because it required no equipment beyond a compass. Some proposals suggested a "magnetic clock" to exploit variation patterns.

\textsc{The Chronometer (Mechanical Clock):} A sufficiently accurate clock, compared against a reference clock set to Greenwich time before departure, could directly yield longitude: the time difference multiplied by 15 degrees per hour gives the longitude difference. The method is conceptually simple and requires no astronomy. Its disadvantages are severe: pendulum clocks fail at sea (as we have seen), temperature changes affect all mechanical systems, and building a clock accurate to seconds per day seemed nearly impossible. Yet some clockmakers, particularly John Harrison, believed the problem could be solved through mechanical ingenuity.

\section{Why Astronomers Dominated the Board}

The Board of Longitude, established by the 1714 Act, consisted of astronomers, naval officers, mathematicians, and royal officials. The Astronomer Royal (designated as a permanent member) held significant influence. The composition of the Board during its active decades skewed toward learned men—academics rather than practical craftsmen, astronomers rather than clockmakers.

This institutional bias had consequences. The lunar distance method, championed by astronomers, received the most sustained attention and resources. Maskelyne, who became Astronomer Royal in 1765, was a passionate advocate for the method and invested enormous effort in refining lunar tables and computing the Nautical Almanac. When John Harrison claimed success with his chronometers, the Board subjected his devices to skeptical scrutiny, imposing stringent tests and demanding repeated trials before awarding the prize.

\section{The Historiographical Debate}

For two centuries, historians portrayed this episode according to a clear narrative: the Board was dominated by astronomers who had vested interests in the lunar distance method and who obstructed Harrison's chronometer out of academic conservatism and professional jealousy. In this traditional view, Harrison was a lone genius, a self-taught craftsman whose mechanical intuition transcended academic obstacles. The Board was the villain—bureaucratic, narrow-minded, defensive of the astronomical establishment.

This interpretation, championed by popular histories and Dava Sobel's influential \emph{Longitude} (1995), contains truth but oversimplifies. \textcite{Sobel1995} emphasizes Harrison's forty-year struggle and the Board's repeated demands for additional trials. The chronometer is portrayed as obviously superior—faster, simpler, more reliable—yet the Board delayed awarding full recognition until political pressure (including an appeal to King George III) forced their hand.

But a revisionist interpretation, articulated by scholars such as \textcite{Howse1980} and \textcite{Andrewes1998}, offers greater nuance. The Board's skepticism had legitimate grounds. Harrison's devices were unique: no independent replication existed until the late 18th century. Chronometers were expensive to build and even Harrison could not guarantee that a copy built by another craftsman would maintain his level of accuracy. The lunar distance method, by contrast, was mathematically grounded and reproducible—any astronomer with proper tables could employ it. Furthermore, the method did not require every ship to carry an expensive precision instrument; the tables could be produced at the Royal Observatory and distributed to mariners.

The Board faced a genuine choice under uncertainty. Should they champion a method that was theoretically sound and proven to work on multiple instruments (the lunar distance method), or should they gamble on a technology that had worked brilliantly in one craftsman's hands but lacked independent verification? From the perspective of risk management and reproducibility, the Board's caution was not unreasonable.

\section{Resolution and Legacy}

Ultimately, both methods succeeded—and both ultimately proved limited. The lunar distance method was operationalized in Maskelyne's Nautical Almanac (first published in 1767) and taught to mariners for decades. At sea, skilled navigators using the method achieved accuracies on the order of 1–2 degrees—worse than the 30-nautical-mile standard, but sufficient for many purposes. The computational burden remained a barrier to widespread adoption.

Harrison's chronometers, once demonstrated to be reproducible (later builders including Pierre Le Roy and John Arnold produced comparable instruments), became the standard solution. By the early 19th century, chronometers—now produced by multiple makers and steadily improving—had become the primary navigation instrument for any ship that could afford them.

Reflecting on the episode, \textcite{Howse1980} argues that the Board's institutional structure actually served a valuable function: by demanding reproducibility, independent verification, and multiple trials, the Board ensured that the final solution was robust. A method adopted based on a single brilliant success might later prove to have been a fluke or dependent on unreproducible craftsmanship. By insisting on evidence, the Board (perhaps inadvertently) imposed standards that the modern scientific community would recognize as sound.

The Longitude Act itself had broader consequences. It established precedent for government investment in technological innovation. It showed that substantial prizes could mobilize talent and resources. It demonstrated that the path from scientific problem to practical solution was rarely straight—both the astronomical and mechanical approaches had merit, and wisdom lay in supporting multiple efforts rather than betting exclusively on one. In this sense, the Board's pluralism, which might have appeared as confusion or indecision, was actually a more sophisticated strategy than any single-method approach could have been.

\section{Timeline of Key Board Decisions}

The following timeline summarizes major decisions by the Board of Longitude from its creation through the resolution of Harrison's claims:

\begin{itemize}
  \item \textsc{1714 (June):} Longitude Act passes with overwhelming parliamentary support.
  \item \textsc{1714--1720:} Initial proposals flood in; Board establishes criteria and begins evaluation.
  \item \textsc{1730:} Harrison completes H1 (first sea clock); Board provides modest funding.
  \item \textsc{1736:} H1 tested at sea; performance promising but not meeting full Act requirements.
  \item \textsc{1765:} Maskelyne becomes Astronomer Royal; lunar distance method gains official backing.
  \item \textsc{1767:} \emph{Nautical Almanac} first published; lunar distance method now accessible to mariners.
  \item \textsc{1761--1762:} H4 (Harrison's watch) tested in Jamaica; extraordinary accuracy (5.1 seconds error).
  \item \textsc{1772:} H5 tested before King George III; error of 4.5 seconds over ten weeks.
  \item \textsc{1773:} Board finally awards Harrison £4,400 as a "gift" in recognition of long service; Parliament subsequently awards additional £8,750, bringing total remuneration near the prize level.
  \item \textsc{1828:} Board of Longitude formally dissolved; prizes system discontinued.
\end{itemize}

The timeline reveals a process of negotiation, partial recognition, and eventual acceptance. The delays were neither simple obstruction nor oversight; they reflected genuine institutional uncertainty about which method would prove most practical for widespread use.

\section{Thematic Lessons}

The story of the Longitude Act illuminates several principles about how societies mobilize solutions to difficult problems:

\textsc{Incentive structures matter.} The Act offered large rewards, but only for demonstrated success. This attracted talent and resources but also ensured that claims had to be substantiated. The structure of the prize—tiered by precision level—allowed partial credit and encouraged incremental progress.

\textsc{Multiple approaches have value.} The Board's willingness to support both the astronomical method (lunar distances) and mechanical approaches (chronometers) meant that if one method stalled, another could advance. In retrospect, this pluralism appears wise; at the time, it looked like indecision.

\textsc{Verification is difficult under uncertainty.} The Board faced a genuine problem: how to verify that a claimed solution actually works? For the lunar distance method, grounded in celestial mechanics and astronomical theory, verification could rely on mathematical reasoning. For Harrison's chronometers, verification required repeated trials and independent replication—a much more demanding standard. The Board's insistence on verification, though costly in time, ensured that the final solution was robust.

\section{Looking Forward}

With the Longitude Act established and multiple methods under active development, the stage was set for the detailed investigation of individual approaches. The lunar distance method, despite its computational burden, would be refined by successive astronomers and would serve navigators for decades. Harrison's pursuit of the mechanical solution would consume forty years and produce devices of extraordinary sophistication. And yet both approaches, though they succeeded in their own terms, would eventually give way to technologies that their 18th-century inventors could not have foreseen. This is the arc we now trace: the triumph and limitation of each method, and the ultimate resolution through entirely different means. We turn first, in \cref{ch:lunar-distance}, to the astronomical approach in its full technical glory.

