\chapter{The Founding of the Royal Observatory}
\label{ch:founding-observatory}

On March 4, 1675, King Charles II\index{Charles II} appointed a 28-year-old self-taught astronomer from Derbyshire to an unprecedented position: ``Astronomical Observator to His Majesty.'' The title was invented for the occasion, the salary absurdly modest, the expectations revolutionary. John Flamsteed\index{Flamsteed, John}\index{Royal Observatory Greenwich!founding} stood at the threshold of a task that would consume four decades of his life: to map the British heavens with a precision no one had achieved before, and in doing so, to provide the celestial reference frame upon which the solution to longitude would ultimately rest.\footnote{Flamsteed's appointment warrant: Baily, \emph{Account of the Revd. John Flamsteed} (1835), p. 7. The position was officially created by Royal Warrant, 22 June 1675. Flamsteed did not receive the formal appointment document until that date, though he had been working informally since March.}

\section{The Political Context}
\label{sec:political-context}

The founding of a royal observatory in 1675 was an act of strategic ambition. England's maritime power depended on navigation; navigation depended on accurate astronomy; accurate astronomy required stable institutions and patient investment. Yet such institutions were rare, and the investment was unprecedented.\footnote{The Observatoire de Paris had been founded eight years earlier, in 1667, under Louis XIV's patronage. It was the model for Greenwich, but also the provocation: England could not afford to lag in astronomical prestige.}

Jonas Moore,\index{Moore, Jonas} Surveyor-General of the Ordnance and a mathematician of considerable repute, had advocated for a national observatory for years.\footnote{Moore's correspondence on the subject: Royal Society Archives, MS 66A-66B. Moore was not himself a skilled observer but understood the strategic importance of celestial reference frames for military engineering and navigation.} The longitude problem\index{longitude!problem} had crystallized the argument. Ships were lost at sea not because the Navy lacked courage or skill, but because no one knew where they were. The gaps in the celestial map were gaps in the foundations of empire.

Charles II, advised by Moore and the President of the Royal Society, approved the establishment. The warrant, issued on 22 June 1675, was terse: the King had ``resolved to establish an observatory'' and to employ an ``astronomical observator'' charged with ``rectifying the tables of the motions of the heavens.'' The term was deliberate. This was not a natural philosopher's cabinet of curiosities. It was an instrument of state, focused on a single, practical aim: accurate star positions.\footnote{The Royal Warrant establishing the Observatory is reproduced in \textcite{Baily1835}, p. 31. The phrase ``rectifying the tables of the motions of the heavens'' appears in the original document.}

\section{The Site and the Architect}
\label{sec:site-architect}

Greenwich was chosen for sound reasons. It lay downriver from London, far enough to escape the smoke and haze of the city, close enough to remain connected to the institutions of power. Crown land was available: the site of the old medieval castle in Greenwich Park, with a clear southern exposure and sufficient elevation to see the horizon.\footnote{Historical accounts of Greenwich's selection are given in \textcite{Howse1980} and \textcite{Willmoth1992}. The Park itself was a royal hunting ground; the Observatory sat at the edge of the park, commanding a view of the Thames valley.}

Christopher Wren,\index{Wren, Christopher} Surveyor of the King's Works, was commissioned to design the building. Wren's brief was paradoxical: he was given a budget of \pounds500 and asked to produce a structure that would house precision instruments, living quarters for an observer, and an assistant's rooms.\footnote{The budget was specified in the warrant. \textcite{Baily1835}, p. 32, notes that Wren complained the sum was ``far too little'' for the intended purpose. Contemporary documents suggest actual costs ran to approximately \pounds520, with Moore supplying additional funds.} 

The result was Flamsteed House,\index{Flamsteed House} completed in 1676. It was a compact structure of brick and stone, four stories tall, with a distinctive octagonal turret rising from the northwest corner. The Octagon Room,\index{Octagon Room} on the top floor, was designed to house long-focus telescopes and the great mural arc that would become Flamsteed's principal instrument. Large windows provided sight lines to the south, east, and west. The design was elegant and efficient, though perhaps too decorative for pure observational work. Wren had created not a utilitarian shed, but an architectural statement: the physical embodiment of royal patronage directed toward the heavens.\footnote{Architectural drawings of Flamsteed House are preserved in the RIBA (Royal Institute of British Architects) collections. Technical descriptions are given in \textcite{Howse1980}, Chapter 2, and \textcite{WilmothFlamsteeds2002}.}

\section{The Instruments and the Budget}
\label{sec:instruments-budget}

When Flamsteed arrived at Greenwich in the summer of 1675, the Observatory was incomplete. The building was still under construction. The instrument suite was meager.

What Flamsteed inherited came largely from Jonas Moore's generosity. Moore had commissioned two exceptional pendulum clocks from Thomas Tompion,\index{Tompion, Thomas}\index{clocks!pendulum} the finest horologist in England. These were masterpieces: clocks accurate to within ten or fifteen seconds per day, mounted on the ground floor with long pendulums hanging through carefully designed voids to achieve isolation from vibration.\footnote{Tompion's clocks are documented in \textcite{Betts1978}. One of the clocks survives at the National Maritime Museum, Greenwich. Specifications: a 13-foot pendulum regulated each clock's swing; the mechanism incorporated Harrison's grasshopper escapement principles, though Harrison's work came later.} With these clocks, Flamsteed could time observations to a precision previously unimaginable.

He also inherited a sextant of seven-foot radius, an instrument capable of measuring angles between celestial objects but ill-suited to the work of building a star catalog. The Observatory possessed a few smaller instruments, but none of the large fixed instruments necessary for systematic observation.

The remaining instruments, Flamsteed would have to build himself. And here the constraint became apparent. His salary was \pounds100 per year. From this, he was expected to pay his assistant and to finance the acquisition and construction of instruments.\footnote{\textcite{Baily1835}, p. 68, reproduces Flamsteed's correspondence with the Board expressing his difficulty in meeting expenses. His assistant, Abraham Sharp (who would later become the second-best observational astronomer in Britain), was paid \pounds20 per year---leaving Flamsteed less than \pounds80 for his own subsistence.} 

This parsimony shaped the Observatory's early years. Flamsteed could not commission instruments from professional makers as Tycho Brahe had done. Instead, he designed his own, working with craftsmen to construct them as economically as possible. His great achievement was not to invent new principles, but to work within stringent constraints to produce instruments of unprecedented precision.

\section{The Mural Arc: Flamsteed's Principal Instrument}
\label{sec:mural-arc}

Between 1679 and 1691, Flamsteed designed and constructed his greatest instrument: the mural arc.\index{mural arc}\index{instruments!mural arc} This was a quarter-circle of radius nearly seven feet, graduated to quarter-minute divisions and mounted permanently in the meridian plane---the vertical plane running due north-south through the zenith (the point directly overhead).\footnote{The mural arc was not original to Flamsteed; mural quadrants existed in antiquity. But Flamsteed's refinements to the design, and the precision he achieved, were unprecedented. See \textcite{Howse1980}, Chapter 3, for a detailed technical description.}

The principle was elegant. By mounting the arc in a fixed plane oriented north-south, Flamsteed eliminated errors from flexure (the bending and warping that occurs under temperature changes and vibration) and misalignment that plagued portable instruments. As a star crossed the meridian (the north-south line), he could measure both its right ascension (the celestial coordinate measured along the celestial equator, determined from the time of transit across the meridian, which he recorded on Tompion's clocks) and its declination (from the arc's graduated scale). The geometry was pure: no moving parts except the eyepiece itself.

With Abraham Sharp\index{Sharp, Abraham} as his assistant, Flamsteed would observe a star as it approached the meridian, call the exact time to the second from the clock, note the precise moment of transit, and then read the altitude (the angular height above the horizon) from the graduated scale. The routine was ritualistic: clear nights, cold hands, precise procedure, repetition. Over decades, this discipline would yield the most accurate star catalog the world had yet seen.

The construction was laborious. The arc itself had to be carefully graduated. Flamsteed and Sharp spent months dividing the scale, engraving the marks, checking and rechecking for uniformity. Every division had to be accurate to a small fraction of an arc-minute. The task required extraordinary care and considerable expense---expense that Moore and other patrons helped defray, since the modest salary could not cover it.\footnote{\textcite{Baily1835}, p. 102--110, reproduces Flamsteed's detailed account of the mural arc's construction, including the costs incurred and the difficulties encountered.}

\section{The Catalog Emerges}
\label{sec:catalog-emerges}

By the 1680s, systematic observation was underway. Flamsteed and Sharp worked methodically through the bright stars, following a systematic right ascension order. They observed circumpolar stars (near the north celestial pole), then stars at lower declinations, gradually building a comprehensive and precise catalog.

The work was slow, demanding, and often frustrating. Flamsteed's perfectionism meant that many observations were repeated. He distrusted his first measurements and insisted on verification through multiple observations, sometimes separated by years. This caution cost time, but it produced data of unparalleled reliability.

By 1712, Flamsteed had completed observations of approximately 3,000 stars. The positions were accurate to typically 10--20 arc-seconds, an order of magnitude improvement over Tycho's century-old catalog. More importantly, the observations were systematic, carefully recorded, and reducible to celestial coordinates using consistent methods.\footnote{The final \emph{Historia Coelestis Britannica} lists exactly 2,934 stars. This is discussed in detail in Chapter 5.}

\section{The Funding Crisis}
\label{sec:funding-crisis}

Yet throughout these decades, Flamsteed struggled against perpetual financial constraint. The \pounds100 salary was intended to be supplemented by various perquisites---income from livings, patronage grants, contributions from interested parties. These were unreliable.

In 1709, Flamsteed petitioned the Admiralty, noting that his salary had never been increased in the 34 years since his appointment, while prices had risen and his assistants' wages had increased.\footnote{\textcite{Baily1835}, p. 148--150, reproduces Flamsteed's petition. He notes that the cost of living had roughly doubled since 1675, while his income remained fixed.} He requested an increase to \pounds200. The request was denied.

More problematically, instrument maintenance and construction absorbed resources that he could not spare. When the great mural arc needed repair or the Octagon Room required structural work, Flamsteed often had to find private patrons or leave projects incomplete. This constraint shaped everything: which stars were observed, how often, and how thoroughly.

Yet Flamsteed's response was not to abandon rigor. Instead, he worked more methodically, sought support from patrons, and built a network of collaborators. Abraham Sharp remained his indispensable assistant. Later, Edmond Halley (despite their occasional tensions) and Isaac Newton (despite greater tensions still) recognized the importance of Flamsteed's work and advocated for its continuation.\footnote{The relationship between Flamsteed, Halley, and Newton is complex and has been much discussed by historians. \textcite{Willmoth1992} provides nuanced analysis; see also \textcite{Sobel1995}, Chapter 6, for a more narrative treatment.}

\section{An Institutional Foundation}
\label{sec:institutional-foundation}

What emerged by the early 18th century was something unprecedented in the history of astronomy: a national institution dedicated to systematic observation, funded (albeit inadequately) by the state, staffed by capable observers, and directed toward a specific practical aim---furnishing the accurate stellar positions required for solving longitude.

This was not Flamsteed's intention alone. It was the result of royal policy, the advocacy of Moore and others, the talent of Flamsteed and Sharp, and the persistence of a mission through decades of difficulty. The Observatory would outlast its founders. The catalog Flamsteed painstakingly built would become the foundation for celestial mechanics, aberration theory, and the lunar distance method for determining longitude.

More immediately, it established that astronomy could be an institutional endeavor, conducted over decades, requiring investment and discipline. The model would be replicated: at Paris, at Berlin, across Europe. Precision became a national enterprise. Greenwich became the reference point from which longitude and time were measured.\footnote{The modern definition of ``Greenwich Mean Time'' traces directly to the Observatory that Flamsteed founded. The Prime Meridian (zero degrees longitude) is defined by the meridian circle at Greenwich, a direct descendant of Flamsteed's work. See \textcite{Howse1980}, pp. 1--10, for the institutional history.}

The next chapter describes the methods and instruments that made this precision possible. We turn now from the founding to the practice: how the observations were actually made, and how raw stellar positions were reduced to the coordinates that would populate the catalog and answer the longitude question.
