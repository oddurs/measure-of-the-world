\chapter{The State of the Art in 1675}
\label{ch:state-of-art}

\section{The Instrument Maker's Workshop}
\label{sec:instrument-shop}

In a narrow shop on Cornhill or in Fetter Lane, a brass worker bent over a quadrant in the making. The metal curved in a perfect quarter-circle, its outer edge marked with a scale of degrees. The craftsman's tools were ancient: a ruler, a divider, a fine burin for engraving. His task was to divide the ninety degrees into smaller units---not all the way to minutes, mind you, but to enough subdivisions that a careful observer could read between the lines. The human eye, guided by steady hands and a lifetime of practice, was the measure of all instruments.

This was the state of the art in 1675, when John Flamsteed would establish the Observatory at Greenwich. No telescope yet aimed at the sky for precise measurement. No mechanical vernier scale. No screw micrometer. The best instruments the world could produce were the work of patient craftsmen---men like Henry Sutton or Walter Hayes---who combined ancient geometry with Renaissance precision and an almost monastic devotion to accuracy.

The ships were still lost. The navigation was still blind. And yet the instruments existed that could, in theory, provide the answer. The barrier was not knowledge but precision: the ceiling imposed by the hand with a burin and the eye with a naked pupil.

\section{Instruments for Finding Latitude}
\label{sec:latitude-instruments}

If the problem of longitude was hard, latitude's solution was almost elegant. The celestial pole marks the Earth's rotation axis, and its altitude above the horizon equals the observer's latitude. A navigator needed only to measure one angle.

\subsection{The Astrolabe}

The astrolabe was ancient technology, perfected in the Islamic Golden Age and transmitted to Europe through Al-Andalus in the medieval period. It was a disk of brass, beautifully engraved, projecting the celestial sphere onto a flat plate using stereographic projection. The projection was ingenious: it preserved angles, so that the celestial equator, the ecliptic, and the star positions could all be drawn accurately on the flat surface. An observer could hold the astrolabe at arm's length, sight along the alidade (a rotating ruler), and read the altitude of a star from a scale around the rim.

The principle was sound. The execution was limited. The engraved scales could not be divided more finely than the eye could read them---roughly to a degree, perhaps better in the hands of a master. More fundamentally, the astrolabe was delicate. The moving parts could jam. The projection introduced distortion near the poles. And the whole device, held aloft by the observer's hand, introduced motion and vibration that added its own error.

For all these reasons, by 1675, the astrolabe was becoming obsolete for serious observation. It remained a navigator's tool, valuable enough for rough determination of latitude, but its precision---perhaps a degree in untrained hands, half a degree in practiced ones---was insufficient for the astronomical work Flamsteed would undertake.

\subsection{The Cross-Staff and Backstaff}

Where the astrolabe measured angles with an engraved scale, simpler instruments used geometry. The cross-staff, or \textsc{Jacob's staff}, was nothing more than a wooden rod with a movable crosspiece at right angles. The observer would hold it at arm's length, position the crosspiece so that one end aligned with the horizon and the other with the Sun or a star, and then read the angle from the divisions marked along the rod.

The geometry was pure similar triangles. If the rod was held at a fixed distance from the eye, and the crosspiece was moved until its ends aligned with the two celestial objects, the angle between them could be read directly.

The problem was obvious: to measure the Sun's altitude, the observer had to stare into the Sun. John Davis's 1594 improvement, the \emph{backstaff}, solved this by working backward. The observer faced away from the Sun, using a shadow to fix the Sun's position while sighting a star or the horizon ahead. The geometry was more complex, requiring both a fore-staff and an aft-staff, but the result was a practicable method that protected the observer's eyes and, more importantly, improved accuracy.

By the 1670s, the backstaff was standard equipment aboard English ships. A skilled observer could achieve an accuracy of perhaps half a degree. It was simple, robust, and required no tools for repair at sea. For navigation, it was adequate.

\subsection{The Quadrant and the Vernier Scale}

For astronomical observation, the quadrant was the workhorse. It was a quarter-circle of wood or brass, its arc divided into ninety degrees and subdivided into smaller increments. The observer would sight along one arm, align the other with a star or the Sun's limb, and read the angle from the graduated scale.

The precision of the quadrant depended entirely on the fineness of the divisions. A quadrant divided to the nearest degree gave precision of $\pm 0.5\degree$. If the divisions could be made ten times finer---to one-tenth of a degree, or six arc-minutes---the precision might improve tenfold.

But the hand could only cut so fine. A division smaller than half a millimeter was nearly invisible to the naked eye. A craftsman working with a burin could produce divisions accurate to perhaps one-tenth of a millimeter, sufficient for about one arc-minute---the angular width of a grain of wheat held at arm's length.

The vernier scale, invented in 1631, offered a mechanical solution. It used two scales offset slightly from each other. The main scale was divided into larger units; a secondary scale, the vernier, had its divisions slightly compressed. By finding which mark on the vernier aligned with a mark on the main scale, an observer could read to a fraction of the main division. A good vernier could extend the readable precision of a quadrant to five or ten arc-seconds.

By 1675, vernier scales were known, but not yet universal. Tycho Brahe had built instruments that could read to perhaps a minute of arc. The best work was limited by the sharpness of the engraved lines and the resolving power of the human eye.

---

\section{The Astronomer's Toolkit}
\label{sec:existing-knowledge}

By the time the Observatory was founded, the celestial map was crude. Tycho Brahe, working in the late sixteenth century at his private observatory Uraniborg on the island of Hven, had created a star catalog of roughly one thousand bright stars, determining their positions with unprecedented care. His great mural quadrant, mounted on the wall of his instrument room, could read to about one arc-minute---extraordinary precision for naked-eye work.

But Tycho's catalog, compiled from observations spanning decades and published posthumously, was already thirty years in the past when Flamsteed arrived at Greenwich. More fundamentally, it was incomplete. The Southern Hemisphere was nearly blank. The positions of the brighter stars were known to perhaps one to two arc-minutes, which was often sufficient for navigation but inadequate for testing gravitational theories or predicting planetary motions with precision.

\subsection{The Telescope's Arrival}

The telescope, invented in 1608, changed the rules. Suddenly, the eye could see fainter stars, could measure the positions of Jupiter's moons with greater certainty, could resolve the disk of Saturn. But the telescope introduced its own errors. The lens introduced chromatic aberration. The narrow field of view made finding a star difficult. The motion of the observer's hand or the vibration of the instrument could introduce errors larger than the improvements the magnification provided.

Most critically, the telescope could measure angles no better than a well-made quadrant when the measuring device was used properly. The tube itself had no graduations; the observer still had to use an external measuring instrument or a reticle to determine the angular position.

\subsection{Clock Technology and the Pendulum}

For timekeeping, the best instruments before 1656 were foliot escapements---verge-and-foliot mechanisms that regulated the motion of a falling weight. They were crude by later standards, with errors of fifteen minutes or more per day. In 1656, Christiaan Huygens invented the pendulum clock, and accuracy improved by orders of magnitude.

The pendulum's period of oscillation depends only on its length and gravitational acceleration, not on its amplitude (for small angles). This made it nearly isochronous---each swing took the same time. Coupled with an escapement mechanism, a well-made pendulum clock could keep time to within ten or twenty seconds per day.

This was revolutionary for land-based astronomy. Flamsteed would use pendulum clocks to time his observations, to verify the motions of the stars, to test gravitational theories. The precision they offered transformed observational astronomy.

But there was a catch: they failed at sea. The motion of the ship, the vibration of the hull, the tilt of the deck---all of these disrupted the pendulum's regular swing. A pendulum clock aboard ship would lose minutes per hour, sometimes more. This is why the longitude problem could not be solved by simply putting an accurate clock in a ship and comparing it with time at a reference meridian.

---

\section{The Gap Between Need and Capability}
\label{sec:precision-requirements}

By 1675, the demands of navigation and astronomy had begun to outpace the capabilities of instruments. This gap would define the next hundred years.

\subsection{Latitude Determination}

For navigation, the determination of latitude was reliable. A backstaff could measure the altitude of the Sun or a star to within half a degree or better. This translated to a position error of perhaps \SI{30}{nm} at the equator, narrowing toward the poles. This was often adequate for warning of danger.

For astronomy, the requirements were stricter. Tycho's star catalog had positioned stars to within a minute or two of arc. Flamsteed's mission at Greenwich was to improve on this---to catalog the brighter stars to a precision of perhaps thirty arc-seconds, and to verify Tycho's positions for accuracy.

To achieve this precision, an instrument needed to read reliably to a few arc-seconds. This required either a vernier scale of exceptional finesse, or a micrometer screw that could be turned to track a star as it moved.

\subsection{Longitude by Lunar Distance}

For longitude by lunar distance, the precision requirements were extreme. The method depended on measuring the angular distance between the Moon and a known star with high precision, then looking up the time from a table. The tables came from theory and calculation, based on precise knowledge of the Moon's orbital position and the star positions.

If a star's position was uncertain to one arc-minute, this introduced an uncertainty of roughly one minute of time in the calculated longitude---equivalent to a fifteen-minute error in the final answer. At the latitude of Greenwich, this was a position error of several miles. For a ship at sea, this was marginal.

To make lunar distance reliable, the star positions needed to be known to better than thirty arc-seconds. This meant instruments that could read to better than thirty arc-seconds, and even then, observer error would introduce additional uncertainty.

The existing catalogs fell far short. Tycho's stars were good to a minute or two; some positions, especially of faint stars, were uncertain by several minutes. The gap between what the Moon's motion method needed and what the existing data could provide was a major obstacle.

---

\section{Precision and the Measuring Hand}
\label{sec:precision-limits}

All of this precision depended on the same bottleneck: the engraved scale and the human eye. A division finer than half a millimeter could not be seen. A line thinner than the eye's resolving power blurred into ambiguity.

This is why Flamsteed's great achievement would not be a new instrument, but a new method: the mural circle and the micrometer. By fixing an instrument in the meridian plane and using a telescope with a reticle to track a star, he could eliminate many sources of error. By using a micrometer screw to measure the small distances the eyepiece image moved as a star passed through the field, he could measure angles to precision hitherto unachieved.

But that lay ahead. In 1675, the best instruments the world could produce were the work of patient craftsmen, their precision limited by hand and eye, their capability sufficient for navigation but not yet adequate for the precision science that was emerging.

\begin{table}[htbp]
  \centering
  \caption{Precision of navigational and astronomical instruments, circa 1675.}
  \label{tab:instrument-precision}
  \small
  \begin{tabular}{lll}
    \toprule
    \textsc{Instrument} & \textsc{Precision} & \textsc{Notes} \\
    \midrule
    Astrolabe & $\pm 30'$ & Best case; scale limited \\
    Cross-staff & $\pm 20'$ & Trained hands \\
    Backstaff & $\pm 15'$ & Standard at sea \\
    Quadrant & $\pm 5'$ & Tycho-level \\
    Quadrant+vernier & $\pm 30''$ & Rare \& difficult \\
    Pendulum clock & $\pm 10\mathrm{s}$ & Land only \\
    Foliot clock & $\pm 15\mathrm{m}$ & Pre-pendulum \\
    \bottomrule
  \end{tabular}
\end{table}

---

\section{Forward to Methods and Instruments}
\label{sec:bridge-chapter-3}

The precision ceiling of 1675---perhaps a minute of arc with the best available instruments---was a reflection of the fundamental limits of hand-made scales and naked-eye observation. Within a few years, Flamsteed would push against this limit, designing new instruments and developing new methods. The quadrant with a micrometer eyepiece would emerge as the first truly modern astronomical instrument, capable of precision approaching ten arc-seconds.

But even that would not be enough. The next chapter describes the founding of the Observatory itself and the instruments Flamsteed began with. It is a story of ambition constrained by budget, of royal patronage competing with the persistent realities of craftsmanship and cost. \cref{ch:founding-observatory} takes us to Greenwich, where the real work began.
