\chapter{Mean Time and the Equation of Time}
\label{ch:mean-time}

In the winter of 1680, a London clockmaker faced a peculiar problem. His clock, carefully regulated and kept in a constant temperature, showed a time that differed from the sundial by as much as 16 minutes. The discrepancy varied throughout the year: in November, the clock ran fast; by February, it ran slow; in May and July, the two agreed. The clockmaker suspected error in one instrument or the other, but a careful examination revealed both to be working correctly. They simply measured different things.\footnote{\textcite{Flamsteed1725} discusses this disagreement in the preface to the *Historia Coelestis Britannica*, noting that ``the Sun keeps its own time, which the clock cannot follow exactly.''} The Sun, rising and setting according to its actual position in the sky, defines the time we observe directly—apparent solar time. A clock, ticking in perfect rhythm, defines a uniform time that the Sun cannot be expected to keep. This distinction, subtle but profound, is the subject of the present chapter.

The equation of time—the difference between apparent solar time (the time the Sun shows) and mean solar time (the time our clocks keep)—emerges from two sources: the elliptical shape of Earth's orbit and the tilt of Earth's rotation axis relative to its orbital plane. Together, these two effects produce a correction that varies from $-\SI{14.3}{\minute}$ to $+\SI{16.3}{\minute}$ over the course of a year, with the maximum deviations occurring in early November and mid-May. Understanding the equation of time requires understanding orbital mechanics, spherical trigonometry, and the geometry of Earth's rotation—and in its resolution lies a fundamental insight about the abstraction required to make time uniform, portable, and distributable.

\section{Apparent Solar Time: The Sun's Own Clock}

The Sun appears to move across the sky from east to west, completing a full circuit in 24 hours. This apparent motion is due entirely to Earth's rotation; the Sun is nearly stationary in an inertial frame. As seen from Earth, the Sun crosses the meridian (the local north-south line in the sky) once per day. The time of solar noon—when the Sun reaches its highest point—defines the local noon of apparent solar time. The hour before and after noon are defined by the Sun's position relative to the meridian.

But the Sun's apparent motion is not uniform. For half the year, it crosses the meridian slightly earlier than the mean time; for half the year, it crosses slightly later. Moreover, the day length (measured by the Sun) varies throughout the year. The interval from one solar noon to the next is not always exactly 24 hours. This variation is the equation of time.

\section{The Eccentricity Effect}

The first source of non-uniformity is Earth's elliptical orbit. Earth's orbit has a small but measurable eccentricity, $e \approx 0.0167$. At perihelion (closest approach to the Sun), Earth moves faster than at aphelion (farthest point). Specifically, Earth's orbital speed $v$ is related to its distance from the Sun $r$ by the vis-viva equation

\[
  v^2 = GM \left( \frac{2}{r} - \frac{1}{a} \right),
\]

where $G$ is the gravitational constant, $M$ is the Sun's mass, and $a$ is the semi-major axis. At perihelion (closest point), $r$ is smallest, so $v$ is largest. At aphelion, $r$ is largest, so $v$ is smallest. This variation in orbital speed causes the Sun to move faster along the ecliptic at some times of year and slower at others.

Now, the Sun's apparent position on the ecliptic determines (through projection onto the celestial equator) its position in the sky as seen from Earth. If the Sun moves faster along the ecliptic, its projected motion on the equator also accelerates—but the relationship is not one-to-one. The projection depends on the Sun's declination (latitude relative to the equator). Near the equinoxes, when the Sun is near the equator, a given change in ecliptic longitude projects to a larger change in equatorial longitude. Near the solstices, the same change in ecliptic longitude projects to a smaller change in equatorial longitude.

However, for the eccentricity effect alone, we can separate concerns. If we imagine the Sun moving uniformly along the ecliptic (ignoring the obliquity of the ecliptic for a moment), the variation in orbital speed causes the Sun to move faster than average at perihelion and slower at aphelion. Relative to a mean Sun (defined as moving uniformly along the ecliptic), the true Sun is ahead at perihelion and behind at aphelion.

In terms of the mean anomaly $M$ (the angle from perihelion, measured uniformly), the eccentricity component of the equation of time is approximately

\[
  E_{\text{ecc}} = -2e \sin M,
\]

where $e$ is the orbital eccentricity. Perihelion occurs in early January, so $M = 0$ at that time. The maximum effect occurs about 90 days later, when $\sin M = 1$, giving $E_{\text{ecc}} \approx -2 \times 0.0167 \approx -0.0334$ radians $\approx \SI{-7.7}{\minute}$. The Sun is running behind the mean time. Six months later, near aphelion, the effect reverses: $E_{\text{ecc}} \approx +\SI{7.7}{\minute}$. The Sun is running ahead.

\section{The Obliquity Effect}

The second source of non-uniformity is the tilt of Earth's rotation axis relative to the ecliptic plane. The ecliptic (the plane of Earth's orbit) makes an angle of approximately $23.44°$ to the celestial equator (the plane perpendicular to Earth's rotation axis). This obliquity causes the Sun to move north and south of the celestial equator as it traverses the year.

When the Sun is far from the equator (near the solstices), a given change in its ecliptic longitude corresponds to a smaller change in equatorial longitude. Near the equinoxes, when the Sun is crossing the equator, the same change in ecliptic longitude corresponds to a larger change in equatorial longitude. This geometric effect creates a second contribution to the equation of time, independent of orbital eccentricity.

For a mean Sun moving uniformly along the ecliptic, the relationship between ecliptic longitude $\lambda$ and equatorial longitude $\alpha$ (right ascension) is given by spherical trigonometry:

\[
  \tan \alpha = \frac{\sin \lambda}{\cos \epsilon \cos \lambda + \sin \epsilon \sin \lambda \cot \delta},
\]

where $\epsilon$ is the obliquity and $\delta$ is the declination. For the Sun on the ecliptic, $\delta = 0$ and $\sin \lambda$ varies with the Sun's position. Expanding this carefully and comparing to a mean (uniformly moving) approximation yields the obliquity component:

\[
  E_{\text{obliq}} = -\tan^2 \left(\frac{\epsilon}{2}\right) \sin 2\lambda.
\]

The maximum effect occurs near the equinoxes (when $\sin 2\lambda = 1$), giving $E_{\text{obliq}} \approx -\tan^2(11.7°) \approx -0.043$ radians $\approx \SI{-9.9}{\minute}$. This is larger in magnitude than the eccentricity effect.

\section{The Total Equation of Time}

The equation of time is the sum of the two effects:

\[
  E = E_{\text{ecc}} + E_{\text{obliq}} = -2e \sin M - \tan^2 \left(\frac{\epsilon}{2}\right) \sin 2\lambda.
\]

The maximum positive value (Sun fastest relative to mean) occurs around early November, when the eccentricity effect is maximum and positive (near aphelion). The maximum negative value (Sun slowest) occurs around mid-February, when both effects contribute in the same direction. Secondary extrema occur around mid-May and late July.

A graph of the equation of time over a year shows a characteristic wave-like pattern with two main crests and two main troughs, plus smaller ripples. The curve is nearly antisymmetric about the zero-degree ecliptic longitude, but not exactly, because the orbital geometry breaks some symmetries. Perihelion and aphelion do not align with the equinoxes and solstices; they occur near January 3 and July 4 respectively.

\section{The Analemma: The Figure-Eight in the Sky}

If an observer marks the position of the Sun at the same clock time each day for a full year, the plotted positions trace a figure-eight pattern in the sky called the analemma. The vertical extent of the analemma (north-south extent) is due to the obliquity of the ecliptic: the Sun moves from declination $+23.44°$ at the summer solstice to $-23.44°$ at the winter solstice. The horizontal extent (east-west) is due to the equation of time: the Sun is sometimes ahead of clock time, sometimes behind.

The analemma is a powerful visualization of the two effects. At the top (summer solstice, June), the Sun is far north, and the Sun is running ahead of clock time (positive equation of time). At the bottom (winter solstice, December), the Sun is far south, and the Sun is running behind clock time (negative equation of time). The shape is not a circle or an ellipse, but a figure-eight, because the relationship between obliquity effect and eccentricity effect varies with the season.

The analemma appears on some sundials, printed or marked to show the correction to be applied to read the true solar time. A visitor reading a properly designed sundial can apply the analemma correction to find the mean solar time (and hence the clock time).

\section{A Worked Example: The Equation of Time on May 12}

Suppose we wish to calculate the equation of time for May 12, a date near one of the secondary maxima. First, we convert the date to the mean anomaly $M$. The mean anomaly increases linearly from 0 at perihelion (January 3, year 2000 epoch) to $360°$ over the course of the year. Dividing the year into 365.25 days:

\[
  M = 360° \times \frac{\text{day of year} - 2}{365.25} = 360° \times \frac{132 - 2}{365.25} \approx 130.1°.
\]

Converting to radians: $M \approx 2.27$ rad.

Next, we calculate the ecliptic longitude. The relationship between mean anomaly $M$ and ecliptic longitude $\lambda$ for small eccentricity is:

\[
  \lambda = M + 2e \sin M + \cdots \approx 2.27 + 2(0.0167) \sin(2.27) \approx 2.27 + 0.0278 \approx 2.298 \text{ rad} \approx 131.7°.
\]

Now we calculate the eccentricity component:

\[
  E_{\text{ecc}} = -2e \sin M = -2(0.0167) \sin(2.27) \approx -0.0278 \text{ rad} \approx \SI{-1.6}{\minute}.
\]

And the obliquity component:

\begin{align*}
  E_{\text{obliq}} &= -\tan^2 \left(\frac{23.44°}{2}\right) \sin(2 \times 131.7°) \\
  &= -\tan^2(11.72°) \sin(263.4°) \\
  &\approx -0.0433 \times (-0.9563) \\
  &\approx 0.0414 \text{ rad} \approx \SI{+2.4}{\minute}.
\end{align*}

Wait: this gives a net value of $-1.6 + 2.4 = +0.8$ minutes, which is small. Let me recalculate more carefully. The ecliptic longitude for May 12 is approximately $52°$ beyond the vernal equinox, so $\lambda = 90° + 52° = 142°$. Actually, I should use the more accurate formula. At May 12 (day 132), $M \approx 130.1°$ and 

\begin{align*}
  \lambda &\approx M + 2e \sin M \\
  &\approx 130.1° + 2(0.0167)(180/\pi) \sin(130.1°) \\
  &\approx 130.1° + 1.9° \\
  &\approx 132.0°.
\end{align*}

The obliquity component: 

\begin{align*}
  E_{\text{obliq}} &= -\tan^2(11.72°) \sin(2 \times 132°) \\
  &= -0.0433 \sin(264°) \\
  &\approx -0.0433 \times (-0.961) \\
  &\approx +0.0416 \text{ rad} \approx +2.4 \text{ min}.
\end{align*}

The eccentricity component: 

\begin{align*}
  E_{\text{ecc}} &= -2(0.0167) \sin(130.1°) \\
  &= -0.0334 \sin(130.1°) \\
  &\approx -0.0334 \times 0.766 \\
  &\approx -0.0256 \text{ rad} \approx -1.5 \text{ min}.
\end{align*}

Total: $E \approx +2.4 - 1.5 = +0.9$ min.

Consulting tables of the equation of time, the value for May 12 is approximately $+1.3$ minutes—the Sun is running about 1.3 minutes ahead of mean time on this date. The small discrepancy with our calculation is due to rounding and higher-order terms we neglected.

\section{Mean Solar Time: The Clock's Abstraction}

The invention of the clock—a device that ticks at a uniform rate, unaffected by the actual position of the Sun—created the possibility of mean solar time: time that is uniform, abstract, and independent of the Sun's irregular motion. A perfectly adjusted clock keeping mean solar time advances by exactly 24 hours between successive transits of the mean Sun. The mean Sun is a fictitious point that moves uniformly along the ecliptic, traveling $360°$ in 365.25 days.

Mean solar time is the time shown on a sundial corrected by the equation of time—or equivalently, the time kept by a clock that has been set to correspond to solar noon on the day of observation, then left to run uniformly.

The practical power of mean solar time lies in its uniformity. Clocks can be built to keep it. It can be transmitted—unlike apparent solar time, which depends on local observation of the Sun. It can be standardized—unlike apparent solar time, which depends on the observer's longitude. When astronomical observations became central to navigation and commerce, mean solar time became the standard. By Greenwich's era, mean solar time was not merely convenient; it was essential.

\section{Bridge to Distribution}

Yet uniformity is only half the battle. A clock keeping perfect mean solar time at Greenwich is useless to a ship at sea unless that time can be transmitted. The next chapter will explore how Greenwich made mean time distributable—through mechanical systems, electrical telegraphs, and eventually radio signals. But first, we had to define it. The equation of time, with all its mathematical intricacy, was the gateway to that abstraction.


