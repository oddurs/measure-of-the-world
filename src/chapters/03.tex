\chapter{The Founding of the Observatory}
\label{ch:founding-observatory}

\section{The Appointment}
\label{sec:flamsteed-appointment}

In March 1675, John Flamsteed, aged twenty-eight, received a letter from Jonas Moore, the Surveyor-General of the Ordnance. It contained an offer that would consume the rest of his life. He was to be appointed ``astronomical observator'' of England---a title never before used, for the office had never existed. His salary would be one hundred pounds per annum, paid from the royal purse. He would observe the heavens from an observatory yet to be built, in a park yet to be selected, using instruments yet to be designed. The only certainties were that the nation demanded it and the work was impossible.

Flamsteed, self-taught, from Derby, had long corresponded with Moore and with others in the growing circle of natural philosophers---Hooke, Boyle, Oldenburg at the Royal Society. His observational notebooks from the 1670s were already legendary in this small world: precise measurements of stellar positions, lunar eclipses, planetary motions, recorded night after night with an intensity that suggested obsession. Now he was being asked to institutionalize that obsession.

He accepted immediately.

\section{The Political Foundation}
\label{sec:political-context}

The founding of the Royal Observatory was not an act of curiosity. It was an act of naval strategy.

England and France had been rivals in every domain since Cromwell's death and the Restoration. When Louis XIV established the Observatoire de Paris in 1667, with Cassini and Picard and the finest instruments that French money could buy, the English court took notice. The French were not building observatories for idle stargazing. They were building them to understand the heavens, to navigate the oceans, to dominate trade. If France possessed the mathematical mastery of the cosmos, it would possess the oceans.

The problem that Flamsteed was meant to solve was this: to create an accurate catalog of the fixed stars, precise enough to serve as a reference for all future navigation. The Tycho Brahe catalog, published nearly a century earlier, was the standard. But Tycho's measurements, though legendary, had errors. Some stars were misplaced by several arc-minutes. For navigation at sea, where a single minute of arc in declination could mean the difference between a safe passage and a wreck, even Tycho's accuracy was insufficient.

Jonas Moore, Flamsteed's patron, understood this perfectly. Moore had served Cromwell and remained powerful under Charles II. He had overseen the construction of coastal fortifications, surveyed England, understood precision as a form of power. He saw the astronomical problem as analogous to the surveying problem: one must establish fixed reference points, measure from them, build a network of locations of absolute certainty. From such a network, everything else could be derived.

Charles II approved the proposal. On 22 June 1675, the royal warrant was issued. An observatory would be built. Flamsteed would be its director.

\section{Greenwich Park}
\label{sec:site-selection}

The choice of location was pragmatic. Greenwich lay east of London, beyond the Thames, on crown land. The old castle stood there, its grounds suitable for renovation. It was far enough from the smoke and dust of the city that the sky remained relatively clear on decent nights. The magnetic compass worked better there than in the iron-rich soils around some alternative locations. The site commanded a view to the north and south, with a reasonably unobstructed horizon toward the meridian.

But Greenwich was not ideal. It was damp. The Thames was nearby, bringing moisture off the water. The marshes extended toward Essex. In winter, fog rolled up from the river and hung for days, rendering the sky opaque. Flamsteed would curse his isolation frequently in his later years---too far from London society, too close to the weather.

Still, it was the royal choice. Work began in 1675.

\section{Wren's Octagon}
\label{sec:architecture}

Christopher Wren, at the peak of his power as Surveyor of the King's Works, received the commission to design the building. His brief was sparse: create a house with a room suitable for astronomical observation, provide mounting points for instruments, keep the cost below five hundred pounds.

The budget was absurd. Five hundred pounds was enough for a modest house, not an instrument platform. But Wren was resourceful. He designed a compact, efficient structure: Flamsteed House, as it would be called. The principal room was octagonal, on the west side of the building, with windows oriented to the cardinal directions. The idea was that telescopes could be mounted in the window frames, pointing north, south, east, and west, capturing observations throughout the night.

The octagon was too decorative for serious observation. Wren's aesthetics prevailed over functional necessity. The room was meant to be elegant, to impress the king, to look like a place where natural philosophy happened rather than a functional instrument platform. But Flamsteed accepted it. He would work with what he was given.

The building was completed in 1676, a year after the warrant. It was small, solidly constructed, somewhat damp. Flamsteed arrived and looked at the octagon room. He saw its flaws immediately. But he also saw its possibilities. And he had already begun planning something more ambitious: an instrument of his own design that would become the foundation of all his work.

\section{The Initial Suite}
\label{sec:instruments}

Moore had given Flamsteed two large clocks, made by Thomas Tompion, the finest clock maker in England. These were long-case clocks with pendula---relatively new technology, having been invented by Christian Huygens only years before. Pendulum clocks were far more accurate than the foliots and escapements that had preceded them. Accuracy to within a few seconds per day was possible. For astronomical observation, where the time of a star's passage across the meridian was critical, such clocks were transformative.

Moore also provided a sextant---a large instrument of nearly seven feet radius, capable of measuring angles with modest precision. It was built along traditional lines, descended from instruments that Tycho Brahe himself had used.

But Flamsteed had grander ambitions. In the months before arriving at Greenwich, he had designed an instrument that would become his trademark: a mural arc. This was a great arc, graduated and mounted vertically against the meridian wall of the octagon room. It would be nearly 140 degrees in extent, with a radius of nearly seven feet. Its face would be carefully divided into single degrees, with subdivisions allowing readings to within a few arc-minutes.

The mural arc would be Flamsteed's principal instrument for thirty years. From it would come the bulk of the observations that fed his great catalog. But it had to be built, and building it required skill, resources, and a craftsman capable of dividing the graduated scale with sufficient accuracy.

Flamsteed found that craftsman in Abraham Sharp, an Yorkshire instrument maker and mathematician who had already built some of the nation's finest instruments. They began work on the arc in 1689, fourteen years into Flamsteed's tenure. The arc was completed two years later and immediately put to use.

\section{The Underfunded Astronomer}
\label{sec:funding-crisis}

One hundred pounds per annum. From this, Flamsteed had to survive. He had to pay for an assistant. He had to contribute to instrument costs from his own pocket. He had to buy paper, ink, candles for the long observational nights, coal for the fires in winter. He had to maintain the buildings and instruments.

The salary was barely livable. Contemporary records suggest that a skilled tradesman earned roughly the same amount. Flamsteed had no inheritance, no independent means. He had accepted the post out of conviction that the work was essential, that he would somehow manage. He did manage, but with constant anxiety about money and with repeated requests to the government for additional funding that mostly went unanswered.

This penury shaped everything that followed. It meant that Flamsteed could not hire the best assistants, only those he could afford. It meant that instrument improvements happened slowly, driven by ingenuity rather than resources. It meant that the work of reduction---converting raw observations into useful catalogs, writing the preface and introduction to his great work---happened largely at night, after observing sessions, with minimal support.

Yet this very constraint may have sharpened his focus. With limited resources, every observation had to matter. With limited staff, he had to do much of the work himself, which meant he understood every piece of it intimately. When he published his results, they bore the mark of a man who had paid for every measurement in his own labor.

The Observatory was designed as a state investment in navigational infrastructure. But it was starved, perpetually, of the resources that would have made it comfortable. Flamsteed worked in this scarcity for fifty-four years, from 1675 until his death in 1719. When he died, the catalog was not yet published. It would take another generation to complete what he had begun.

\begin{table}[htbp]
  \centering
  \caption{Greenwich Observatory: founding facts, 1675.}
  \label{tab:founding-facts}
  \small
  \begin{tabular}{ll}
    \toprule
    \textbf{Fact} & \textbf{Value} \\
    \midrule
    Royal warrant issued & 22 June 1675 \\
    Flamsteed appointed & March 1675 \\
    Flamsteed's age & 28 years \\
    Annual salary & £100 \\
    Building budget & £500 \\
    Site: location & Greenwich Park, London \\
    Architect & Christopher Wren \\
    Primary room & Octagon (8-sided) \\
    Initial major instruments & 2 Tompion clocks, 1 sextant \\
    Mural arc built & 1689--1691 \\
    Mural arc radius & $\approx 7$ feet \\
    Arc extent & $\approx 140\degree$ \\
    \bottomrule
  \end{tabular}
\end{table}

---

The Observatory had been founded. The astronomer had arrived. The work of decades lay ahead: nights of observation, calculations that would consume thousands of hours, a catalog that would eventually contain the positions of nearly 3,000 stars, measured and remeasured, corrected and refined. But all of that depended on a single instrument and a method of observation that was revolutionary for its time. The next chapter turns to that instrument---the mural arc---and to the systematic method by which Flamsteed would transform raw angular measurements into the most accurate celestial map that science had yet produced. \cref{ch:mural-arc-transits} describes how he did it.
